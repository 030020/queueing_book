\section
[$M/M/1$ queue]
{$\mathbf{M/M/1}$ queue}
\label{sec:mm1}

In the $M/M/1$ queue, one server serves jobs arriving with
exponentially distributed interarrival times and each requiring an
exponentially distributed processing time.  With Eq.~(\ref{eq:25}),
i.e., $\lambda(n)p(n)= \mu(n+1)p(n+1)$ we can derive a number of
important results for this queue.

For the $M/M/1$ queue it is reasonable to assume that
$\lambda(n) = \lambda$ and $\mu(n)= \mu$, so that from \eqref{eq:25},
\begin{equation*}
  p(n+1) = \frac{\lambda(n)}{\mu(n+1)} p(n) = \frac{\lambda}{\mu} p(n) = \rho p(n),
\end{equation*}
as $\rho=\lambda/\mu$. Since this holds for any $n\geq 0$, it follows with
recursion that
\begin{equation*}
  p(n+1) = \rho^{n+1} p(0).
\end{equation*}
Then, from the normalization condition
\begin{equation*}
1=  \sum_{n=0}^\infty p(n) = p(0)\sum_{n=0}^\infty \rho^n = \frac{p(0)}{1-\rho},
\end{equation*}
so 
\begin{align}\label{eq:23}
p(0) &=1-\rho, &   p(n) &=  (1-\rho)\rho^{n}.
\end{align}

How can we use these equations? First, note that $p(0)$ must be the
fraction of time the server is idle. Hence, the fraction of time the
server is busy, i.e., the utilization, is
\begin{equation*}
  1-p(0) = \rho = \sum_{n=1}^\infty p(n).
\end{equation*}
Here the last equation has the interpretation of the fraction of time
the system contains at least 1 job. Next, in the exercises below we
derive that
\begin{equation}\label{eq:el}
  \E L = \frac \rho{1-\rho}.
\end{equation}
and 
\begin{equation}\label{eq:pn}
  \P{L\geq n} = \rho^n.
\end{equation}

Let us interpret these expressions. First of all, we only need
estimates of $\lambda$ and $\mu$ to characterize the utilization of
the server, the average queue length and the queue length
distribution. In fact, only the ratio $\rho=\lambda/\mu$ is required.
As such, assuming that interarrival times and services are
exponentially distribtued, by measuring the fraction of time the
server is busy, we obtain an estimate for $1-p(0)$, hence for $\rho$.
With this, we can estimate $\E L$ and $\P{L\geq n}$.  Next, the fact
that $\E L \sim (1-\rho)^{-1}$ for $\rho\to 1$ implies that the
average waiting time \recall{increases asymptotically fast} to
infinity when $\rho\to1$.  In practical terms, this means that
striving for a high load results in (very) long average waiting
times. As a consequence, this formula tells us that, in the design and
operation of queueing systems, we should avoid the situation in which
$\rho\approx 1$. Thus, we need to make a trade-off between the server
utilization $1-p(0) = \rho$ and the waiting time. In high load
regimes, the server is used efficiently, but the queue lengths are
excessive. In any sensible system, we should keep $\rho$ quite a bit
below 1, and accept lower utilization then might seem necessary.

The expression for $\P{L\geq n}$ shows that the probability that the
queue length exceeds some threshold decreases exponentially fast (for
any reasonable $\rho$). Moreover, if we make the simple assumption
that customers decide to leave (or rather, not join) the system when
the queue is longer than $9$ say, then $\P{L\geq 10} = \rho^{10}$ is
an estimator of the fraction of customers lost. Again, for the $M/M/1$
queue, the knowledge of $\rho$ suffices to compute this estimate.

In the context of inventory theory these equations are particularly useful, see Question~\ref{q:basestock}.

\begin{question}
  Derive expression~\ref{eq:el} for the average number of jobs in an
  $M/M/1$ queue.  
  \begin{solution}
    \begin{equation*}
      \begin{split}
\E L 
&= \sum_{n=0}^\infty n\pi(n) 
= \sum_{n=0}^\infty \sum_{i=1}^n 1\{i\leq n\} \pi(n) \\        
&= \sum_{i=1}^\infty \sum_{n=0}^\infty  1\{i\leq n\} \pi(n) 
= \sum_{i=1}^\infty \sum_{n=i}^\infty \pi(n) \\
&= \sum_{i=1}^\infty \sum_{n=i}^\infty (1-\rho)\rho^n         
= \sum_{i=1}^\infty (1-\rho)\rho^i \sum_{n=0}^\infty \rho^n \\
&= \sum_{i=1}^\infty (1-\rho)\rho^i \frac1{1-\rho}   
= \sum_{i=1}^\infty \rho^i = \rho \sum_{i=0}^\infty \rho^i = \frac{\rho}{1-\rho}.
      \end{split}
    \end{equation*}
  \end{solution}
\end{question}

\begin{question}
  Derive expression~\ref{eq:pn}.
  \begin{solution}
    \begin{equation*}
      \begin{split}
\P{L\geq n} 
&= \sum_{k=n}^\infty p(n) = \sum_{k=n}^\infty p(0)\rho^n = (1-\rho)\sum_{k=n}^\infty \rho^k \\
&= (1-\rho)\rho^n \sum_{k=0}^\infty\rho^k = \rho^n \sum_{k=0}^\infty(1-\rho) \rho^k = \rho^n \sum_{k=0}^\infty p(k)  = \rho^n.
\end{split}
\end{equation*}
\end{solution}
\end{question}

\begin{question}
\label{q:basestock}
Customers of, for example, a fast-food restaurant or a production
facility, prefer to be served from stock. For this reason such
companies often use a `produce-up-to' policy: When the on-hand
inventory $I$ is equal or lower than some threshold $r$, the company
produces items until the inventory level equals $r+1$ again. The level
$r$ is the known as the reorder level.

Suppose that customers arrive as a Poisson process with rate $\lambda$
and the production times of single items are i.i.d. and exponentially
distributed with parameter $\mu$. Assume also that customers that
cannot be served from on-hand stock are backlogged, that is, they wait
until their item has been produced. What are the average on-hand
inventory level, the average number of customer in backlog, and the
fraction of customers that are backlogged?

If $h$ is the cost per item per unit time in stock and $b$ the cost
per customer per unit time in backlog and $\pi$ the cost per customer
backlogged, what are the average costs of using a reorder level $r$? 

What is the optimal reorder level $r^*$, i.e., the level that achieves
minimal average cost?

\hint{Realize that the inventory level $I(t)$ at time $t$ can be
  modeled as $I(t) = r+1-L(t)$, where $L$ is the number of jobs in an
  $M/M/1$ queue.}

  \begin{solution}
\tbd
  \end{solution}
\end{question}

%%% Local Variables:
%%% mode: latex
%%% TeX-master: "book"
%%% End:
