\section
[A Relation Between the $M^X/M/1$ and $M/G/1$  Queue]
{A Relation Between the $\mathbf{M^X/M/1}$ and $\mathbf{M/G/1}$  Queue}
\label{sec:relat-batch-queu}


\subsection*{Theory and Exercises}

\Opensolutionfile{hint}
\Opensolutionfile{ans}


There is an interesting relation between the $M^X/M/1$ and the $M/G/1$
queueing systems. As an example, consider the $M^{10}/M/1$ queue in
which each job corresponds to the arrival of ten items of
work. Intuitively, it is reasonable that this queueing process is
quite similar to an $M/G/1$ queue in which each job requires
approximately 10 time units. Below we will make this relation
precise. We will also use this to find another derivation of the
Pollackzek-Khintchine equation, if we consider a certain limiting
case.


The basic observation is that by discretizing the service time $S$ of
a job in an $M/G/1$ queue we can approximate $S$ by the service time
of a batch of small units of work. To make this more specific, assume
that the distribution of the service time $S$ is given. Use a grid of
size $1/n$ and take
\begin{equation}\label{eq:69}
  \P{B^{(n)} = i} = \P{ S\in\left(\frac in, \frac{i+1}n\right]}
\end{equation}
as the distribution of the batch size $B^{(n)}$. Let the service time
of one unit of the batch be exponentially distributed with mean $1/n$.
Thus, if $n$ becomes larger, the service time $S$ is represented by
ever larger batches, but such that the service time of each unit
becomes ever smaller. These arguments can be made precise in a
probabilistic sense, but here we rely on our intuition to conclude
that, in some sense, $B^{(n)}/n \to S$\footnote{For the interested, this is convergence in distribution here.}.

\begin{exercise}
Why do we consider $\P{B^{(n)} = i} = \P{ S\in(i/n, (i+1)/n]}$ rather than
\begin{equation*}
\P{B^{(n)} = i} = \P{S\in[i/n, (i+1)/n]},  
\end{equation*}
i.e., we use a half open interval $(,]$ rather than the closed
interval $[,]$?
\begin{solution}
  Starting from the distribution function $\P{S\leq x}$, we define
  \begin{equation*}
  \P{B^{(n)} = i} = \P{ S\leq (i+1)/n} - \P{S\leq i/n} = \P{
  S\in(i/n, (i+1)/n]}.
     \end{equation*}
\end{solution}
\end{exercise}


Let us see how this limit can be used to show that the expected
waiting in queue for the batch queue leads to the
Pollackzek-Khintchine equation, i.e., the expected waiting time in
queue for the $M/G/1$ queue.  Writing $\E{L^{(n)}}$ for the batch
queue with grid size $1/n$, we use~\eqref{eq:43} to see that
\begin{align}\label{eq:1pp}
\frac{\E{L^{(n)}}}{n}
= \frac{\rho}{1-\rho}\cdot \frac12\left(1 + \frac{\V{B^{(n)}}}{(\E{B^{(n)}})^2}\right) \frac{\E{B^{(n)}}}n +  \frac1{2n}\frac{\rho}{1-\rho}.
\end{align}
By assumption\footnote{ Here we interchange limits and expectations. A
  formal approach should pay due attention to the validity of all
  these interchanges.}:
\begin{align*}
\frac{B^{(n)}}n &\to S \\
\intertext{so that must be that
}
\E{\frac{B^{(n)}}n} &\to \E S \\
\intertext{and}
\E{\frac{(B^{(n)})^2}{n^2}} &\to \E{S^2},
\end{align*}
as $n\to \infty$.  
Hence, 
\begin{equation*}
  \begin{split}
\frac{\V{B^{(n)}}}{n^2} 
&=  
\frac{\E{(B^{(n)})^2} - (\E{B^{(n)}})^2}{n^2} = 
\E{\frac{(B^{(n)})^2}{n^2}} - \left(\E{\frac{B^{(n)}}{n}}\right)^2 \\
&\to \E{S^2} - (\E{S})^2  =  \V{S}.
  \end{split}
\end{equation*}
from which
\begin{equation*}
\frac{\V{B^{(n)}}} {(\E{B^{(n)}})^2} = \frac{\V{B^{(n)}}}{n^2}\frac{n^2}{(\E{B^{(n)}})^2} \to 
\frac{\V S}{(\E S)^2} = C_s^2,
\end{equation*}
as $n\to \infty$.  Moreover, we \emph{expect} that
\begin{equation*}
\frac{\E{L^{(n)}}}n \to \E{W_Q},
\end{equation*}
since the number of items in the system, $\E{L^{(n)}}$, becomes ever
larger as $n$ increases, but the service time of one item is $1/n$,
which becomes ever smaller. Now, taking the limit $n\to\infty$
in~\eqref{eq:1pp} we obtain
\begin{equation*}
  \frac{\E{L^{(n)}}}{n}
  \to \frac{\rho}{1-\rho}  \frac{1 + C_s^2}2\E{S}.
\end{equation*}
Thus, based on this  limit we expect that 
\begin{equation*}
\E{W_Q}=  \frac{\E{L^{(n)}}}{n}
  \to \frac{\rho}{1-\rho}  \frac{1 + C_s^2}2\E{S},
\end{equation*}
but this is the same as Eq.~\eqref{eq:710}!

\begin{exercise}
  Why do we consider $\E{L^{(n)}}/n \to \E{W_Q}$, and not
  $\E{L^{(n)}}/n \to \E W$? 
  \begin{hint}
What is the expected service time of one
  unit of a job in the $M^X/M/1$ queue in the limiting case?
  \end{hint}
  \begin{solution}
    The time it takes to serve one unit is $1/n$. Thus, the fraction
    of time an item spends in service in the batch queue becomes negligible compared
    to the time in queue. Another way to see this, is to use 
    \begin{equation*}
      \frac{\E{L^{(n)}}}n = 
      \frac{\E{L^{(n)}_Q}}n + \frac{\E{L^{(n)}_s}}n.
    \end{equation*}
Since $L^{(n)}_s \leq 1$, $\E{L^{(n)}}/n \approx \E{L^{(n)}_Q}/n$ for $n$ large.
  \end{solution}
\end{exercise}

\begin{comment}
  
\begin{exercise}[use=false]
Can you model the $M^X/M/1$ as an $M/G/1$ queue, thereby deriving Eq.~(\ref{eq:43}) from Eq.~(\ref{eq:710})? 
  \begin{solution}
I guess this is a bit hard, so I include a partial answer for the interested. 


Starting from the $M/G/1$ queue, suppose first that the service time
$S$ is an exponential random variable with mean $1/\mu$.  Then the
$M/G/1$ reduces to the $M/M/1$ queue, which is evidently equal to
the $M^1/M/1$ queue in which the jobs arrive as single units. Next, if
$S$ consists of the sum of $k$ i.i.d. exponential random variables
$S_i$ with mean $1/\mu$, i.e., $S=\sum_{i=1}^k S_i$, then the
$M^k/M/1$ queue results. Generalizing still further, we can take a
random batch $B$ of i.i.d. exponentials so that $S = \sum_{i=1}^B S_i$. Now
\begin{equation*}
  \P{S\leq x}
= \sum_{k=1}^\infty \P{S \leq x\given B=k}\P{B=k}
= \sum_{k=1}^\infty f(k) \P{\sum_{i=1}^B X_i \leq x}
\end{equation*}
where $f(k)=\P{B=k}$, and we obtain the $M^X/M/1$ from the $M/G/1$
queue by taking this distribution function for the service times.
  \end{solution}
\end{exercise}
\end{comment}

% the distribution $F$ of the service times of the $M/G/1$ on a grid of size $1/n$, like so:
% \begin{equation*}
%   F^{(n)}(x) = F(\lceil nx \rceil/n). 
% \end{equation*}
% Then $F^{(n)}(x) \downarrow F(x)$ for all $x$; recall that $F(x)$ is a
% distribution function, i.e., right continuous, non-decreasing, and such
% that $F(0) = 0$ and $F(\infty) = 1$. 







\Closesolutionfile{hint}
\Closesolutionfile{ans}

\opt{solutionfiles}{
\subsection*{Hints}
\input{hint}
\subsection*{Solutions}
\input{ans}
}
%\clearpage
%%% Local Variables:
%%% mode: latex
%%% TeX-master: "../book"
%%% End:
