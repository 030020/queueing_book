\section{Some Useful Identities}
\label{sec:some-usef-ident}

With the PASTA property and Little's law we can derive a number of
useful and simple results for the $M/G/1$ queue. Recall, to use the
PASTA, we need to assume that jobs arrive as a Poisson process.

The fraction of time the server is empty is $1-\rho = p(0)$. By PASTA,
$\pi(0)=p(0)$, hence the fraction of customers that enter an empty
system is also $1-\rho$. 

Suppose that at $A_k$, i.e., the arrival epoch of the $k$th job, the
server is busy.  The remaining service time $S_{r,k}$ as seen by job
$k$ is the time between $A_k$ and the departure epoch of the job in
service. If the server is free at $A_k$, we set $S_{r,k}=0$.  Define
the average \recall{remaining service time} as seen at arrival epochs by
the limit
\begin{equation*}
  \E{S_r} = \lim_{n\to\infty} \frac1n \sum_{k=1}^n S_{r,k},
\end{equation*}
provided this limit exists. Note that $\E{S_r}$ includes the fraction
of jobs that find the server idle. If we need the remaining service
time for the jobs that see the server busy, use that a fraction~$\rho$
sees the server occupied upon arrival, while a fraction $1-\rho$ sees
a free server. Therefore
\begin{equation}\label{eq:37}
\E{S_r} =   \rho \E{S_r\given S_r >0} + (1-\rho)\cdot \E{S_r\given S_r = 0} = \rho \E{S_r\given S_r>0},
\end{equation}
since, evidently, $\E{S_r\given S_r=0}=0$.  Observe that we used the
PASTA property here.

Next, consider the waiting time in queue.  It is evident that the
expected waiting time for an arriving customer is the expected
remaining service time plus the expected time in queue. The expected
time in queue, in turn, must be equal to the expected number of
customers in queue at an arrival epoch times the expected service time
per customer, assuming that service times are i.i.d. If the arrival
process is Poisson, it follows from PASTA that the average number of
jobs in queue perceived by arriving customers is also the
\emph{time-average} number of jobs in queue~$\E{L_Q}$.  Thus, the
expected time in queue is
\begin{equation}\label{eq:24}
  \E{W_Q} = \E{S_r} + \E{L_Q} \E S.
\end{equation}
Now, from Little's law we know that $\E{L_Q} = \lambda \E{W_Q}$. Using this,
\begin{equation*}
  \E{W_Q} = \E{S_r} + \lambda \E{W_Q} \E S  =\E{S_r} + \rho \E{W_Q},
\end{equation*}
since $\rho=\lambda \E S$. But this gives for the $M/G/1$ queue that
\begin{equation}\label{eq:35}
  \E{W_Q} = \frac{\E{S_r}}{1-\rho} = \frac{\rho}{1-\rho} \E{S_r\given S_r>0},
\end{equation}
where we use~(\ref{eq:37}) in the last equation.


The average waiting time $\E W$ in the entire system, i.e., in queue
plus in service, becomes
\begin{equation*}
  \E W = \E{W_Q}+ \E S = \frac{\E{S_r}}{1-\rho} + \E S.
\end{equation*}

The situation can be significantly simplified for the $M/M/1$ queue as
then the service times are also exponential, hence memoryless,
implying that $\E{S_r\given S_r>0} = \E S$. Thus, for the $M/M/1$ queue,
\begin{equation*}
  \E W = \E{W_Q}+ \E S = \frac{\E S}{1-\rho}.
\end{equation*}


Another way to derive the above result is to conclude from PASTA that
the expected number of jobs in the system at an arrival is $\E{L}$.
Since all these jobs require an expected service time $\E S$,
including the job in service by the memory-less property, the time in
queue is $\E L \E S$. The time in the system is then the waiting time
plus the service time, hence,
\begin{equation}\label{eq:61}
  \E W = \E L  \E S + \E S = \lambda \E W \E S + \E S = \rho \E W  + \E S,
\end{equation}
where we use Little's law $\E L = \lambda \E W$.  Hence,
\begin{equation*}
\E W = \frac{\E S}{1-\rho}.
\end{equation*}
Also,
\begin{equation}\label{eq:wqes}
  \E{W_Q} = \E L \E S = \lambda \E W \E S= \rho \E W = \frac{\rho}{1-\rho} \E S = \frac{\rho^2}{1-\rho} \frac 1 \lambda,
\end{equation}
which is consistent with our earlier result.  

For the average queue length we use Little's law. Then
\begin{equation*}
  \begin{split}
\E L &= \lambda \E W = \frac{\lambda \E S}{1-\rho} = \frac\rho{1-\rho}, \\
  \E{L_Q} &= \lambda \E{W_Q} = \frac{\rho^2}{1-\rho}.
      \end{split}
\end{equation*}
Finally, the expected number of jobs in service $\E{L_s}$, which is
equal to the expected number of busy servers, must be
\begin{equation*}
  \E{L_s} = \E L - \E{L_Q} = \frac{\rho}{1-\rho} - \frac{\rho^2}{1-\rho} = \rho, 
\end{equation*}
again in accordance with our earlier result.

\begin{question}
  It is an easy mistake to think that $\E{S_r} = \E S$ when service
  times are exponential. Why is this wrong?
\hint{Realize again that $\E{S_r}$ includes the jobs that arrive at an empty system.}
  \begin{solution}
    $\E{S_r \given S_r>0} = \E S$ for the $M/M/1$ queue, and
    $\E{S_r} = \rho \E{S_r \given S_r>0}$ for the $M/G/1$ queue, it
    follows that
  \begin{equation*}
 \E{S_r} = \rho \E{S_r\given S_r>0} = \rho \E S.
  \end{equation*}
  \end{solution}
\end{question}

\begin{question}
  Try to derive relation~(\ref{eq:37}) with sample path
  arguments. 
  \hint{ This requires some extra definitions, similar to
  $A(n,t)$ as defined in~\eqref{eq:19}.}
  \begin{solution}
    This is, admittedly, not simple, so let us work in stages. Define
    $A(+,t)$ as the number of arrivals that see a job in service. As
    the server is only busy at time $t$ when $L(t)>0$, we define
    \begin{equation*}
      A(+,t) = \sum_{k=1}^\infty \1{A_k\leq t}\1{L(A_k) > 0}.
    \end{equation*}
    Next, define $\tilde D_k = \min\{D_i: D_i \geq A_k\}$ as the first
    departure after the arrival time $A_k$ of job $k$. We need to
    consider two cases. If $\tilde D_k = D_k$ then, obviously, the
    first departure after $A_k$ coincides with the departure time of
    job $k$. This is only possible if $L(A_k) = 0$. But then, it must
    be that the remaining service time of the job in service at time
    $A_k$ is 0, as there is no job in service. Otherwise, if
    $L(A_k)>0$ it must be that $\tilde D_k < D_k$, and, as a
    consequence, the remaining service time of the job in service at
    time $A_k$ is equal to $\tilde D_k - A_k$. All in all, the total
    amount of remaining service times added up for all arrivals up to
    time $t$ can be written as
    \begin{equation*}
      S_r(t) = \sum_{k=1}^\infty (\tilde D_k - A_k) \1{A_k \leq t, L(A_k) > 0}.
    \end{equation*}
    The remaining service time averaged over all arrivals up to
    time $t$ is therefore $S_r(t)/A(t)$. We can now rewrite this to
\begin{equation*}
  \frac{S_r(t)}{A(t)} = 
  \frac{A(+,t)}{A(t)} \frac{S_r(t)}{A(+,t)}, 
\end{equation*}
and interpret these fractions.  First, consider
\begin{equation*}
  \frac{A(+,t)}{A(t)} = 
\frac{\sum_{k=1}^\infty \1{A_k\leq t, L(A_k) > 0}}{\sum_{k=1}^\infty \1{A_k\leq t}}.
\end{equation*}
This is the number of jobs up to time $t$ that see at least one job in
the system divided by the total number of arrivals up to time $t$. Thus, as $t\to\infty$, 
\begin{equation*}
  \frac{A(+,t)}{A(t)} \to \sum_{n=1}^\infty \pi(n) = 1-\pi(0).
\end{equation*}
With PASTA we can conclude that $1-\pi(0) = 1-p(0)= \rho$. Second,
\begin{equation*}
\frac{S_r(t)}{A(+,t)} 
= \frac{\sum_{k=1}^\infty (\tilde D_k - A_k) \1{A_k\leq t, L(A_k) > 0}}{\sum_{k=1}^\infty \1{A_k\leq t, L(A_k)>0}},
\end{equation*}
is the total amount of remaining service time up to time $t$ divided by
the number of arrivals up to time $t$ that see at least one job in the system. Thus, 
if the limit exists, we can \emph{define}
\begin{equation*}
\E{S_r \given S_r>0}  =\lim_{t\to\infty} \frac{S_r(t)}{A(+,t)}.
\end{equation*}
  \end{solution}
\end{question}

\begin{question}[use=false]
  As a challenge you can try to derive \eqref{eq:24} by means of sample path arguments.
  \begin{solution}
    \TBD
  \end{solution}
\end{question}



\begin{question}
  \begin{enumerate}
  \item Why is this true for the $M/M/1$ queue
  \begin{equation*}
\E{L_Q} = \sum_{n=1}^\infty (n-1)\pi(n)?
  \end{equation*}
\item   Derive an expression for $\E{L_Q}$ based on this observation.
  \end{enumerate}
  \begin{solution}
    \begin{enumerate}
    \item 
    The fraction of time the system contains $n$ jobs is $\pi(n)$ (by
    PASTA). When the system contains $n>0$ jobs, the number in queue
    is one less, i.e., $n-1$.
  \item 
    \begin{equation*}
      \begin{split}
\E{L_Q} 
&= \sum_{n=1}^\infty (n-1)\pi(n) 
= (1-\rho)\sum_{n=1}^\infty (n-1) \rho^n\\
&= \rho (1-\rho)\sum_{n=1}^\infty (n-1) \rho^{n-1}
= \rho \sum_{n=1}^\infty (n-1) \pi(n-1)\\
&= \rho \sum_{n=0}^\infty n \pi(n)
= \rho \frac{\rho}{1-\rho}.
      \end{split}
    \end{equation*}
Another way to get the same result is by splitting: 
\begin{equation*}
  \begin{split}
\E{L_Q} 
&= \sum_{n=1}^\infty (n-1)\pi(n) 
=\sum_{n=1}^\infty n\pi(n) -\sum_{n=1}^\infty \pi(n)\\
&= \E L - (1-\pi(0)) = \E L - \rho.
  \end{split}
\end{equation*}
    \end{enumerate}
  \end{solution}
\end{question}

\begin{question}
  Why is Eq.~(\ref{eq:61}) not true in general for the $M/G/1$ queue?
\hint{Think about the consequences of memoryless service times.}
  \begin{solution}
    Because the remaining service time of the job in service, provided
    there is a job in the system upon arrival, is not exponentially
    distributed in general. Only for the $M/M/1$ queue the service
    times are exponentially distributed, hence memoryless. And only
    when the service time is memoryless, the service time after an
    interruption is still exponential.
  \end{solution}
\end{question}

\begin{question}
  \begin{enumerate}
  \item 
  Use the PASTA property to see that the expected waiting time in the
  system must also be equal to
\begin{equation*}
  \E W = \sum_{n=0}^\infty \E{W_Q \given N=n} \pi(n) + \E S,
\end{equation*}
where $\E{W_Q \given N=n}$ is the waiting time in queue given that an
arrival sees $N=n$ customers upon arrival. 
\item Motivate that $\E{W_Q \given N=1} = \E S$. 
\item  Combine the above to see that 
\begin{equation*}
  \E W = \E S \sum_{n=0}^\infty n \pi(n) + \E S = \E S \E L + \E S.
\end{equation*}
  \end{enumerate}
  \begin{solution}
    \begin{enumerate}
    \item The time in the system $\E W$ is the sum of the time in
      queue plus the service time of the job itself.  Conditioning on
      the number of jobs $N$ in the system at arrival moments, i.e.,
      conditioning on $\pi(n)$, gives the result.  
    \item $\E{W_Q\given N=0}=0$, of course. If $N=1$, there must be a job
      in service.  Because of the memoryless property of the service
      times, the remaining service time of the customer in service is
      still exponentially distributed with mean $\E S$. Thus,
      $\E{W_Q\given N=1}=\E S$. 
    \item The expected service time for a customer still in queue is
      $\E S$.  Therefore, for each $n$ we have that, when service
      times are exponential, $\E{W_Q\given N=n}= n \E S$. The probability
      to see $n$ jobs in the system by an arrival is $\pi(n)$ which,
      by PASTA, is equal to $p(n)$. Since $\E L = \sum_n n p(n)$, the
      result follows.
    \end{enumerate}
  \end{solution}
\end{question}

\begin{question}
  There is distinction between $\E{L_Q}$, i.e., the time-average queue
  length, and the expected number of jobs in the system given that the
  server is busy, i.e., $\E{L\given B}$.  is $\E{L_Q} \neq \E{L\given B}$?
\begin{solution}
  $\E{L_Q}$ is the time average of the queue length process (in steady
  state). Hence, $\E{L_Q}$ contains also the time the queue is
  empty. $\E{L\given B}$ the other hand, is the time average of the
  queue length process, provided the server is busy. Even though the
  queue may be empty while the server is busy, the fraction of time
  the queue length is zero given that the server is busy, is smaller
  than the fraction of time the queue length is zero.  Hence,
  $\E{L\given B} > \E{L_Q}$.
\end{solution}
\end{question}

\begin{question}
  Compute the variance of the number of jobs in the system $L$ for the $M/G/1$ queue.
  \begin{solution}
    $\V{L} = \E{L^2} - (\E L)^2$. We already know $\E L$ so it remains
    to compute $\E{L^2}$. With wolfram alpha we get
    \begin{equation*}
      \begin{split}
      \E{L^2 }
&= \sum_{n=0}^\infty n^2 \pi(n) \\
&= \rho \frac{1+\rho}{(1-\rho)^2}.
      \end{split}
    \end{equation*}
    Thus,
\begin{equation*}
\V L = \frac{\rho(1+\rho)}{(1-\rho)^2}-\frac{\rho^2}{(1-\rho)^2} = \frac{\rho}{(1-\rho)^2}.
\end{equation*}

To see how large this variance is, relative to the mean number of jobs
in the system, we typically consider the square coefficient of
variation (SCV). As $\E L = \rho/(1-\rho)$,
\begin{equation*}
  \frac{\V L}{(\E{L})^2} = \frac 1 \rho.
\end{equation*}
Thus, the SCV becomes smaller as $\rho$ increases, but does not become
lower than $1$. So, realizing that the SCV of the exponential
distribution is 1, the distribution of the number of jobs in the
system has larger relative variability than the exponential
distribution.


Lets see whether I can get the result for $\E{L^2}$ by myself.  One way
is to use the standard formula for a geometric series with $\rho < 1$:
\begin{equation*}
\dfrac{1}{1-\rho} = \sum_{n=0}^{\infty}\rho^n.
\end{equation*}
If we differentiate the left and right hand side with respect to
$\rho$  and then  multiply with $\rho$ we obtain
\begin{equation*}
\dfrac{\rho}{(1-\rho)^2}=\sum_{n=0}^{\infty}n\rho^n.
\end{equation*}
Again, differentiating and multiplying with $\rho$ yields, 
\begin{equation*}
  \begin{split}
\rho \frac{(1-\rho)^2 + \rho2(1-\rho)}{(1-\rho)^4} 
&= \rho \frac{1-2\rho+\rho^2 + 2\rho-2\rho^2}{(1-\rho)^4} \\
&= \rho \frac{(1-\rho)^2}{(1-\rho)^4} \\
&=\rho \dfrac{1+\rho}{(1-\rho)^3}\\
&=\sum_{n=0}^{\infty}n^2\rho^n
  \end{split}
\end{equation*}
and hence
\begin{equation*}
(1-\rho)\sum_{n=0}^{\infty}n^2\rho^n = \rho\dfrac{1+\rho}{(1-\rho)^2}
\end{equation*}
% Write
% $\partial_\rho$ as a shorthand for the derivative with respect to
% $\rho$. Then observe that
% $(\rho \partial_\rho) \rho^n = \rho n \rho^{n-1} = n\rho^n$, hence
% $(\rho \partial_\rho)^2 \rho^n =
% (\rho \partial_\rho)(\rho\partial_\rho) \rho^n = (\rho \partial
% \rho) n \rho^n = n^2\rho^n$. With this,
% \begin{equation*}
%   \begin{split}
% (1-\rho)  \sum_{n=0}^\infty n^2 \rho^n 
% &=  (1-\rho)(\rho \partial_\rho)^2   \sum_{n=0}^\infty \rho^n\\
% &=  (1-\rho)(\rho \partial_\rho)^2   \frac1{1-\rho}\\
% &=  (1-\rho)\rho \partial_\rho   \frac\rho{(1-\rho)^2}\\
% &=  (1-\rho) \rho  \frac{1+\rho}{(1-\rho)^3} \\
% &=  \rho  \frac{1+\rho}{(1-\rho)^2}.
%   \end{split}
% \end{equation*}
Observe that here we just compute the second moment of a geometric
random variable.


Another way to derive the result is by noting that
$\sum_{i=1}^n i= n(n+1)/2$ from which we get that
$n^2 = -n + 2\sum_{i=1}^n i$. Substituting this relation into
$\sum_n n^2 \rho^n$ leads to
\begin{equation*}
  \begin{split}
    \sum_{n=0}^\infty n^2 \rho^n 
&=    \sum_{n=0}^\infty \left(\sum_{i=1}^\infty 2i \1{i\leq n}  - n\right)\rho^n 
=    \sum_{n=0}^\infty \sum_{i=0}^\infty 2i\1{i\leq n}\rho^n  - \sum_{n=0}^\infty n\rho^n \\
&=    \sum_{i=0}^\infty 2i \sum_{n=i}^\infty \rho^n  - \frac{\E L}{1-\rho} 
=    \sum_{i=0}^\infty 2i \rho^i \sum_{n=0}^\infty \rho^n  - \frac{\E L}{1-\rho} \\
&=    \frac2{1-\rho} \sum_{i=0}^\infty i \rho^i   - \frac{\E L}{1-\rho} 
=    \frac2{(1-\rho)^2} \E L - \frac{\E L}{1-\rho} \\
&=    \frac{\E L}{1-\rho}  \left(\frac2{1-\rho}  - 1\right) 
=    \frac{\E L}{1-\rho}  \frac{1+\rho}{1-\rho} \\
&=    \frac{\rho}{1-\rho}  \frac{1+\rho}{(1-\rho)^2}.
\end{split}
\end{equation*}

A last method is based on $z$-transforms:
\begin{equation*}
  \phi(z) = \E{z^L} = \sum_{n=0}^\infty z^n p(n) = (1-\rho) \sum_{n=0}^\infty (\rho z)^n = \frac{1-\rho}{1-\rho z}.
\end{equation*}
Then 
\begin{equation*}
  \E L = \left.\frac d {dz} \phi(z)\right|_{z=1} = \frac{\rho}{1-\rho},
\end{equation*}
and 
\begin{equation*}
  \E{L(L-1)}= \left.\phi''(z)\right|_{z=1} = \frac{2\rho^2}{(1-\rho)^2}.
\end{equation*}
Thus,
\begin{equation*}
\E L^2 =   \E{L(L-1)} + \E L.
\end{equation*}
Hence,
\begin{equation*}
  \begin{split}
\E L^2 
&=   \frac{2\rho^2}{(1-\rho)^2} + \frac\rho{1-\rho} \\
&=   \frac{2\rho^2}{(1-\rho)^2} + \frac\rho(1-\rho){(1-\rho)2} \\
&=   \frac{\rho^2+\rho}{(1-\rho)^2} = \rho \frac{1+\rho}{(1-\rho)^2}.
  \end{split}
\end{equation*}

A bit of algebra gives the previous results.
  \end{solution}
\end{question}


\begin{question}
  Use the PASTA property and the ideas of Section~\ref{sec:level-cross-balance}
 to derive  for the $M/M/1$ queue that
  \begin{equation*}
  (\lambda + \mu) \pi(n) = \lambda \pi(n-1) + \mu \pi(n+1)
  \end{equation*}
  \begin{solution}
    Consider some state $n$ and count all transitions that `go in and
    out of' this state. Specifically, $A(n,t) + D(n-1,t)$ counts all
    transitions out of state $n$: $A(n,t)$ counts the number of
    arrivals that see $n$ in the system upon arrival, hence
    immediately after such arrivals the system contains $n+1$ jobs;
    likewise, $D(n-1,t)$ counts all jobs that leave $n-1$ jobs behind,
    hence immediately before such jobs depart, the system contains $n$
    jobs.  In a similar way, $A(n-1,t) + D(n+1,t)$ counts all
    transitions into state $n$.  Therefore the diffence between the
    `in' transitions and the `out' transitions is at most 1 for all
    $t$, so that we can write 
    \begin{equation*}
      \begin{split}
      A(n,t) + D(n-1,t) &\approx A(n-1,t) + D(n+1,t), \text{ hence} \\
      \frac{A(n,t) + D(n-1,t)}t &\approx \frac{A(n-1,t) + D(n+1,t)}t, \text{ hence} \\
      \frac{A(n,t)}t + \frac{D(n-1,t)}t &\approx \frac{A(n-1,t)}t + \frac{D(n+1,t)}t.
      \end{split}
    \end{equation*}
Using the ideas of Section~\ref{sec:level-cross-balance} this becomes for $t\to\infty$, 
\begin{equation*}
  (\lambda(n) +\mu(n))p(n) = \lambda(n-1)p(n-1) + \mu(n+1)p(n+1).
\end{equation*}
Since we are concerned here with the $M/M/1$ queue we have that
$ \lambda(n) = \lambda$ and $\mu(n) = \mu$, and using PASTA we have
that $p(n) = \pi(n)$. We are done.
  \end{solution}
\end{question}


\begin{question}
  What would you guess for $\E{S_r\given S_r>0}$ for the $M/D/1$ queue? 
  \begin{solution}
    Since the service times are deterministic (and constant), I would
    guess that on average half of the service time remains at the
    moment a job arrives. If the service time is $D$ always, then
    $\E{S_r\given S_r>0} = \lambda D/2$.  
  \end{solution}
\end{question}


\begin{question}(Hall 5.2) After observing a single server queue for
  several days, the following steady-state probabilities have been
  determined: $p(0)=0.4$, $p(1) = 0.3$, $p(2)=0.2$, $p(3)=0.05$ and
  $p(4)=0.05$. The arrival rate is 10 customers per hour. 
  \begin{enumerate}
  \item Determine $\E L$ and  $\E{L_Q}$. 
  \item Using Little's formula, determine $\E W$ and $\E{W_Q}$. 
\item Determine $\V{L}$ and $\V{L_Q}$.
  \item Determine the service time and the utilization.
  \end{enumerate}
  

    \begin{solution}
      \begin{enumerate}
      \item 
      When a problem is mainly of a computational type, I coded the
      solutions and show you all the numerical answers. Thus, please
      read the code too since this shows the precise steps by which I
      obtain the answer. (The code is typically nearly identical to
      the mathematical formulas; you should not have any difficulty
      understanding the code.) I also show many intermediate numerical
      results so that you can check each step in your computations. (In
      the computations below I typically use the simplest, but often
      not the most efficient, code.)

<<term=True>>=
P = [0.4, 0.3, 0.2, 0.05, 0.05]
L = sum(n*P[n] for n in range(len(P)))
L
@ 

There can only be a queue when a job is in service. Since there is
$m=1$ server, we subtract $m$ from the amount of jobs in the system.
Before we do this, we need to ensure that $n-m$ does not become
negative. Thus, $\E{L_Q} = \sum_n \max\{n-m, 0\} p(n)$.

<<term=True>>=
m = 1
Lq = sum(max(n-m,0)*P[n] for n in range(len(P)))
Lq
@

\item 

<<term=True>>=
labda = 10./60
Wq = Lq/labda # in minutes
Wq
Wq/60 # in hours

W = L/labda # in minutes
W
W/60 # in hours
@ 

\item 

<<term=True>>=
from math import sqrt
var_L = sum((n-L)**2*P[n] for n in range(len(P)))
var_L
sqrt(var_L)
@


<<term=True>>=
var_Lq = sum((max(n-m,0)-Lq)**2*P[n] for n in range(len(P)))
var_Lq
sqrt(var_Lq)
@ 

\item 

<<term=True>>=
mu = 1./(W-Wq)
1./mu # in minutes

rho = labda/mu
rho
@ 

<<term=True>>=
rho = L-Lq
rho
@ 
This checks the previous line.

The utilization must also by equal to the fraction of time the server is busy. 
<<term=True>>=
u = 1 - P[0]
u
@ 

Yet another way: Suppose we have $m$ servers. If the system is empty,
all $m$ servers are idle. If the system contains one customer, $m-1$
servers are idle. Therefore, in general, the average fraction of time
the server is idle is
\begin{equation*}
1- u = \sum_{n=0}^\infty \max\{n-m, 0\}  p_n,
\end{equation*}
as in case there are more than $m$ customers in the system, the
number of idle servers is $0$.


<<term=True>>=
idle = sum( max(m-n,0)*P[n] for n in range(len(P)))
idle
@ 
  \end{enumerate}
   \end{solution}
 
\end{question}

\begin{question}
  (Hall 5.5) An $M/M/1$ queue has an arrival rate of 100 per hour and
  a service rate of 140 per hour.
  \begin{enumerate}
  \item What is $p(n)$?
\item What are $\E{L_Q}$, $\E L$ 
\end{enumerate}

\begin{solution}
  \begin{enumerate}
  \item 
$p(n) = (1-\rho)\rho^n$

\item 

<<term=True>>=
labda = 100. # per hour
mu = 140. # per hour
ES = 1./mu
rho = labda/mu
rho 
1-rho

L = rho/(1.-rho)
L
Lq = rho**2/(1.-rho)
Lq

W = 1./(1.-rho) * ES
W
Wq = rho/(1.-rho) * ES
Wq
@ 
  \end{enumerate}
  \end{solution}
\end{question}

\begin{question}(Hall, 5.6)
  An $M/M/1$ queue has been found to have an average waiting time in queue of 1 minute. The arrival rate is known to be 5 customers per minute.
  \begin{enumerate}
  \item What are the service rate and utilization?
  \item Calculate $\E{L_Q}$,  $\E L$ and $\E W$.
  \item The queue operator would like to provide chairs for waiting customers. He would like to have a sufficient number so that all customers can sit down at least 90 percent of the time. How many chairs should he provide?
  \end{enumerate}
  
    \begin{solution}
      \begin{enumerate}
      \item $\E{W_Q} = \frac{1}{\lambda}\frac{\rho^2}{1-\rho}$. Since
        $\E{W_Q}$ and $\lambda$ is given we can use this formula to
        solve for $\rho$ with the abc-formula (and using that
        $\rho > 0$):

<<term=True>>=
labda = 5. # per minute
Wq = 1.
a = 1.
b = labda*Wq
c = -labda*Wq
rho = (-b + sqrt(b*b-4*a*c))/(2*a)
rho 

ES = rho/labda
ES
@ 

\item 

<<term=True>>=
Lq = labda*Wq
Lq

W = Wq + ES
W

L = labda*W
L
@ 

\item 

The problem is to find $n$ such that
      $\sum_{j=0}^n p_j > 0.9$.

<<term = False>>=
tot = 0.
j = 0
while tot <= 0.9:
   tot += (1-rho)*rho**j
   j += 1
n = j- 1
@ 

Observe that $j$ is one too high once the condition is satisfied, thus subtract one.

<<term=True>>=
tot
n  # the number of chairs 
@      

As a check, I use that $(1-\rho) \sum_{j=0}^n \rho^j = 1-\rho^{n+1}$.

<<term=True>>=
1-rho**(n) #  this must be too small.
1-rho**(n+1) # this must be ok.
@ 

And indeed, we found the right $n$.

  \end{enumerate}

    \end{solution}
\end{question}

\begin{question}
  (Hall 5.7). A single server queueing system is known to have Poisson
  arrivals and exponential service times. However, the arrival rate
  and service time are state dependent. As the queue becomes longer,
  servers work faster, and the arrival rate declines, yielding the
  following functions (all in units of number per hour):
  $\lambda(0) = 5$, $\lambda(1)=3$, $\lambda(2)=2$,
  $\lambda(n)=0, n\geq 3$, $\mu(0) = 0$, $\mu(1)=2$, $\mu(2)=3$, $\mu(n)=4, n\geq 3$. 
Calculate the state probabilities, i.e., $p(n)$ for $n=0,\ldots$. 
\hint{ Use the level-crossing equations of the $M(n)/M(n)/1$ queue.}
    \begin{solution}
      Follows right away from the hint.
    \end{solution}
\end{question}

\begin{question}
  (Hall 5.14) An airline phone reservation line has one server and a
  buffer for two customers. The arrival rate 6 customers per hour, and
  a service rate of just 5 customers per hour. Arrivals are Poisson and service times are exponential. 
  \begin{enumerate}
  \item Estimate $\E{L_Q}$ and the average number of customers served per hour.
  \item Estimate $\E{L_Q}$ for a buffer of size 5. What is the impact of the increased buffer size on the number of customers served per hour?
  \end{enumerate}
  
\hint{ This is a queueing system with loss, in particular the $M/M/1/1+2$ queue.}
    \begin{solution}
      \begin{enumerate}

      \item 

<<term=True>>=
labda = 6.
mu = 5.
rho = labda/mu
c = 1
b = 2
@ 

Set $p(n) = \rho^n$ initially, and normalize later. Use the
expressions for the $M(n)/M(n)/1$ queue.  Observe that $\rho>1$. Since
the size of the system is $c+b+1$ is finite, all formulas work for
this case too.


There are 4 states in total: $0,1,2,3$.

<<term=True>>=
P = [rho**n for n in range(c+b+1)]
P

tot = sum(P)
tot

P = [p/tot for p in P] # normalize
P
@ 

<<term=True>>=
L = sum(n*P[n] for n in range(len(P)))
L

Lq = sum((n-c)*P[n] for n in range(c,len(P)))
Lq
@ 


The number of jobs served per hour must be equal to the number of jobs
accepted, i.e., not lost. The fraction of customers lost is equal to
the fraction of customers that sees a full system.

<<term=True>>=
lost = labda*P[-1] # the last element of P
lost

accepted = labda*(1.-P[-1]) # rate at which jobs are accepted
accepted
@  

\item 

 Increase $b$ to 5
<<term=True>>=
b = 5
P = [rho**n for n in range(c+b+1)]
P
tot = sum(P)
tot

P = [p/tot for p in P] # normalize
P

L = sum(n*P[n] for n in range(len(P)))
L

accepted = labda*(1.-P[-1])
accepted
@      
  \end{enumerate}
    \end{solution}
\end{question}

\begin{question}[use=false]
\nvf{The code is in \texttt{progs/mm1\_waiting\_time.py}. In
  mm1\_waiting\_time_distribution.py we compute the waiting time
  distribution. At this point in the text this distribution has not
  yet been derived.  This problem should be moved to another section.}

Consider an $M/M/1$ queue with $\mu=1.2$ per hour. Make plots of the
waiting time distribution $\P{W_Q\leq t}$ for various values of
$\lambda$ to show the dependency on the arrival rate. Compare this to
the dependence of average waiting time $\E{W_Q}$ on $\lambda$.

The idea of making such plots is to analyze
  whether the service capacity suffices for a situation in which one
  doesn't know the arrival rate very accurately, but one can control
  the service rate, e.g., by hiring personel.
  \begin{solution}
    Let's just plot the waiting time distribution
for various values of $\lambda$. 




\begin{figure}[ht]
  \centering
\includegraphics{progs/mm1_waiting_time.tex}
  \caption{The waiting time distribution for various values of the arrival rate.}
  \label{fig:waitingtime}
\end{figure}


\end{solution}

\end{question}




\begin{question}(Hall 5.3) After observing a queue with two servers
  for several days, the following steady-state probabilities have been
  determined: $p(0)=0.4$, $p(1) = 0.3$, $p(2)=0.2$, $p(3)=0.05$ and
  $p(4)=0.05$. The arrival rate is 10 customers per hour.
  \begin{enumerate}
  \item Determine $\E L$ and  $\E{L_Q}$. 
  \item Using Little's formula, determine $\E W$ and $\E{W_Q}$. 
  \item Determine $\V L$ and $\V{L_Q}$.
  \item Determine the service time and the utilization.
  \end{enumerate}

  \begin{solution}
    \begin{enumerate}
    \item 
<<term=True>>=
P = [0.4, 0.3, 0.2, 0.05, 0.05]

c = 2
Lq = sum((n-c)*P[n] for n in range(c,len(P)))
Lq

L= sum(n*P[n] for n in range(len(P)))
L
@

\item 
<<term=True>>=
labda = 10./60
Wq = Lq/labda # in minutes
Wq
Wq/60 # in hours

W = L/labda
W
@ 

\item 
<<term=True>>=
from math import sqrt
var_L = sum((n-L)**2*P[n] for n in range(len(P)))
var_L
sqrt(var_L)
@ 

\item 
<<term=True>>=
var_q = sum((max(n-c,0)-Lq)**2*P[n] for n in range(len(P)))
var_q
sqrt(var_q)
@

\item 
<<term=True>>=
idle = sum( (c-n)*P[n] for n in range(c))
idle
@ 
  \end{enumerate}
    \end{solution}
\end{question}  

\begin{question}
  (Hall 5.8) The queueing system at a fast-food stand behaves in a
  peculiar fashion. When there is no one in the queue, people are
  reluctant to use the stand, fearing that the food is
  unsavory. People are also reluctant to use the stand when the queue
  is long. This yields the following arrival rates (in numbers per hour): $\lambda(0) = 10$, $\lambda(1)=15$, $\lambda(2)=15$, $\lambda(3)=10$, $\lambda(4)=5$, $\lambda(n)=0, n\geq 5$. The stand has two servers, each of which can operate at 5 per hour. Service times are exponential, and the arrival process is Poisson.
  \begin{enumerate}
  \item Calculate the steady state probabilities.
  \item What is the average arrival rate?
  \item Determine $\E L$, $\E{L_Q}$, $\E W$ and $\E{W_Q}$.
  \end{enumerate}
  \begin{solution}
      \begin{enumerate}
      \item 

<<term=True>>=
import numpy as np
labda = [10., 15., 15., 10., 5.]
c = 2
mn = np.array([0,1,2,2,2]) # number of active servers
mu = 5*mn
mu
@

Use the level crossing result for the $M(n)/M(n)/1$ queue:

<<term=True>>=
P = [1]*5
for i in range(1,5):
    P[i] = labda[i-1]/mu[i]*P[i-1]

P
tot = sum(P)
tot
P = [p/tot for p in P]
P # normed
@ 

\item 

$\lambda = \sum_{n}\lambda(n) p(n)$.

<<term=True>>=
labdaBar = sum(labda[n]*P[n] for n in range(len(P)))
labdaBar
@



\item 
<<term=True>>=
Ls = sum(n*P[n] for n in range(len(P)))
Ls

c = 2
Lq = sum((n-c)*P[n] for n in range(c,len(P)))
Lq
@ 

And now the waiting times:

<<term=True>>=
Ws = Ls/labdaBar
Ws

Wq = Lq/labdaBar
Wq
@ 

  \end{enumerate}
    \end{solution}
\end{question}

\begin{question}
  (Hall 5.10) A repair/maintenance facility would like to determine
  how many employees should be working in its tool crib. The service
  time is exponential, with mean 4 minutes, and customers arrive by a
  Poisson process with rate 28 per hour. The customers are actually
  maintenance workers at the facility, and are compensated at the same
  rate as the tool crib employees.
  \begin{enumerate}
  \item What is $\E W$ for $c=1, 2, 3$, or $4$ servers?
  \item How many employees should work in the tool crib?
  \end{enumerate}

  \begin{solution}
    \begin{enumerate}
    \item 
      Would one server/person do? 
<<term=True>>=
labda = 28./60 # arrivals per minute
ES = 4.
rho = labda*ES
rho
@ 

A load $\rho>1$ is clearly undesirable for one server.  We need at
least two servers.

The remark that maintenance workers are compensated at the same rate
as the tool crib workers confused me a bit at first.  Some thought
revealed that the consequence of this remark is that it just as
expensive to let the tool crib workers wait (to help maintenance
workers) as to let the maintenance workers wait for tools. (Recall, in
queueing systems always somebody has to wait, either the customer or
the server. If it is very expensive to let customers wait, the number
of servers must be high, whereas if servers are relatively very
expensive, customers have to do the waiting.)

<<term = False>>=
from math import factorial

def WQ(c, labda, ES):
    rho = labda*ES
    tot = sum([rho**n/factorial(n) for n in range(c)])
    tot += rho**c/factorial(c)/(1.-rho/c)
    P0 = 1./tot
    Lq = rho**(c+1)/c/(factorial(c)*(1.-rho/c)**2)*P0
    return Lq/labda # Wq
@ 

Considering the scenario with one server is superfluous as $\rho>1$ in
that case.

<<term=True>>=
WQ(2, 28./60, 4)
WQ(2, 28./60, 4)/60. # in hours
WQ(3, 28./60, 4)
WQ(3, 28./60, 4)/60. # in hours
WQ(4, 28./60, 4)
WQ(4, 28./60, 4)/60. # in hours
@ 

\item It is not relevant to focus on the time in the system , as time
  in service needs to be spent anyway. Hence, we focus on the waiting
  time in queue.

Since both types of workers cost the same amount of money per unit
time, it is best to divide the amount of waiting/idleness equally over
both types of workers.  I am inclined to reason as follows. The
average amount of waiting time done by the maintenance workers per
hour is $\lambda \E{W_Q}$. To see this, note that $\lambda$
customers arrive, on average, each hour, and each customer waits on
average $\E{W_Q}$ minutes. On the other hand, the total amount of
minutes wasted by the tool crib employees is $c-\rho$, as
$\rho = \lambda \E S$ is the average number of servers busy,
while $c$ crib servers are available.

Now, while this makes perfect sense, to me at least, the formulas tell
their own story. $\lambda\E{W_Q} = \E{L_Q}$. In other words,
$\E{L_Q}$ is the average number of maintenance employees wasting
their time in queue, while $c -\rho$ is the average number of
crib workers wasting their time being idle. (To paraphrase Prince,
doing something close to nothing.) Therefore, as both types of
employees are equally expensive, we need to choose $c$ such that
$\E{L_Q} \approx c- \rho$, where, of course, $\E{L_Q}$ depends on
$c$.

<<term=True>>=
labda = 28./60
ES = 4.
rho = labda*ES
c = 2
labda*WQ(c, labda, ES)
c-rho
@ 
Now the maintenance employees wait more than the tool crib employees.

<<term=True>>=
c = 3
labda*WQ(c, labda, ES)
c-rho
@ 

<<term=True>>=
c = 4
labda*WQ(c, labda, ES)
c-rho
@ 

Clearly, $c=3$ should do.
  \end{enumerate}
    \end{solution}
\end{question}

\begin{question}
  An $M/M/2$ queueing system is very heavily utilized, with an arrival rate of 11 customers per hour, and a  service rate of 6 customers per hour per server.
  \begin{enumerate}
  \item Determine $\E{L_Q}$ and $\E{W_Q}$.
  \item Compare your solution to an $M/M/1$ queue with $\rho=11/12$. Why are your answers similar or different?
  \end{enumerate}
  \begin{solution}
  \begin{enumerate}
  \item 

<<term=True>>=

labda = 11.
mu = 6
c = 2

rho = labda/mu
rho < c  # is the system stable?
@ 

Compute Ls and so on


<<term=True>>=
tot = sum(rho**n/factorial(n) for n in range(c))
tot += rho**c/factorial(c)/(1.-rho/c)
P0 = 1./tot
P0

Lq = rho**(c+1)/c/(factorial(c)*(1.-rho/c)**2)*P0
Lq
Lq/labda # Wq

Ls = Lq + rho
Ls
Ls/labda # Ws

ES = 1./mu # expected service time
ES

@ 

\item 

<<term=True>>=
mu = c*6

rho = labda/mu

P0 = 1-rho
P0

Lq = rho**2/(1.-rho)
Lq
Lq/labda # Wq

Ls = rho/(1.-rho)
Ls
Ls/labda # Ws

ES = 1./mu # expected service time
ES
@ 
It is of interest to compare the M/M/2 system to the M/M/1 system with twice the capacity. 
  \end{enumerate}
    \end{solution}
\end{question}




\begin{question}
  Generalize the single-server relation
  $\E{W_Q} = \rho \E{S_r\given S_r>0} + \E{L_Q} \E S$ to the $M/M/c$
  multi-server queue. 

We will use this result later to derive algorithms to compute the
performance measures for the analysis of closed-queueing network.

  \hint{ Interpret each component at the right hand side of the equation
    and generalize it to a multi-server queueing system. }
  \begin{solution}

    Recall that for the $M/M/1$ queue, $\E{S_r\given S_r>0} = 1/\mu$.
    Applying Little's law to the queue, we have that
    $\lambda \E{W_Q} =\ E{L_Q}$, so that it follows that
    \begin{equation*}
    \E{W_Q} = \rho \E{S_r \given S_r > 0} + \E{L_Q}\E S = \frac{\rho}\mu + \rho \E{W_Q},
    \end{equation*}
    where we use that $\rho=\lambda \E S$. In words, the first term
    (at the right-hand side) is the probability to find a job in
    service, which is $\rho$ times the expected remaining service
    time if the server is busy, which is $1/\mu$ for exponentially
    distributed processing times. The second term is the probability
    to find the server busy, so that the arriving job has to wait,
    times the expected waiting time. 

    For the $M/M/c$ queue we now derive the corresponding expressions
    for this first and second term.

    For the first, note that the probability that the job has to wait
    is the same as the probability that all servers are occupied at
    arrival. This is, clearly, $\sum_{k=c}^\infty p(k)$. If all
    servers are busy, the expected time until a service completion is
    $1/c\mu$, since there are $c$ servers and the expected time to
    completion is $1/\mu$ for each individual server. Thus, the time
    until the first departure is the minimum of the remaining services
    at each of the servers. This has expection $1/c\mu$. Thus, for the
    $M/M/c$ queue
    \begin{equation*}
\rho\E{S_r\given S_r>0}  = \sum_{k=c}^\infty p(k) \frac 1{c\mu}.
    \end{equation*}


    For the second term, since there are $c$ servers, jobs depart from
    the queue (hence enter service) at rate $c\mu$. The expected time
    in queue when there are $n$ jobs in front, is therefore
    $n/c\mu$. Thus, on average, since the expected queue length is $\E{L_Q}=\sum_{k=c}^\infty (k-c)p(k)$, we 
get 
\begin{equation*}
  \E{L_Q}\frac{\E S}c = \sum_{k=c}^\infty \frac{k-c}{c\mu}p(k).
\end{equation*}

All in all,
\begin{equation*}
  \begin{split}
  \E{W_Q} 
&= \rho\E{S_r\given S_r>0}  +   \E{L_Q}\frac{\E S}c \\
&= \sum_{k=c}^\infty p(k)\frac{1+k-c}{c\mu}.
  \end{split}
\end{equation*}

We can also give this a natural interpretation. How many jobs have to
leave before the service of the arriving job can start? Well, the
entire queue must be cleared plus one job in service. Thus, if there
are $k-c$ job in queue, we have to wait for these jobs to finish, plus
one of the jobs in service. 

Observe also that, by PASTA, we have $\pi(k) = p(k)$, so that the
arriving job also sees $p(k)$.

  \end{solution}
\end{question}


%%% Local Variables:
%%% mode: latex
%%% TeX-master: t
%%% End:
