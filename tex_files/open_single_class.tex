\section{Open Single-Class Product-Form Networks}
\label{sec:jackson-networks}


\subsection*{Theory and Exercises}

\Opensolutionfile{hint}
\Opensolutionfile{ans}

The remark above Zijm.Eq.2.11 is not entirely correct. Remove the
sentence: `These visit ratios satisfy \ldots up to a multiplicative
constant'.


I don't like the derivation of Zijm.Eq.2.20. The appearance of the
visit ratios $\lambda_i/\gamma$ seem to come out of thin air. The
argument should be like this. Consider the entire queueing network as
one `box' in which jobs enter at rate $\gamma=\sum_{i=1}^M
\gamma_i$.
Assuming that there is sufficient capacity at each station, i.e.,
$\lambda_i < c_i \mu_i$ at each station $i$, the output rate of the `box' must also be $\gamma$. Thus, by applying Little's law to the `box', we have that 
\begin{equation*}
  \E L = \gamma \E W. 
\end{equation*}
It is also evident that the average total number of jobs must be equal
to the sum of the average number of  jobs at each station: 
\begin{equation*}
  \E L = \sum_{i=1}^M \E{L_i}.
\end{equation*}
Applying Little's law to each station separately we get that
$\E{L_i} = \lambda_i\E{W_i}$. Filling this into the above,
\begin{equation*}
\E W = \frac{\E L}{\gamma}  = \sum_{i=1}^M \frac{\E{L_i}}\gamma = \sum_{i=1}^M \frac{\lambda_i \E{ W_i}}\gamma, 
\end{equation*}
where we recognize the visit ratios.




\begin{comment}
\begin{exercise}[use=false]
  Use the ideas of Zijm.Eq.2.2 to set up a set of balance equations
  for a single-server queueing station with two queues such that queue
  1 is served with strict priority over queue 2, and service of a job
  of type 2 is preemptive. Assume that type i jobs arrive at rate
  $\lambda_i$ and require average service time
  $\mu_i^{-1}$. Inter-arrival times of jobs each type are exponentially distributed,
  just as the service times.

  Can you find similar equations for the case with multiple servers at
  the station, all serving type 1 jobs with strict priority over type
  2 jobs?
  \begin{solution}
    TBD.
  \end{solution}
\end{exercise}
  
\end{comment}

\begin{exercise}(Linear algebra refresher) Can you find an example to
  show for two matrices $A$ and $B$ that $AB\neq BA$, hence
  $x A \neq A x$.
  \begin{hint}
    Let 
    \begin{equation*}
A =
    \begin{pmatrix}
      1 & 1 \\ 
0&1
    \end{pmatrix},
\quad   B=
    \begin{pmatrix}
      1 & 0 \\ 
1&1
    \end{pmatrix}
    \end{equation*}.
  \end{hint}

  \begin{solution}
    \begin{equation*}
      AB =  
    \begin{pmatrix}
      2 & 1 \\ 
1&1
    \end{pmatrix} 
\neq
    \begin{pmatrix}
      1 & 1 \\ 
1&2
    \end{pmatrix} 
= BA.
    \end{equation*}

Take $x=(1,1)$, then $x A=(1,2)$. Now, taking $x=
\begin{pmatrix}
  1 \\
1
\end{pmatrix}
$, we get $Ax = 
\begin{pmatrix}
  2 \\
1
\end{pmatrix}.  $
Recall, horizontal vectors are not vertical vectors. The horizontal
ones are to the left of a matrix, and the vertical ones to the right.
  \end{solution}
\end{exercise}

\begin{exercise}(Linear algebra refresher, 2) Suppose the matrix $A$
  has an eigenvalue $0$. What is the geometric meaning of this fact? 
  \begin{solution}
    Many students think that a matrix is just a bunch of numbers
    ordered in a grid. This is, in my opinion, the most unproductive
    way to think about matrices. A much more useful way is to see a
    matrix as an \emph{operator}. For instance, take $A$ to be a
    $3\times3$ matrix. Then it can be seen as \emph{mapping} from
    $\R^3$ to $\R^3$; it takes a vector $x\in\R^3$ and changes $x$
    into a new vector $Ax\in \R^3$. Thus, a square matrix $A$ typically changes the
 direction of a vector $x$. 

With this idea, consider the simple example with
    \begin{equation*}
A =
    \begin{pmatrix}
      1 & 0 & 0 \\ 
      0 & 1 & 0 \\ 
      0 & 0 & 0 \\ 
    \end{pmatrix}.
  \end{equation*}
Clearly, $A$ has an eigenvalue $0$. Now take $v=(x,y,z)'$, so that
    \begin{equation*}
A v = A
 \begin{pmatrix}
x\\
y\\
z
    \end{pmatrix}
= \begin{pmatrix}
x\\
y\\
0
    \end{pmatrix}.
  \end{equation*}
  We see that $A$ removes any information about the $z$ direction from
  the vector $v$. (It projects $v$ on the $x-y$ plane, and throws away
  the $z$ component of $v$.) But then, for a given vector $w=(x,y,0)$
  in the $x-y$ plane, it is impossible to use $A$ to retrieve the
  original vector $v=(x,y,z)$. Thus, $A$ cannot have an inverse on all
  of $\R^3$.

  So, hopefully, with this example, you can memorize that for any
  matrix $A$ to have an inverse, it is essential that it has no zero
  eigenvalues. When the \emph{operator} $A$ (don't think of a matrix
  as a set of numbers) throws away part of the dimension of the space
  on which it operates (i.e., it has one or more eigenvalue(s) $0$), it is impossible to retrieve the part of the space it throws away. Hence, its inverse cannot be used to get this part of the space back. 

\end{solution}


\end{exercise}


\begin{exercise}
Zijm.Ex.2.2.1
\begin{solution}
Because jobs cannot leave faster than they arrive.
\end{solution}
\end{exercise}

\begin{exercise}
Zijm.Ex.2.2.2
\begin{solution}
  Observe from Exercise~\ref{ex:dep} that the inter-departure times
  of the $M/M/1$ queue are also independent and identically
  exponentially distributed with rate $\lambda$. Since the arrival
  process at the second station is the departure process of the first
  station, it must be that arrival process at the second station is
  also Poisson with rate $\lambda$.  Interestingly, from the
  perspective of the second station it is as if there is not first
  station.
\end{solution}
\end{exercise}



\begin{exercise}
Zijm.Ex.2.2.3 
\begin{solution}
  The question is not well specified. We know from Burke's law, see Exercise~\ref{ex:burke}, that
  the arrival \emph{process} at the second station is Poisson.  If,
  however, we know that station 1 is empty, then it is unlikely that a
  job will arrive at station 2 in the very near future.

  Note that only the steady-state distributions of the queue lengths
  are independent. Once you have information about the state of one of
  the queues, then certainly this is not in `steady-state'.
\end{solution}
\end{exercise}

\begin{exercise}
Zijm.Ex.2.2.4
\begin{solution}
  Simple algebra.  (I am not going to write it out here. If you are
  willing to provide me the answer in \LaTeX\/ I'll include it.)
\end{solution}
\end{exercise}

\begin{exercise}
  Zijm.Ex.2.2.5. The problem is not entirely correctly formulated. It
  should be, if for at least one $i$, $\sum_{j=1}^M P_{i j} <1$ \ldots
\begin{solution}
Linear algebra is quite useful here!

Observe that $P_{i j}$ is the probability that a job, after completing
its service at node $i$, moves to node $j$. Then $\sum_{j=1}^M P_{i j}$
is the probability that a job moves from node $i$ to another node in
the network, i.e., stays in the network, while $P_{i0}$ is the
probability that a job departing from node $i$ leaves the network, in
other words, the job is finished. When $\sum_{j=1}^M P_{i j} < 1$, then
more jobs enter node $i$ from the network than that node $i$ sends
`back' into the network. Conceptually, node $i$ `leaks jobs'.

  Now, consider some node $k$ such that $P_{ki} > 0$, then the
  probability that a job that starts at node $k$, moves to node $i$
  and then leaves the network is equal to $P_{ki}P{i0}$. Thus, since
  $P_{ki}>0$ and $P_{i0}>0$, the probability that a job leaves the
  network from note $k$ in two steps is positive.  More specifically,
  $P^2_{k0} = \sum_{j=0}^M P_{kj}P_{j0} \geq P_{ki}P_{i0} > 0$. 

  The irreducibility assumption implies that in at most $M$ steps it
  is possible to reach, with positive probability, any node from any
  other node in the network. Thus, for any node $j$ to any other node
  $k$ there is a sequence of nodes $j_1, j_2, \ldots j_{M-1}$ such
  that  $P^{M}_{jk} \geq P_{j j_1}P_{j_1 j_2}\cdots P_{j_{M-1}k} > 0$.


  Thus, if there is a node $i$ such that $P_{i0}>0$, then it is
  possible from any node that sends jobs to node $i$ directly to leave
  the network in two steps. Likewise, when node $i$ can be reached
  from node $k$ in $n$ steps, say, then
  $P^{n+1}_{k0} \geq P^n_{ki}P_{i0} > 0$, i.e., in at most $n+1$ steps
  it is possible to leave the network from such node $k$. This
  implies, in particular, that for all nodes $k=1,2,\ldots M$, i.e.,
  all nodes in the network, $P^{M+1}_{k0} >0$.  For this reason we
  consider $P^{M+1}$ in the hint.


  As a final remark for students with knowledge of Markov chains,
  observe that the routing matrix $P$ does not correspond to the
  transition matrix of a recurrent Markov chain. Since for at least
  one row $i$, $\sum_{j=1}^N P_{i j}<1$, the matrix $P$ is
  sub-stochastic. Hence, a Markov chain induced by $P$ cannot be
  irreducible, because for this to happen, the chain must stay in some
  absorbing set with probability 1.
\end{solution}
\end{exercise}

\begin{exercise}
Zijm.Ex.2.2.6
\begin{solution}
  Since $M$ is finite, and $k\leq M$, the set of numbers
  $P^{M+1}_{k0}$ is finite. This, together with the fact that
  $P^{M+1}_{k0}>0$ for all $k$, implies that there is some number
  $\epsilon>0$ such that $P^{M+1}_{k0}>\epsilon$. Hence, for all
  entries $k=1, 2, \ldots M$, we have that
  $P^{M+1}_{kj} < 1- \epsilon$. This, in turn, implies that
  $P^{2(M+1)}_{kj} < (1- \epsilon)^2$, and so on, so that for any $n$,
  $P^{n(M+1)}_{kj} < (1- \epsilon)^n$. This implies, in more general
  terms, that the entries of $P^n$ decrease at some geometric rate to
  $0$.

  It is well known that for any bounded sequence $x_i$ and
  $0\leq \alpha < 1$, $ \sum_{i=0}^\infty x_i \alpha^i <
  \infty$. Applying this insight to the entries of $P^n$ it follows that 
$\sum_{n=0}^\infty P^n_{jk} < \infty$. 

Finally, applying  $\lambda = \gamma + \lambda P$ recursively, we get
\begin{equation*}
  \lambda = \gamma + \lambda P = \gamma + (\gamma + \lambda P)P = \gamma (1+P) + \lambda P^2 = \gamma(1+P+P^2) + \lambda P^3 \to \gamma \sum_{n=0}^\infty P^n.
\end{equation*}
By the above reasoning this last sum is well defined, and finite.  (By
the way, the above argument is not necessarily valid for matrices $P$
that are infinite, since then $\inf\{P^{M}_{ik}\}$ need not be
strictly positive.)

Another interesting way to see all this is by making the simplifying
assumption that $P$ is a diagonalizable matrix. (The argument can be
generalized to include matrices reduced to Jordan normal form, but
this gives optimal clutter, but does not change the line of reasoning in
any fundamental way.) In that case, there exists an invertible matrix
$V$ with the (left) eigenvectors of $P$ as its rows and a diagonal
matrix $\Lambda$ with the eigenvalues on its diagonal such that
\begin{equation*}
  V P = \Lambda V.
\end{equation*}
Hence, premultiplying with $V^{-1}$, 
\begin{equation*}
  P = V^{-1}\Lambda V.
\end{equation*}
But then
\begin{equation*}
P^2 = V^{-1}\Lambda V \cdot V^{-1}\Lambda V= V^{-1}\Lambda^2 V,
\end{equation*}
and in general $P^n = V^{-1}\Lambda^n V$. If each eigenvalue
$\lambda_i$ is such that its modulus $|\lambda_i| < 1$, then
$\Lambda^n \to 0$ geometrically fast, hence $P^n\to n$ geometrically
fast, hence the sequence of partial sums $\sum_{n=0}^N P^n$ must
converge to some finite number as $N\to\infty$. 

So, we are left with proving that the eigenvalues of $P$ must have
modulus less than $1$. This fact follows from Gerschgorin's disk
theorem, which I include for the interested student. Define the disk
$B(a,r)=\{z\in \mathbb{C} | |z-a|\leq r\}$, i.e., the set of complex numbers
such that the distance to the center $a\in \mathbb{C} $ is less than or equal
to the radius $r$. With this, the Gerschgorin disks of a matrix are
defined as $B(a_{ii}, \sum_{j\neq i} |a_{i j}|)$, i.e., disks with
center at the diagonal elements $a_{ii}$ of $A$ and radius equal
to the sum of the (modulus of the) elements of $A$ on the $i$th row
except $a_{ii}$. Then Gerschgorin's theorem says that all eigenvalues
of $A$ lie in the union of these disks, i.e., all eigenvalues
$\lambda_i \in \bigcup_i B_{a_{ii}, \sum_{j\neq i}|a_{i j}})$.

Assume for notational simplicity that for each row $i$ of $P$ we have
that $\sum_{j} a_{i j}<1$. (Otherwise apply the argument to $P^{M+1}$.)
Then this implies for all $i$ that
\begin{equation*}
  a_{ii} + \sum_{j\neq i} a_{i j} < 1. 
\end{equation*}
Since all elements of $P$ are non-negative, this also implies that
\begin{equation*}
-1 <   a_{ii} - \sum_{j\neq i} a_{i j} \leq  a_{ii} + \sum_{j\neq i} a_{i j} < 1. 
\end{equation*}
With this and using that $a_{ii}$ is a real number (so that it lies on
the real number axis) it follows that all elements in the disk
$B(a_{ii}, \sum_{j\neq i} a_{i j})$ have modulus smaller than 1.  As
this applies to any row $i$, all disks lie strictly within the complex
unit circle. But then, by Gerschgorin's theorem, all eigenvalues of
$P$ also lie strictly in the unit circle, hence all eigenvalues have
modulus smaller than 1.
\end{solution}
\end{exercise}

\begin{exercise}\label{ex:23}
  Show that Zijm.Eq.2.13 and 2.14 can be written as
  \begin{equation*}
    f_i(n_i) = \frac{1}{G(i)}\frac{1}{\Pi_{k=1}^{n_i} \min\{k, c_i\}}\left( \frac{\lambda_i}{\mu_i}\right)^{n_i}.
  \end{equation*}
  \begin{solution}
    Take $n_i<c_i$. Then
    $\Pi_{k=1}^{n_i} \min\{k, c_i\} = \Pi_{k=1}^{n_i} k = n_i!$, and
    $(c_i\rho_i)^{n_i} = (\lambda_i/\mu_i)^{n_i}.$ If $n_i\geq c_i$,
    then $\Pi_{k=1}^{n_i} \min\{k, c_i\} = c_i! c_i^{n_i-c_i}$, and
    $(c_i\rho_i)^{n_i} = (\lambda_i/\mu_i)^{n_i} c_i^{n_i}.$.
  \end{solution}
\end{exercise}


\begin{exercise}
Zijm.Ex.2.2.7
\begin{solution}
  \begin{equation*}
    P = 
    \begin{pmatrix}
      \alpha & 1- \alpha \\
      \beta_1 & \beta_2
    \end{pmatrix}.
  \end{equation*}

  \begin{equation*}
    (\lambda_1, \lambda_2) = (\gamma, 0) + (\lambda_1, \lambda_2) P.
  \end{equation*}
Solving first for $\lambda_2$ leads to $\lambda_2 = (1-\alpha) \lambda_1 + \beta_2 \lambda_2$, so that  
\begin{equation*}
  \lambda_2 = \frac{1-\alpha}{1-\beta_2} \lambda_1. 
\end{equation*}
Next, using this and that $\lambda_1 = \alpha \lambda_1 + \beta_1 \lambda_2 + \gamma$ gives with a bit of algebra
\begin{equation*}
  \begin{split}
\gamma 
&=  \lambda_1(1-\alpha) - \beta_1\lambda_2 \\
&=  \lambda_1\left(1-\alpha - \beta_1\frac{1-\alpha}{1-\beta_2}\right) \\
&=  \lambda_1(1-\alpha)\left(1 - \frac{\beta_1 }{1-\beta_2}\right) \\
&=  \lambda_1(1-\alpha)\frac{1-\beta_1-\beta_2 }{1-\beta_2}.
  \end{split}
\end{equation*}
Hence, 
\begin{equation*}
  \lambda_1 = \frac\gamma{1-\alpha}\frac{1-\beta_2}{1-\beta_1-\beta_2}. 
\end{equation*}
Thus, 
\begin{equation*}
  \lambda_2 = \frac{1-\alpha}{1-\beta_2} \lambda_1 = \frac\gamma{1-\beta_1-\beta_2}. 
\end{equation*}


We want of course that $\lambda_1 < \mu_1$ and $\lambda_2 < \mu_2$.
With the above expressions this leads to conditions on $\alpha$,
$\beta_1$ and $\beta_2$. Note that we have three parameters, and two
equations; there is not a single condition from which the stability
can be guaranteed.

If $\alpha\uparrow 1$, the arrival rate at node $1$ explodes. If
$\beta_1=0$ no jobs are sent from node 2 to node 1.

\end{solution}
\end{exercise}
\begin{exercise}
Zijm.Ex.2.2.8
\begin{solution}
  Yes, the network remains a Jackson network. By Burke's law, see Exercise~\ref{ex:burke}, the
  departure process of each node is Poisson. In one of the earlier
  questions we derived that splitting (also known as thinning) and
  merging Poisson streams again lead to Poisson streams. The
  departures from node $j$ to node $k$ forms a thinned Poisson
  stream. The external arrivals plus internal arrivals are merged into
  one Poisson stream, hence the arrivals at a station also form a Poisson stream.

  Observe that the exponentiality of the service times and external
  inter-arrival times and Burke's law are essential for the argument.
\end{solution}
\end{exercise}

\begin{exercise}
  (Hall 5.22). At a large hotel, taxi cabs arrive at a rate of 15 per
  hour, and parties of riders arrive at the rate of 12 per
  hour. Whenever taxicabs are waiting, riders are served immediately
  upon arrival. Whenever riders are waiting, taxicabs are loaded
  immediately upon arrival. A maximum of three cabs can wait at a time (other cabs must go elsewhere).
  \begin{enumerate}
  \item Let $p_{ij}$ be the steady-state probability of there being $i$ parties of riders and $j$ taxicabs waiting at the hotel. Write the state transition equation for the system. 
  \item Calculate the expected number of cabs waiting and the expected number of parties waiting.
  \item Calculate the expected waiting time for cabs and the expected waiting for parties. (For cabs, compute the average among those that do not go elsewhere.)
  \item In words, what would be the impact of allowing four cabs to wait at a time?
  \end{enumerate}
  \begin{hint}
This is really a neat problem. Please spend serious time on it to
solve before looking at the answer. It requires some ingenuity on your part.  Try to adapt the ideas behind Figure 2.2 of Zijm to this case.
  \end{hint}
    \begin{solution}
Let $p_{ij}$ be the fraction of time that the system contains $i$
riders and $j$ taxi cabs. I assume that all members of a party of
riders can be served by a single cab (that is, the parties do not
exceed the capacity of a cab and all members of a party have the same
destination). For clarity, write $\mu$ for the rate at which cabs
arrive, and $\lambda$ for the arrival rate of parties of riders.  Then
the transitions are as in the figure below. Suppose first that there
are $3$ taxi cabs. When a group arrives (at rate $\lambda$), there is
one taxi less, and so on, until there are no more taxi's
left. Finally, if yet more groups arrive, they have to wait. When a
new taxi arrives, the number of groups is reduced by one, and so on,
until there are $3$ taxi's waiting and no groups of people.


    \begin{center}

\begin{tikzpicture}[->,>=stealth',shorten >=1pt,auto,node distance=1.8cm,
                    semithick]
  \node[state] (0) {$p(0,3)$} ;
  \node[state] (1) [right of=0] {$p(0,2)$};
  \node[state] (2) [right of=1] {$p(0,1)$};
  \node[state] (3) [right of=2] {$p(0,0)$};
  \node[state] (4) [right of=3] {$p(1,0)$};
  \node[state] (5) [right of=4] {$p(2,0)$};
  \node[state] (6) [right of=5] {$p(\cdot, 0)$};

\path 
 (0) edge [bend left] node {$\lambda$} (1)
 (1) edge [bend left] node {$\mu$} (0)
 (1) edge [bend left] node {$\lambda$} (2)
 (2) edge [bend left] node {$\mu$} (1)
 (2) edge [bend left] node {$\lambda$} (3)
 (3) edge [bend left] node {$\mu$} (2)
 (3) edge [bend left] node {$\lambda$} (4)
 (4) edge [bend left] node {$\mu$} (3)
 (4) edge [bend left] node {$\lambda$} (5)
 (5) edge [bend left] node {$\mu$} (4)
 (5) edge [bend left] node {$\lambda$} (6)
 (6) edge [bend left] node {$\mu$} (5)
;
\end{tikzpicture}
      
    \end{center}

From this figure, we see that
\begin{align*}
\lambda p_{0,3} &= \mu p_{0,2} \\
(\lambda+\mu) p_{0,2} &= \mu p_{0,1} + \lambda p_{0,3}\\
(\lambda+\mu) p_{0,1} &= \mu p_{0,0} + \lambda p_{0,2}\\
(\lambda+\mu) p_{0,0} &= \mu p_{1,0} + \lambda p_{0,1}\\
(\lambda+\mu) p_{1,0} &= \mu p_{2,0} + \lambda p_{0,0}\\
(\lambda+\mu) p_{2,0} &= \mu p_{3,0} + \lambda p_{1,0}\\
\end{align*}
And so on. Thus, it is left to compute $p_{ij}$. Observe from this
scheme, or the above figure, that the situation with the taxi's
correspond to an $M/M/1$ queue, only the states have a `different
name'. Let $q$ be the number of jobs in an M/M/1 queue. Some thought
will reveal that the queueing system with cabs and parties can be
mapped to an equivalent M/M/1 queueing system. In fact, consider the
following table
\begin{center}
\begin{tabular}{ccc}
$j$ & $i$ & $q$\\
3&         0 &         0\\
2 &        0&          1\\
1 &        0&          2\\
0&         0&          3\\
0&         1&          4\\
0&         2&          5\\
\end{tabular}
\end{center}
and so on. Therefore, in general, it must be that 

\begin{equation*}
q = 3 - j +i.
\end{equation*}
From the M/M/1 queue we know right away that $p_q = \rho^q
(1-\rho)$.  With the above relation we can therefore immediately find
that $p_{ij} = \rho^{3-j+i}(1-\rho)$, save that $i$ and
$j$ must satisfy the constraints imposed by the model.

Second, the expected number of cabs waiting must be 
\begin{equation*}
1p_{0,1} + 2 p_{0,2} + 3p_{0,3}
\end{equation*}
and the expected number of parties waiting must be $\sum_{j=1}^\infty j p_{j,0}$

\begin{pyconsole}
labda = 12. # per hour
mu = 15. # per hour
rho = labda/mu

def p(i,j):
    q  = 3 - j + i
    return rho**q*(1.-rho)

\end{pyconsole}
Expected number of  cabs waiting:
\begin{pyconsole}
Lc = sum(j*p(0,j) for j in range(0,4)) 
# Recall this sums up to 4, not including 4
Lc
  
\end{pyconsole}


To compute the expected number of parties waiting we formally have to
sum to infinity. Rather than doing the algebra, I chose to truncate
the summation at an $i$ such that $\rho^i \ll 1$, i.e.,
negligible.  Truncating at 30 seems reasonable enough:

\begin{pyconsole}
trunc = 30
rho**trunc
\end{pyconsole}

At second thought this is not yet really small. 

\begin{pyconsole}
trunc = 50
rho**trunc
\end{pyconsole}


This is better. Now go for what we want to know:

\begin{pyconsole}
Lp = sum(i*p(i,0) for i in range(trunc))
Lp
\end{pyconsole}

For the last part: This is tricky. I first, naively, computed $W_q = L_c/\mu$. This
seems to make sense, as cabs arrive at rate $\mu$, so that this
expression follows from a standard application of Little's
law. However, this is wrong, of course. When using Little's law to
relate the number of jobs in queue (i.e., in the M/M/1 queue) and the
queueing time we need to use $\lambda$, not
$\mu$. Similarly (and more formally by the mapping developed in
part a), for our cab system we also need to use $\lambda$.

\begin{pyconsole}
Wq = Lc/labda
Wq
\end{pyconsole}

Thinking in templates is often useful, but makes one sloppy\ldots

What would be the impact of allowing 4 cabs? Funny question, and with the above, trivial to answer.

\begin{pyconsole}
def p(i,j):
    q  = 4 - j + i
    return rho**q*(1.-rho)
  
\end{pyconsole}

\begin{pyconsole}
Lc = sum(j*p(0,j) for j in range(0,4))
Lc

Lp = sum(i*p(i,0) for i in range(trunc))
Lp
  
\end{pyconsole}
    \end{solution}
\end{exercise}



\begin{comment}
  
\begin{exercise}[use=false]
  Include an question that shows that it is necessary to check the
  guess for the product form solution everywhere on the state space,
  that is, in the interior, but also on the boundaries.
  \begin{solution}
    TBD
  \end{solution}
\end{exercise}

\begin{exercise}[use=false]
  Apply the formula's of Zijm.Section 2.4 to a tandem queueing network.
  \begin{solution}
    TBD.
  \end{solution}
\end{exercise}

\begin{exercise}[use=false]
  Include a case for Section 2.4.
  \begin{solution}
    TBD.
  \end{solution}
\end{exercise}
\end{comment}


\Closesolutionfile{hint}
\Closesolutionfile{ans}

\opt{solutionfiles}{
\subsection*{Hints}
\input{hint}
\subsection*{Solutions}
\input{ans}
}
%\clearpage



%%% Local Variables:
%%% mode: latex
%%% TeX-master: "../queueing_book"
%%% End:
