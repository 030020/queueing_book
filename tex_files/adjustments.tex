\section{Non-preemptive Interruptions, Server Adjustments}


\subsection*{Theory and Exercises}

\Opensolutionfile{hint}
\Opensolutionfile{ans}

%See section 1.11 of Zijm's book for the theory.

Besides failures, service interruptions  may occur between any two jobs, for instance when a machine needs to adjustments. Such interruptions are \emph{non-preemptive} as they do not preempt a job, 
but  \emph{between} two jobs.  In this section we develop a simple model to incorporate the effects on the SCV of the service times of non-preemptive outages, so that we again can estimate waiting times with the $G/G/1$ waiting time formula. 


Suppose that an interruption occurs on average between $B$ jobs and the adjustment time is a random variable $T$. Let the average net service time of a job be $\E{S_0}$. 
\begin{exercise}
  Show that the average \recall{effective} processing time $S_e$, i.e., the time the service is occupied with processing jobs including adjustments, is 
  \begin{equation*}
    \E{S_e} = \E{S_0} + \frac{\E{T}} B.
  \end{equation*}
  \begin{hint}
    What fraction of the repair time $\E T$ `belongs' to one job?
  \end{hint}
  \begin{solution}
    The total service time spent on a batch of size $B$ is $B \E{S_0} + \E T$. The expected effective time per job is then the average time per job, i.e.,  $(B \E{S_0}+\E T)/B$. 
  \end{solution}
\end{exercise}

Thus the effective server load is
\begin{equation*}
  \rho = \lambda \E{S_e}.
\end{equation*}

\begin{exercise} 
Show that the minimal number~$B$ of jobs between adjustments must satisfy the constraint
  \begin{equation*}
 B>\frac{\lambda \E T}{1-\lambda \E{S_0}}.
  \end{equation*}

\begin{hint}
 If the batch sizes $B$ are small, relatively much time is
  spent on repairs. 2: If the repair time $T$ is constant, then the
  number of orders that arrive is Poisson distributed. 3: Given a
  batch size $B$, find the time to produce such a batch, then include
  a repair time $T$ to compute the length of one production cycle. The
  expected number of jobs that arrive during such a cycle should not exceed the number of jobs that are served during the cycle.
\end{hint}
\begin{solution}
The expected number of arrivals during a repair is
\begin{equation*}
    \lambda \E T.
  \end{equation*}
To see this, note that  the number of arrivals $N$ during a constant time is Poisson. Thus,
\begin{equation*}
  \E{N} = \E{ \E{N\given T}} = \E{ \E{\lambda T\given T}} = \lambda \E{\E{T \given T}} = \lambda \E{T}.
\end{equation*}

The number of jobs that arrive, on average, during one
  production cycle must be smaller than the total amount of jobs that
  can be served, on average, during one cycle.   The number of arrivals during the repair is $\lambda \E{T}$. The number of arrivals during serving the batch is $\lambda B \E{S}$. Thus, 
  \begin{equation*}
    \lambda ( B \E S + \E{T}) \leq B.
  \end{equation*}
  When equality holds here, the system is critically loaded. As we
  discussed before, that it not a good idea.

Finally, bring $B$ to one side.
\end{solution}
\end{exercise}

To compute average waiting times in queue, we need the variance of the effective service times, which requires to make a model of the probability of a repair occurring between any two jobs. One particularly simple model is to assume that such adjustments occur geometrically distributed  (i.e., have the memoryless property in discrete time) between any two jobs with a mean of $B$ jobs between any two repairs.  Thus, the probability of a repair between two jobs is $p=1/B$. 

\begin{exercise}
  Show that, with this model of interruptions, $E{S_e} = \E{S_0} + \E T/ B.$
  \begin{hint}
    What is the  time a job occupies the server when a repair occurs just before the job? What is the probability that this occurs?
  \end{hint}
  \begin{solution}
    \begin{equation*}
      \E{S_e} = \E{S_0} \frac{B-1}B + (\E{S_0} + \E T) \frac 1B.
    \end{equation*}
  \end{solution}
\end{exercise}




The next step is to compute $\E{S_e^2}$. 
\begin{exercise}
  Show that
  \begin{equation*}
    \E{S_e^2} = \E{S_0^2} + 2\frac{\E{S_0}\E{T}} B + \frac{\E{T^2}}B.
  \end{equation*}
  \begin{solution}
  \begin{equation*}
    \begin{split}
    \E{S_e^2} 
&= \E{S_0^2}\frac{B-1}B + \E{(S_0+T)^2}\frac 1B \\
&= \E{S_0^2}\frac{B-1}B + \E{S_0^2} \frac 1 B + 2 \E{S_0} \E T\frac 1B + \E{T^2}\frac 1B
    \end{split}
  \end{equation*}
  \end{solution}
\end{exercise}

\begin{exercise}
  Use the above to find that
  \begin{equation*}
    \V{S_e} = \V{S_0} + \frac{\V{T}} B + (B-1)\left(\frac{\E T}{B}\right)^2.
  \end{equation*}
  \begin{solution}
    \begin{equation*}
      \begin{split}
\V{S_e} 
&=\E{S_e^2} - (E{S_e})^2 \\
&= \E{S_0^2} + 2 \E{S_0} \E T\frac 1B + \E{T^2}\frac 1B  \\
&\quad - (\E{S_0})^2 - 2\E{S_0}\E T \frac 1 B - (\E T)^2\frac1{B^2}\\
&=  \V{S_0} + ((\E{T^2} - (\E T)^2)\frac 1 B + (\E T)^2\left(\frac 1B - \frac 1{B^2}\right).
      \end{split}
    \end{equation*}
  \end{solution}
\end{exercise}


\begin{exercise}
A machine requires an adjustment with average $5$ hours and standard deviation of $2$ hours. Jobs arrive as a Poisson process with rate $\lambda=9$ per working day. The machine works two $8$ hour shifts a day. Work not processed on a day is carried over to the next day. Job service times are 1.5 hours, on average, with standard deviation of $0.5$ hour. Interruptions occur on average after $30$ jobs.

Compute the average waiting time in queue.
\begin{hint}
  Get the units right. First compute the load, and then compute the rest.
\end{hint}
\begin{solution}
  First we determine the load. 
  \begin{pyconsole}
B=30
ES0 = 1.5
labda = 9/(2*8) # arrival rate per hour
ET=5
ESe=ES0+ET/B
ESe
rho = labda*ESe
rho
  \end{pyconsole}
So, at least the system is stable.

\begin{pyconsole}
VS0 = 0.5*0.5
VT = 2*2
VSe = VS0 + VT/B + (B-1)*(ET/B)**2
VSe
Ce2 = VSe/(ESe*ESe)
Ce2
\end{pyconsole}

And now we can fill in the waiting time formula
\begin{pyconsole}
Ca2=1 # Poisson arrivals
EW = (Ca2+Ce2)/2 * rho/(1-rho) * ESe
EW  
\end{pyconsole}
\end{solution}
\end{exercise}

Observe that with these formulas we can obtain quantitative insights into the effects of reducing adjustment times, or the variability of these adjustments times. In particular, sensitivity analysis is very relevant for practical cases to find out the dominating parameters. 



\begin{comment}

\begin{exercise}
Zijm.Ex.1.11.6
 \begin{solution}
   Cleaning times will be pretty constant. Changing dies, or other
   machine parts, is also typically quite predictable, although it can
   take a lot of time, in particular in case a crane or other heavy
   machinery is needed to replace parts. If the machine require
   temporary adjustments, then the variation in setup times may be
   quite a bit higher.
\end{solution}
\end{exercise}

\begin{exercise}
Zijm.Ex.1.11.7
 \begin{solution}
   Then the effective service times, and in particular, $C_e^2$ will
   be quite a bit bigger. It is preferable to avoid such a situation. 

   Mathematically, it is only given that $N_s$ is a random
   variable. As, however, this does not state anything about its
   distribution, we cannot make any general claim. The intent of the
   problem is to have you check the relevant formulas and notice that
   the variance of $N_s$ appears in the formulas.
\end{solution}
\end{exercise}


  
\begin{exercise}
  Suppose a single machine workstation has to process $n$ different
  product families. Jobs of family $i$ arrive as a Poisson process
  with rate $\lambda_i$.  Service times are exponentially distributed
  with average $\mu_i^{-1}$ for family $i$. A setup of time $S$ is
  required if the machine switches from one family to another. If the
  machine works according to a cyclic schedule, i.e., produce family
  1, then 2, then 3, and so on until family $n$, Can you find batch
  sizes $B_i$ for family $i$ such that an average waiting time can be
  guaranteed?
  \begin{solution}
    \TBD. 

    This problem can be generalized and has been widely studied as one
    of the core problems in stochastic machine scheduling. For
    instance, what to do if the setup times are family-dependent, or
    even more general, dependent on the sequence in which the families
    are planned? This is a very hard problem. 
  \end{solution}
\end{exercise}
\end{comment}

\Closesolutionfile{hint}
\Closesolutionfile{ans}
\subsection*{Hints}
\input{hint}
\subsection*{Solutions}
\input{ans}
\clearpage

%%% Local Variables:
%%% mode: latex
%%% TeX-master: "booktest"
%%% End:
