

\section{General Job Shop Manufacturing Systems}
\label{sec:general-job-shop}

Perhaps the term `class' needs some clarification. A network with a
single class of jobs means that the all jobs have the same service
distribution. In multi-class networks jobs of different classes
typically have different service distributions. Note that we do not
discuss multi-class queueing networks in this course.

\begin{question}
  It is mentioned in the text that there are $5M+M^2$ input parameters
  required to characterize an $M$ station network of $G/G/1$ queues. What are these parameters?
  \hint{ Check the formulas in the synthesis carefully.}
  \begin{solution}
    \begin{enumerate}
    \item The routing matrix $P$ is $M$ times $M$, hence contains
      $M^2$ parameters, but often most of them will be zero.
    \item $\lambda_{0i}$ is the arrival rate of external jobs to station $i$, hence $M$ parameters.
    \item $C_{0i}^2$ is the SCV of the interarrival times at station
      $i$, hence $M$ parameters.
    \item $C_{si}^2$ is the SCV of the service times at station $i$,
      hence $M$ parameters.
    \item The expected service times $\E S_i$ at station $i$, 
      hence $M$ parameters.
    \item The number of servers $c_i$ at station $i$, hence $M$
      parameters.
    \end{enumerate}
  \end{solution}
\end{question}


\begin{question}
Consider a tandem of single server work stations. If you would be
given some money to reduce the variance of the service times of one
machine.  Which machine would you prefer? 
\begin{solution}
  From the $G/G/1$ waiting time formula, it is obvious that by
  reducing the variability of the service times the average waiting
  time will also reduce.  The results to be developed in this section
  will show that when the service times become less variable, the
  inter departure times also become less variable. As a a consequence,
  the downstream machine sees a more regular arrival process, hence,
  by the $G/G/1$ waiting time, will have less average waiting
  time. But also its departure process will go down, so that the
  waiting time at the next station also decreases, and so on. Thus,
  the entire chain benefits from reducing variability of processing
  times at the first station. 

  However, all depends of course on the cost. If it is very expensive
  to reduce variability at the first station, or if this station is
  already very `regular', it might be better to improve the second
  station. Thus, as always, what to do depends on the context,
  financially and technically.
\end{solution}
\end{question}


\begin{question}
  Assume that in a network of five stations the last station is the
  bottleneck. Under what conditions would you invest in this last
  work station?
  \begin{solution}
    If it would be cheap. However, as the previous question should
    show you, by reducing variability at the first station the waiting
    time of all stations along the line goes down. Thus, by improving
    the first or second machine the total average waiting time may go
    down more than by only improving the situation at the last,
    bottleneck, machine. 

    Thus, we need models, simulations and/or experiments to find out
    what is the best option.
  \end{solution}
\end{question}


\begin{question}[use=false]

In this section we apply the formulas of $G/G/1$ and $G/G/c$ queue to
analyze the waiting times at a paint factory. The intent of this
section is to show how to approach realistic problems and how queueing
models can be used to obtain insight into the situation and provide
ideas how and where to improve the situation. Since often only gross
data is available, it is of no use to make very detailed models; the
data that is typically needed to estimate the parameters of
fine-grained models is not available. 


Company X produces paint according to a make-to-stock (MTS), also
called production-to-stock, strategy. It makes paint in two steps:
mixing and packaging. In the first step, the ingredients are mixed in
large vessels. Once mixed, the vessels are moved to an intermediate
stocking point. Here the paint waits (in the vessels) until the
planner decides to package the paint. Company X has two packaging
machines: machine A is dedicated to filling small cans (1L), machine B
to large cans (5L). Jobs for machine A (B) line up in front of machine
A (B), i.e., each machine has its own queue.

\begin{enumerate}
\item 
  Make a drawing of the production situation, including the flow of the
  orders, the processing steps and stocking/queueing points.
\end{enumerate}

Orders of type A, i.e,. orders that have to be filled in small cans,
arrive at rate $\lambda_A = 2$ per working day, orders of type B
arrive at rate $\lambda_B =1$ per working day. The processing time at
the mixer is, on average, $\E S_m= 2$ hours per order, independent of
the type of paint. The average packaging time of an order of type A is
$\E S_A = 3$ hours, while the average packaging time of orders of type B
is $\E S_B=7$ hours. The paint shop has 8 hours available per day, and
any unfinished work at the end of a day is carried over to the next
working day.

The data, as obtained from measurements taken by the procedure as
described in Section~\ref{sec:gg1}, show that the
inter-arrival times of orders and the processing times at the machines
have moderate variability, i.e. $C_a^2 \approx 1$ and $C_s^2\approx 1$. 

  \begin{enumerate}[resume]
  \item Show that the average queueing time at the mixing station is $\E W_Q = 6$ hours.
  \item Show that the average cycle time at the mixing station is $\E W = 8$ hours.
  \end{enumerate}

If we are interested in the average cycle time of an arbitrary job,
whether it is an A or a B job, we can use at least three models.  
\begin{enumerate}
\item Model 1. Model all stations as $M/M/1$ queues. As, by
  assumption, the mixing station is an $M/M/1$ queue, its departure
  process is exponential with rate $\lambda_A + \lambda_B$, and of
  these departures, the type A jobs go to filling station A. Thus, we
  can compute the waiting times at stations $A$ and $B$ also. Finally,
  the average waiting time is given by the weighted average of the
  waiting times of the A and B jobs. One weak point of this model is
  the assumption of the exponentiality of the processing times at all
  stations. We do know that, in real systems, processing times are not
  really exponential, but the extent to which a deviation of this
  assumption leads, is not so clear.
\item Model 2. Model the second processing step as \emph{one} filing
  station, i.e., put an imaginary box around both filling stations A
  and B and model it as one station with \emph{one} machine, but
  such that the capacity of this one machine is equal to the capacity
  of the two original filling machines. This is a simple model, but it
  approximates the two filling machines with one machine. 
\item Model 3. Model the second processing step as \emph{one} filing
  station, i.e., put an imaginary box around both filling stations A
  and B and model it as one station with \emph{two identical}
  machines. This modeling assumption might have a large influence,
  since we know that multi-server queueing systems behave quite a bit
  different from single-server queueing systems. On the other hand,
  one basic assumption for the $G/G/c$ queue is that all machines have
  the same capacity, i.e., processing speed. This assumption is not
  satisfied in the paint case.
\end{enumerate}
All in all, all three models capture part of the real production
system, but they are also partly wrong.  Perhaps we need all three
models to analyze the case, and see to what extent these models give
similar or dissimilar results.


\begin{enumerate}[resume]
\item   Use the ideas of Section 2.2. of Zijm's book to carry out the
  details of Model 1.  In other words, solve the stationary
  distribution for the system with Poisson arrivals and two
  exponential servers, one working at rate $\mu_1$ the other at rate
  $\mu_2$.
\end{enumerate}

For Model 2, observe that the arrival process at the filling station,
being the departure process of the mixing machine, is
Poisson. However, the service times of type A and type B are not the
same. Thus we model the queueing situation at the second stations as a
$G/G/1$ queue.
  \begin{enumerate}[resume]
  \item   What assumption of Jackson networks is not met, which forces us to
  use the G/G/1 queue as a model for the filling station?
\item To apply the $G/G/1$ waiting time formula, we need to convert
  the processing times of the job types into one weighted processing
  time and an approximation for the SCVs, i.e., for $C_a^2$ and
  $C_s^2$.  Motivate that
  \begin{equation*}
\E S_c = \frac{\lambda_A}{\lambda_A + \lambda_B} \E S_A + \frac{\lambda_B}{\lambda_A + \lambda_B} \E S_B = \frac23 3 + \frac13 7= 13/4.
  \end{equation*}
  is a reasonable approximation for the average processing time at the
  filling stage, i.e., for a combined job.
\item For model 2, why do we need to use the $G/G/1$ queue and not an
  $M/M/1$ queue?
\item As a simple approximation for $C_s^2$, take
  \begin{equation*}
  C_s^2 = 
\frac{\lambda_A}{\lambda_A + \lambda_B} C_s^2(A) +
  \frac{\lambda_B}{\lambda_A + \lambda_B} C_s^2(B).
  \end{equation*}
  Find an improvement for this estimate.
\item What can we take for $C_a^2$ at the filling station?
\item Compute the average waiting time at the filling station based on
  the above approximations.
\item Use Model 3 and carry out similar computations to find another
  estimate of the waiting times.
  \item If  $\E W_Q$ would be 10 days, what would be the average
    number of jobs waiting in this intermediate storage point?
\item Suppose that the marketing department proposes to start a
  campaign to sell type A paint at a slightly lower price. They
  estimate that for at least one year the demand for type A paint will
  increase with about 1 order per day. What is your opinion about this
  campaign?
\end{enumerate}

\begin{solution}
  \begin{enumerate}
  \item 
\includegraphics{mixingpaint}

To make this, I used the \texttt{asymptote} program, a very nice
drawing package that makes a pdf file that is right away importable in
\LaTeX. This is the code, in case you are interested.

\begin{lstlisting}
settings.outformat = "pdf";
size(8cm);

path[] q = (0,0.25)--(2,0.25)--(2,-0.25)--(0,-0.25);
path[] q = q^^(1.75,0.25)--(1.75,-0.25);
path[] q = q^^(1.5,0.25)--(1.5,-0.25);
path[] q = q^^(1.25,0.25)--(1.25,-0.25);

path[] queue = q;
path[] queueA = shift((5,1))*q;
path[] queueB = shift((5,-1))*q;

label("$\lambda$", (-1,0.25), align=N);
draw((-2,0)--(0, 0), arrow=ArcArrow(TeXHead));
draw(queue);

label("$\lambda_A$", (3.75,0.75), align=N);
draw((2,0)--(5, 1), arrow=ArcArrow(TeXHead));
label("$A$", (6,1.25), align=N);
draw(queueA);
draw((7,1)--(9, 1), arrow=ArcArrow(TeXHead));

label("$\lambda_B$", (3.75,-1.75), align=N);
draw((2,0)--(5, -1), arrow=ArcArrow(TeXHead));
label("$B$", (6,-0.75), align=N);
draw(queueB);
draw((7,-1)--(9, -1), arrow=ArcArrow(TeXHead));
\end{lstlisting}
\item 
  $\lambda_A=2$ per day and $\lambda_B = 1$ per day, hence $1$ per 8
  hours. $\E S_{m}=2$.
  $\rho=(\lambda_A+\lambda_B)t_{e,m} = 3/8 \times 2=3/4$. Fill in
  the $G/G/1$1 queueing time formula to get $\E W_Q=6$ hours.  
\item  $\E W = \E W_Q + \E S_{m} = 8$ hours.
\item     TBD
\item   The job in a Jackson network are assumed to have the same
  distribution at one station. In the paint case, this is not true,
  jobs of type A have different service times than jobs of type B.  
\item  A fraction $\lambda_A/(\lambda_A + \lambda_B)$ of jobs is of
  type A and requires $t_A$ service time on average. The answer above
  is just the weighted average of service times.
\item  The combined station has, by assumption, just one machine that
  serves a combination of type A and type B jobs.  The service time
  distribution of the combined jobs is most surely no longer
  exponential. It may taken to be like this:
  \begin{equation*}
    \P(S_c \leq t) = 
\frac{\lambda_A}{\lambda_A + \lambda_B} \P(S_A \leq t) + 
\frac{\lambda_B}{\lambda_A + \lambda_B} \P(S_B \leq t),
  \end{equation*}
  where $S_C$ is the distribution of the combined jobs.  If $S_A$ and
  $S_B$ have exponential distribution, then $S_c$ is known to have
  hyper exponential distribution.
\item  Use the $G/G/2$ formula. Thus, use the fact that $\rho$ has to be
  raised to some power of $m$, where $m=2$ is the number of
  machines. Many students tend to  use the G/G/1 formula.
\item     TBD
\item  For an improvement, use the definition of the SCV. Let $S_c$ be
  the process time of a combined job: 
  \begin{equation*}
    \text{Var} S_c = \frac{\lambda_A}{\lambda_A + \lambda_B} \text{Var}S_A +
  \frac{\lambda_B}{\lambda_A + \lambda_B} S_B,
  \end{equation*}
  provided the service times of type A and type B jobs are
  independent. Now $\text{Var} S_A = C_s^2(A) \E S_A^2$, and likewise
  for $\text{Var}S_B$.
\item   Use Little's law to convert this to mean queue length,
  $\E L_Q= \E W_Q \times \lambda$, where
  $\lambda=\lambda_A + \lambda_B = 3$ per day. Hence $\E L_Q = 30$.
  Don't divide by $\E S_c$! Note, $1/\E S_c$ is not the arrival rate!
\item This is not a good idea as the utilization becomes larger than
  one at station A: $(2+1)\cdot 3/8>1$.
\end{enumerate}
\end{solution}
\end{question}

\begin{question}[use=false]
  In this question we develop a simple model to analyze the sojourn
  time at a large pancake restaurant. The restaurant has three large
  dining rooms and a terrace. One a busy summer days there can be more
  than 600 customers present at the same time. The kitchen has a stove
  with 12 burners to make pancakes. Baking one pancake takes about 2
  minutes. One out of five pancakes get an extra topping after
  baking. The chef bakes the pancakes, the assistant handles the
  toppings. We model this situation as a two-station queueing system:
  the chef is the server at the first station, the assistant is the
  server at the second station. The chef is a bit clumsy with his
  tools: about 10\% of the pancakes falls on the floor, hence needs to
  be baked again.  We want to compute the throughput time, i.e, the
  total sojourn time, for this system.

  \begin{enumerate}
  \item Suppose we would model the stove as a $M/M/1$ queueing system
    in which the (single) server works at twelve times the rate of the
    one of the burners. Would this be a good model? 
  \item What is the routing matrix for this queueing system?
\item What is the vector $\vec \gamma$ of external arrivals?
\item Can you find an approximation for the net arrival rate
  $\lambda_1$ and $\lambda_2$ at the two stations? 
  \end{enumerate}


  Up to now we have not yet discussed an important detail, the
  distribution of the inter departure times at the chef. 

  \begin{enumerate}[resume]
  \item Why is this departure process most probably not Poisson?
  \item What type of $C_a^2$ do you expect to see at the assistant? 
  \item How would you estimate the waiting time at the assistant?
  \item Estimate the arrival rate $\alpha$, and compute the utilization of  the chef.
  \item What formula would you use to estimate the average waiting time at the chef?
  \end{enumerate}

    \begin{solution}
      \begin{enumerate}
      \item 
      No, not really, it takes 2 minutes to bake a pancake, no matter
      the amount of burners available.  Thus, only when the queues are
      really long, the difference in sojourn time between a fast single server and multiple, parallel, slow servers would be small. 
    \item   The chef drops 10\% of the pancakes. These have to be re-baked. Of
  the remaining 90\% of the pancakes one out five gets a topping,
  thus, $0.9\cdot0.2=0.18$ goes to the assistant. The assistant
  delivers quality work, so no pancakes have to be processed by him
  again. Thus,
\begin{equation*} 
P=
  \begin{pmatrix}
    1/10 & 9/50 \\
0 & 0
  \end{pmatrix}
\end{equation*}
\item Only the chef gets new orders.  $\vec \gamma = (\alpha, 0)$,
where  $\alpha$ is the arrival rate of new orders.
\item The assistant receives
  20\% of all pancakes: $\lambda_2 = \alpha/5$.  As the chef has to
  redo 10\% of the pancakes, $\lambda_1 = \frac{10}{9}\alpha$.

  We can also solve $\vec \gamma + \vec \lambda P = \vec \lambda$.
  This gives, $\lambda_1 = \alpha + \lambda_1/10$, and
  $\lambda_2 = 9/50 \lambda_1$. Solving this gives the same answers.
\item       The baking of a pancake takes about 2 minutes, and it quite
      regular. Thus, the queueing situation at the chef resembles the
      $M/D/12$ queue. However, due to the rework, there is more variability than in the $M/D/12$ queue. 
    \item       I would guess it would be smaller than 1. By how much is not so
      easy to say. Simulation of the $M/D/12$ queue would be most helpful here.
    \item       As both  $C_a^2$ and $C_s^2$ at the second station are smaller than 1, in particular $C_s^2$ must be quite a bit smaller than 1, I would use the formula:
\begin{equation*}
  \E W_Q \approx \frac{C_a^2 + C_s^2}2 \frac \rho{1-\rho} ES_2,
\end{equation*}
where $\rho=\lambda_2E(S_2)$.
\item       Lets concentrate on a busy day. There are 600 people
      present. Assume they stay one hour on average. By Little's law
      we see that the arrival rate of people is $600/60=10$ per
      minute. Assume half of the people order a pancake, then $\alpha=5$ per minute. As the average baking time is 2 minutes, $\rho = \lambda_1 \E S_1/12 = 5\cdot 10/9\cdot 2/12= 25/27$.
Realize that $\rho=\lambda\E S_1/12$ is not OK. 
\item Since we have many customers, independently arriving and ordering pancakes, it is reasonable to model the arrival process as a Poisson process. Thus, as an approximation
  \begin{equation*}
    \E W_Q \approx \frac12 \frac{\rho^{\sqrt{2(12+1)} - 1}}{12(1-\rho)} E(S_1).
  \end{equation*}
where $\rho=\lambda_1 E(S_1)$.
\end{enumerate}
\end{solution}
\end{question}


\begin{question}[use=false]
  \begin{enumerate}
  \item Three M/M/1 FIFO stations in tandem
  \item $\gamma_1 = 2/h$,  $\gamma_2 = 3/h$, $\gamma_3 = 1/h$
  \item All jobs from station 1 go to station 2
  \item jobs that arrive at 2 externally and those from station 1 move
    on to station 3
  \item 1 out of 4 jobs that complete service at station 3 need rework
    at station 2. The other  jobs leave the network
  \item The jobs that arrive at station 2 from station 3 leave the
    network after service at station 2.
  \item Determine the total arrival rates, i.e., $\lambda$, at each
    station. 
  \item Determine the routing matrix $P$. 
  % \item Suppose that at station 2, the average service time jobs from
  %   station~1 require 1 hour of service, the jobs that arrive directly
  %   at station 2 require 15 minutes of service, and the `rework' jobs
  %   require 5 minutes of work.
  \item Assume that $\E S_1 = 5$ min, $\E S_2 = 8$ min, and $\E S_3 =
    5$ min. Determine the expected number of jobs in the system. 
  \item Determine the density of the throughput time in the network
    of the jobs that arrive at station 1. Realize that part of these
    jobs have to undergo rework at station 2.
  \end{enumerate}
\begin{solution}
\TBD.
  \begin{enumerate}
  \item 
  The arrival rate at station 1 is $2$ per hour. Hence, $\rho_1 =
  \lambda_1/\mu_1 = \lambda_1 \E S_1 = 2/60 * 5 = 10/60 =
  1/10$. Hence, $L_{s,1} = \rho_1/(1-\rho_1)$.

  At station 3, only the jobs arrive from station 2 that did not come
  from station 3, plus $\gamma_3$. Hence, $\lambda_3 = \gamma_1 +
  \gamma_2 + \gamma_3 = 6$ per hour. Service time is $\E S_3 = 5$
  min. Hence, $\rho_3 = 6/60 * 5 = 30/60 = 1/2$. Then, $L_{s,2}$
  follows straightaway.

  Now station 2. This receives all jobs from station 1,
  i.e.,$\gamma_1$, plus $\gamma_2$, plus 25\% of the jobs that move on
  to station 3. Hence, $\lambda_2 = \gamma_1 + \gamma_2 + 1/4
  \lambda_3 = \gamma_1 + \gamma_2 + 1/4(\gamma_1 + \gamma_2 +
  \gamma_3) = 2 + 3 + (1+2+3)/4 = 5+6/4 = 5 + 3/2 = 6 1/2 = 13/2$ per
  hour. Therefore, $\rho_2 = \lambda_2 \E S_2 = 13/2/60 * 8 = 13*8/120
  = 104/120 = 52/60 = 26/30 = 13/15$.
\item   Simulate the network, and use the empirical distribution function to
  estimate the distribution function of the throughput times.
  \end{enumerate}
\end{solution}
\end{question}
