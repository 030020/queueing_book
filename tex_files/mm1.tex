\section
[$M/M/1$ queue]
{$\mathbf{M/M/1}$ queue}
\label{sec:mm1}

In the $M/M/1$ queue, one server serves jobs arriving with
exponentially distributed interarrival times and each requiring an
exponentially distributed processing time.  With Eq.~(\ref{eq:25}),
i.e., $\lambda(n)p(n)= \mu(n+1)p(n+1)$ we can derive a number of
important results for this queueing process.

Recall from Section~\ref{sec:queu-proc-as} that we can construct the
$M/M/1$ queue as a reflected random walk where the arrivals are
generated by a Poisson process $N_\lambda(t)$ and the departures
(provided the number in the $L(t)>0$) are generated according to the
Poisson process $N_\mu(t)$. Since the rates of these processes do not
depend on the state of the random walk, or the queue for that matter,
$\lambda(n)=\lambda$ and $\mu(n)=\mu$ for all $n$. Thus, \eqref{eq:25}
reduces to
\begin{equation*}
  p(n+1) = \frac{\lambda(n)}{\mu(n+1)} p(n) = \frac{\lambda}{\mu} p(n) = \rho p(n),
\end{equation*}
where we use the definition of the load $\rho=\lambda/\mu$. Since this
holds for any $n\geq 0$, it follows with recursion that
\begin{equation*}
  p(n+1) = \rho^{n+1} p(0).
\end{equation*}
Then, from the normalization condition
\begin{equation*}
1=  \sum_{n=0}^\infty p(n) = p(0)\sum_{n=0}^\infty \rho^n = \frac{p(0)}{1-\rho},
\end{equation*}
so 
\begin{align}\label{eq:23}
p(0) &=1-\rho, &   p(n) &=  (1-\rho)\rho^{n}.
\end{align}

How can we use these equations? First, note that $p(0)$ must be the
fraction of time the server is idle. Hence, the fraction of time the
server is busy, i.e., the utilization, is
\begin{equation*}
  1-p(0) = \rho = \sum_{n=1}^\infty p(n).
\end{equation*}
Here the last equation has the interpretation of the fraction of time
the system contains at least 1 job. Next, in the exercises below we
derive that
\begin{equation}\label{eq:el}
  \E L = \frac \rho{1-\rho}.
\end{equation}
and 
\begin{equation}\label{eq:pn}
  \P{L\geq n} = \rho^n.
\end{equation}

Let us interpret these expressions. First of all, we only need
estimates of $\lambda$ and $\mu$ to characterize the utilization of
the server, the average queue length and the queue length
distribution. In fact, only the ratio $\rho=\lambda/\mu$ is required.
As such, assuming that interarrival times and services are
exponentially distributed, by measuring the fraction of time the
server is busy, we obtain an estimate for $1-p(0)$, hence for $\rho$.
With this, we can estimate $\E L$ and $\P{L\geq n}$.  Next, the fact
that $\E L \sim (1-\rho)^{-1}$ for $\rho\to 1$ implies that the
average waiting time \recall{increases asymptotically fast} to
infinity when $\rho\to1$.  In practical terms, this means that
striving for a high load results in (very) long average waiting
times. As a consequence, this formula tells us that, in the design and
operation of queueing systems, we should avoid the situation in which
$\rho\approx 1$. Thus, we need to make a trade-off between the server
utilization $1-p(0) = \rho$ and the waiting time. In high load
regimes, the server is used efficiently, but the queue lengths are
excessive. In any sensible system, we should keep $\rho$ quite a bit
below 1, and accept lower utilization then might seem necessary.

The expression for $\P{L\geq n}$ shows that the probability that the
queue length exceeds some threshold decreases exponentially fast (for
any reasonable $\rho$). Moreover, if we make the simple assumption
that customers decide to leave (or rather, not join) the system when
the queue is longer than $9$ say, then $\P{L\geq 10} = \rho^{10}$ is
an estimator of the fraction of customers lost. Again, for the $M/M/1$
queue, the knowledge of $\rho$ suffices to compute this estimate.

% In the context of inventory theory these equations are particularly
% useful, see one of the questions below.

\paragraph{Supermarket Planning}

Let us consider the example of cashier planning of a supermarket to
demonstrate how to use the tools we developed up to now. Out of
necessity, our approach is a bit heavy-handed---Turning the example
into a practically useful scheme requires more sophisticated queueing
models and data assembly---but the present example contains the
essential analytic steps to solve the planning problem.

The \emph{service objective} is to determine the minimal service
capacity such that the fraction of the time the queue length exceeds
10 is less than 1\%.

The next step is to find the \emph{relevant data}. For a supermarket
this is relatively easy: the cash registers track all customers
payments. Thus, we know the number of customers that left the shop,
hence entered the shop. (We neglect the time customers spend in the
shop.) Based on these data we can make a \emph{demand profile}: the
customer arrival rate per hour, c.f. Figure~\ref{fig:loadprofile}. It
is also easy to find the service distribution from the cash
registers. The first item scanned after a payment determines the start
of a new service, and the payment closes the service. (As there is
always a bit of time between the payment and the start of a new
service we might add 15 seconds, say, to any service.)

\begin{figure}[t]
  \centering
\begin{tikzpicture}[scale=.7]
 	%axis
	\draw[->] (0,0) -- coordinate (x axis mid) (13.5,0);
    	\draw[->] (0,0) -- coordinate (y axis mid) (0,5.5);
    	%ticks
    	\foreach \x in {0,...,13}
        \pgfmathsetmacro{\my}{int(\x+8)}
     		\draw (\x,1pt) -- (\x,-3pt)
			node[anchor=north] {$\my$};
    	\foreach \y in {0,...,5}
        \pgfmathsetmacro{\my}{int(\y*40)}
     		\draw (1pt,\y) -- (-3pt,\y) 
     			node[anchor=east] {\my}; 
%labels      
\node[below=0.6cm] at (x axis mid) {hour};
\node[rotate=90, left=1.2cm] at (y axis mid) {$\lambda$};

\draw (0,1)--(1,1);
\draw (1,2)--(2,2);
\draw (2,2.6)--(3,2.6);
\draw (3,2.8)--(4,2.8);
\draw (4,3.)--(5,3.);
\draw (5,3.1)--(6,3.1);
\draw (6,2.7)--(7,2.7);
\draw (7,1.9)--(8,1.9);
\draw (8,2.5)--(9,2.5);
\draw (9,3.3)--(10,3.3);
\draw (10,3.5)--(11,3.5);
\draw (11,2.3)--(12,2.3);
\draw (12,1.2)--(13,1.2);
\end{tikzpicture}
  \caption{A  demand profile of the arrival rate $\lambda$ modeled as constant over each hour.}
  \label{fig:loadprofile}
\end{figure}

To keep things simple, we model the arrival process as Poisson with an
arrival rate that is constant during an hour and is specified by the
demand profile, and we take the service time distribution as
exponential with a mean of $1.5$ minutes. We also \emph{model} the
behavior of the multi-server as a single fast server. Thus, we neglect
any differences between a station with, for instance, 5 cashiers and a
single server that works 5 times as fast as a normal cashier.  As yet
another simplification, we change the objective somewhat such that the
number of jobs in the queueing system should not exceed 10. (Thus, the
objective is no longer formulated in terms of queue lengths.)

To determine the required number of servers per hour let us consider
the hour from 12 to 13. In the demand profile we see that the arrival
rate in this hour is $\lambda_{12}= 120$ customers per hour. The
objective $\P{L>10}\leq 1\%$ translates into $\rho^{11}\leq 1\%$,
hence, $\rho \leq 0.67$. Using that $\E{S}=1.5$ minutes, we need the
number of servers $c$ to be such that
\begin{equation*}
  \frac{120}{60} 1.5 \frac 1 c = \rho < 0.66,
\end{equation*}
hence $c\approx5$.  Interestingly, for this model we only need to be
concerned with $\lambda \E S$ to meet the objective. Therefore $c$ is
very simple to compute by the result
\begin{equation*}
  c \geq \lambda \frac{\E S}{0.66}  = \lambda 2.3 \approx 2.5 \lambda.
\end{equation*}
where $\lambda$ is the arrival rate (per minute, \emph{not} per hour).
Thus, the conversion of the demand profile to a \emph{load} profile is
trivial: divide the hourly arrival rate by $60$ and multiply by
2.5. The load profile tells us the minimal number of cashiers needed
per hour.

Now we need to \emph{cover the load profile with service shifts}. This
is typically not easy since shifts have to satisfy all kinds of rules,
e.g.: after 2 hours of work a cashier should take a break of at least
10 minutes; a shift length must be at least four hours, and not longer
than 9 hours including breaks; when the shift is longer than 4 hours
it needs to contain at least one break of 30 minutes; and so on. These
shifts also have different costs, e.g.: shifts with hours after 18h
are more expensive per hour; when the supermarket covers traveling
costs, short shifts have higher marginal travel-ling costs; and so on.

The usual way to solve the covering problem is by means of an integer
problem. First generate all (or a subset of the) allowed shift types
with associated starting times. For instance, suppose only 4 shift
plans are available
\begin{enumerate}
\item $++-++$
\item $+++-+$
\item $++-+++$
\item $+++-++$,
\end{enumerate}
where a $+$ indicate a working hour and $-$ a break of an hour. Then
generate shift types for each of these plans with starting times
$8$am, $9$am, and on on, until the end of the day. Thus, a shift type
is a shift plan that starts at a certain hour. Let $x_i$ be the number
of shift type $i$ and $c_i$ the cost of this type. Write $t\in s_i$ if
hour $t$ is covered by shift type $i$.  Then the problem is given by
\begin{equation*}
  \min \sum_i c_i x_i.
\end{equation*}
such that 
\begin{equation*}
  \sum_i x_i \1{t \in s_i} \geq \frac{\lambda_t}{20}
\end{equation*}
for all hours $t$ the shop is open and $\lambda_t$ is the demand for
hour $t$.

Conceptually it is easy how to make this approach more accurate and
realistic. The queueing model is too simple, so as start we could
replace it by the multi-server model (with exponential service times)
we will discuss in Section~\ref{sec:mnmn1}. If the data tells us that
the exponential distribution for the service times is very inaccurate,
we might use the results of Section~\ref{sec:gg1}, however, this only
provides us with the mean waiting times for the $M/G/c$ queue, not the
queue length distribution, which is a drawback. (The analytic results
for the queue lenght distribution of the $G/G/c$ queue are not so easy
and also not very accurate. Whether it makes sense to prefer these
expressions over the closed-form results of the $M/M/c$ queue is not
so easy to say. It depends on the context.) Finally, the objective
should depend on the number of cashiers that are open. Assuming that
customers spread evenly over the queues, the objective is such that no
queue is longer than 3 say. The objective is therefore such that the
queue length should be smaller than $3$ times the number of cashiers
$c$. Finally, in many shops, the cashier capacity is made dependent on
the queue length; when the queues are long, stockers open extra
cashier positions until the first server becomes idle. Thus, as there
is spare capacity, the factor $2.5 \lambda$ can be made smaller,
perhaps set to $2\lambda$. However, this factor only changes the load
profile, not the method to solve it.

Clearly, it does not require much effort to make the model very
complicated, and it is easy to make it complicated to the extent that
the model becomes useless. Mind: a good model is simple (but not too
simple). So, notwithstanding the weaknesses of the above model, it is
pretty powerful already. 

\subsection*{Exercises}
\label{sec:exercises}

\Opensolutionfile{hint}
\Opensolutionfile{ans}

\begin{question}
  Derive expression~\ref{eq:el} for the average number of jobs in an
  $M/M/1$ queue.  
  \begin{solution}
\begin{align*}
\E L &= \sum_{n=0}^\infty n p(n) \\
&= \sum_{n=0}^\infty \sum_{i=1}^n \1{i\leq n} p(n)  && n=\sum_{i=1}^n \1{i\leq n}\\
&= \sum_{n=0}^\infty \sum_{i=1}^\infty   \1{i\leq n} p(n)  && i>n\implies \1{i\leq n} = 0\\
&= \sum_{i=1}^\infty \sum_{n=0}^\infty  \1{i\leq n} p(n) &&\text{Fubini} \\
&= \sum_{i=1}^\infty \sum_{n=i}^\infty p(n) && n < i \implies \1{i\leq n}=0 \\
&= \sum_{i=1}^\infty \sum_{n=i}^\infty (1-\rho)\rho^n && p(n) = (1-\rho)\rho^n \\
&= \sum_{i=1}^\infty \sum_{n=0}^\infty (1-\rho)\rho^{n+i} && n\to n+i \\
&= \sum_{i=1}^\infty (1-\rho)\rho^i \sum_{n=0}^\infty \rho^n && \rho^{n+i}=\rho^i \rho^n\\
&= \sum_{i=1}^\infty (1-\rho)\rho^i \frac1{1-\rho}   \\
&= \sum_{i=1}^\infty \rho^i \\
&= \sum_{i=0}^\infty \rho^{i+1} && i\to i+1\\
&= \rho \sum_{i=0}^\infty \rho^i \\
&= \frac{\rho}{1-\rho}.
\end{align*}
Note that, since the summands are positive, we can use Fubini's theorm
to justify the interchange of the summations.

  \end{solution}
\end{question}

\begin{question}
  Derive expression~\ref{eq:pn} for the excess probability $\P{L\geq n}$, i.e., the probability that a long queue occurs. 
  \begin{hint}
\hint{$\P{L\geq n} = \sum_{k\geq n} p(k)$.}
  \end{hint}
  \begin{solution}
    \begin{equation*}
      \begin{split}
 \P{L\geq n} 
 &= \sum_{k=n}^\infty p(k) = \sum_{k=n}^\infty p(0)\rho^k = (1-\rho)\sum_{k=n}^\infty \rho^k \\
 &= (1-\rho)\rho^n \sum_{k=0}^\infty\rho^k = (1-\rho) \rho^n \frac1{1-\rho}.
\end{split}
\end{equation*}
\end{solution}
\end{question}


\begin{question}[use=false]
\label{q:basestock}

merge with inventory model of mnm1.tex file, or in the batchqueues.tex file.

Customers of, for example, a fast-food restaurant or a production
facility, prefer to be served from stock. For this reason such
companies often use a `produce-up-to' policy: When the on-hand
inventory $I$ is equal or lower than some threshold $r$, the company
produces items until the inventory level equals $r+1$ again. The level
$r$ is the known as the reorder level.

Suppose that customers arrive as a Poisson process with rate $\lambda$
and the production times of single items are i.i.d. and exponentially
distributed with parameter $\mu$. Assume also that customers that
cannot be served from on-hand stock are backlogged, that is, they wait
until their item has been produced. What are the average on-hand
inventory level, the average number of customer in backlog, and the
fraction of customers that are backlogged?

If $h$ is the cost per item per unit time in stock and $b$ the cost
per customer per unit time in backlog and $\bar b$ the cost per customer
backlogged, what are the average costs of using a reorder level $r$? 

What is the optimal reorder level $r^*$, i.e., the level that achieves
minimal average cost?

\begin{hint}
  
\hint{Realize that the inventory level $I(t)$ at time $t$ can be
  modeled as $I(t) = r+1-L(t)$, where $L$ is the number of jobs in an
  $M/M/1$ queue.}
\end{hint}

  \begin{solution}
  \end{solution}
\end{question}


\Closesolutionfile{hint}
\Closesolutionfile{ans}
\subsection*{Hints}
\input{hint}
\subsection*{Solutions}
\input{ans}

%%% Local Variables:
%%% mode: latex
%%% TeX-master: "book"
%%% End:
