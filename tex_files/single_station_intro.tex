In this chapter, we start with a discussion of the exponential
distribution and the related Poisson process, as these concepts are
perhaps the most important building blocks of queueing theory. With
these concepts we can specify the arrival and service processes of
customers, so that we can construct queueing processes and define
performance measures to provide insight into the (transient and
average) behavior of queueing processes. As it turns out, these
queueing processes can often be easily implemented as computer programs, thereby
allowing to use simulation to analyze queueing systems. We then
continue with developing models for various single-station queueing
systems in steady-state, which is, in a sense to be discussed later,
the long-run behavior of a stochastic system. In the analysis we use
sample-path arguments to count how often certain events occur as
functions of time. Then we define probabilities in terms of limits of
fractions of these counting processes. Another useful aspect of
sample-path analysis is that the definitions for the performance
measures are entirely constructive, hence by leaving out the limits,
they provide expressions that can be used right away in statistical
analysis of (simulations of) queueing systems. Level-crossing
arguments will be of particular importance as we use these time and
again to develop recursions by which we can compute steady-state
probabilities of the queue length or waiting time process.


