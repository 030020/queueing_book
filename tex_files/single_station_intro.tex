In this chapter, we start with a discussion of the Poisson process and
the related exponential distribution, as these concepts are perhaps
the most important building blocks of queueing theory. With these
concepts we can specify the arrival and service processes of
customers, with which in turn we can construct queueing processes and
define performance measures to provide insight into the (transient and
average) behavior of queueing processes. As it turns out, these
queueing processes can often be easily implemented as computer
programs, thereby allowing us to use simulation to analyze queueing
systems. We then continue with developing models for various
single-station queueing systems in steady-state. In the analysis we
use sample-path and level-crossing arguments to count how often
certain events occur as a function of time. Then we define
probabilities in terms of limits of fractions of these counting
processes. Like this the performance measures can be explicitly
computed for the statistical analysis of (simulations of) queueing
systems.


