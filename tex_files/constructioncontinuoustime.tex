\section
[Construction of the $G/G/1$ Queueing Process in Continuous Time]
{Construction of the $\mathbf{G/G/1}$ Queueing Process in Continuous Time}
\label{sec:constr-gg1-queu}

\subsection*{Theory and Exercises}

\Opensolutionfile{hint}
\Opensolutionfile{ans}

In the previous section we considered time in discrete `chunks',
minutes, hours, days, and so on. For given numbers of arrivals and
capacity per period we use a set of recursions~\eqref{eq:31} to
compute the departures and queue length per period. Another way to
construct a queueing system is to consider inter-arrival times between
consecutive customers and the service times each of these customers
require. With this we obtain a description of the queueing system in
continuous time.  The goal of this section is to develop a set of
recursions for the $G/G/1$ single-server queue.

Assume we are given, as basic data, the \recall{arrival process}
$\{A(t); t\geq 0\}$: the number of jobs that arrived during $[0,t]$.
Thus, $\{A(t); t\geq 0\}$ is a \emph{counting process}.

From this arrival process we can obtain various other interesting
concepts, such as the arrival times of individual jobs. Specially, if
we know that $A(s) = k-1$ and $A(t) = k$, then the arrival time $A_k$
of the $k$th job must lie somewhere in $(s,t]$. Thus, from $\{A(t)\}$, we can define
\begin{equation}\label{eq:27}
  A_k = \min\{t: A(t) \geq k\},
\end{equation}
and set $A_0 = 0$. Once we have the set of arrival times $\{A_k\}$,
the \recall{inter-arrival times} $\{X_k, k=1, 2, \ldots\}$ between
consecutive customers can be constructed as
\begin{equation}
  X_k = A_k - A_{k-1}.
\end{equation}

Often the basic data consists of the inter-arrival times
$\{X_k; k=1,2,\ldots\}$ rather than the arrival times $\{A_k\}$ or
the number of arrivals $\{A(t)\}$. Then we  construct the arrival
times as
\begin{equation*}
  A_k = A_{k-1} + X_k,
\end{equation*}
with $A_0 = 0$.  From the arrival times $\{A_k\}$ we can, in
turn, construct the arrival process $\{A(t)\}$ as 
\begin{subequations}
  \label{eq:2}
\begin{equation}
  A(t) = \sum_{k=1}^\infty \1{A_k \leq t},
\end{equation}
where $\1{}$ is the indicator function. Thus, in the above we count
all arrivals that occur up to time~$t$. Another, equivalent, way to
define $A(t)$ is
\begin{equation}
  A(t) = \max\{k: A_k \leq t\}.
\end{equation}
\end{subequations}

Clearly, we see that from the inter-arrival times $\{X_k\}$ it is
possible to construct $\{A_k\}$ and $\{A(t)\}$, and the other way
around, from $\{A(t)\}$ we can find $\{A_k\}$ and $\{X_k\}$.
Figure~\ref{fig:waitingtimegg1} shows these relations graphically. To
memorize it may be helpfull to write it like this:
\begin{align*}
  A_k : \N \to \R, \quad{\text{job id (integer) to arrival time (real number)}}, \\
  A(t) : \R\to \N, \quad{\text{time (real number) to number of jobs (integer)}}.
\end{align*}

\begin{exercise}
 What are  the meanings of $A_{A(t)}$ and $A(A_n)$?
 \begin{solution}
  $A(t)$ is the number of arrivals during $[0,t]$. Suppose that
    $A(t) = n$. This $n$th job arrived at time $A_n$. Thus, $A_{A(t)}$
    is the arrival time of the last job that arrived before or at time
    $t$. In a similar vein, $A_n$ is the arrival time of the $n$th
    job. Thus, the number of arrivals up to time $n$, i.e., $A(A_n)$,
    must be $n$.
  \end{solution}
\end{exercise}

\begin{exercise}
  Would it make sense to define $A(t)$ as $\min\{k : A_k \geq t\}$?
  \begin{hint}
Compare this to the definition in (\ref{eq:2}).
  \end{hint}
  \begin{solution}
    Suppose $A_3 = 10$ and $A_4 = 20$. Take $t=15$. Then
    $\min\{k : A_k \geq 15\} = 4$ since $A_3 < t=15 < A_4$. On the
    other hand $\max\{k : A_k \leq t\} = 3$. And, indeed, at time $t=15$, 3 jobs arrived, not 4. So defining $A(t)$ as $\min\{k : A_k \geq t\}$ is not ok. 
  \end{solution}
\end{exercise}

To compute the departure times $\{D_k\}$ we proceed in stages. The
first stage is to construct the \recall{waiting time in queue}
$\{W_{Q,k}\}$ as seen by the arrivals. In
Figure~\ref{fig:waitingtimegg1} observe that the waiting time of the
$k$th arrival must be equal to the waiting time of the $k-1$th
customer plus the amount of \recall{service time} required by job
$k-1$ minus the time that elapses between the arrival of job $k-1$ and
job $k$, unless the server becomes idle between jobs $k-1$ and $k$.
In other words,
\begin{equation}\label{eq:56}
  W_{Q,k} = [W_{Q,k-1} + S_{k-1}-X_k]^+,
\end{equation}
where $[x]^+ = \max\{x, 0\}$. If we set $W_{Q,0}=0$, we can compute
$W_{Q,1}$ from this formula, and then $W_{Q,2}$ and so on.

\begin{figure}[t]
  \centering

\begin{tikzpicture}[scale=1,
  open/.style={shape=circle, fill=white, inner sep=1pt, draw, node contents=},
  closed/.style={shape=circle, fill=black, inner sep=1pt, draw, node contents=}]
\draw (-1,0)--(12,0);

\draw 
node (c1) at (0,3) {}
node (Wk) at (1,2) [open, label={}]
(c1) to (Wk);

%\node[below] (Ak) at (1,-0.4) {$A_k$};
\draw[dotted] (1,0) -- (Wk) node[midway,fill=white,rotate=90] {$W_{Q,k-1}$};
\node (Sk) at (1,5) [closed, label={}];
\node (Ak1) at (5,1) [open, label={}];
%\node[above] at (4,0) {$D_k$};
%\draw[dotted,<->] (1, -0.25)--(3,-0.25) node[midway, fill=white] {$W_{Q,k}$};
%\draw[dotted,<->] (3, -0.25)--(6,-0.25) node[midway, fill=white] {$S_k$};
\draw[dashed] (1, 2)--(3,0); 

\draw[|-|]
node[left] (c1) at (1,-0.5) {$A_{k-1}$}
node[right] (c2) at (6,-0.5) {$D_{k-1}$}
(c1) -- (c2)
node[midway, fill=white] {$W_{k-1}$};

\draw[dotted] (Wk) -- (Sk) node[midway,fill=white,rotate=90] {$S_{k-1}$};
\draw[dashed] (Ak1) -- (6,0); % node[midway,fill=white,rotate=90] {$S_k$};

\draw[-] (Sk) to (Ak1);
\node (Sk1) at (5,3) [closed, label={}];
\draw[dotted] (Ak1) -- (Sk1) node[midway,fill=white,rotate=90] {$S_{k}$};
\draw[dotted] (5,0) -- (Ak1);
\draw (Sk1) -- (8,0);

\draw[|-|]
node[left] (c1) at (5,-1) {$A_{k}$}
node[right] (c2) at (8,-1) {$D_{k}$}
(c1) -- (c2)
node[midway, fill=white] {$W_{k}$};


\draw (9,2) -- (11,0);

\draw[dotted]
node (c1) at (9,0) [open, label={}]
node (c2) at (9,2) [closed, label={}]
(c1) -- (c2)
node[midway, fill=white, rotate=90] {$S_{k+1}$};

\draw[|-|]
node[left] (c1) at (9,-1.5) {$A_{k+1}$}
node[right] (c2) at (11,-1.5) {$D_{k+1}$}
(c1) -- (c2)
node[midway, fill=white] {$W_{k+1}$};

\draw[|-|]
node[left] (c1) at (1,-1.5) {$A_{k-1}$}
node[right] (c2) at (5,-1.5) {$A_{k}$}
(c1) -- (c2)
node[midway, fill=white] {$X_{k}$};
\end{tikzpicture}
\caption{Construction of the $G/G/1$ queue in continuous time. The
  sojourn time $W_{k}$ of the $k$th job is sum of the work in queue
  $W_{Q,k}$ at its arrival epoch $A_k$ and its service time $S_k$;
  its departure time is then $D_k=A_k + W_k$. The waiting time of job
  $k$ is clearly equal to $W_{k-1}-X_{k}$. We also see that job $k+1$
  arrives at an empty system, hence its sojourn time
  $W_{k+1}=S_{k+1}$. Finally, the virtual waiting time process is
  shown by the lines with slope $-1$.}
  \label{fig:waitingtimegg1}
\end{figure}

The time job $k$ \emph{leaves the queue and moves on to the server} is
\begin{equation*}
 D_{Q,k} = A_k + W_{Q,k},
\end{equation*}
because a job can only move to the server after its arrival plus the
time it needs to wait in queue.  Note that we here explicitly use the
FIFO assumption.

Right after the job moves from the
queue to the server, its service starts.  Thus, $D_{Q,k}$ is the epoch
at which the service of job $k$ starts. After completing its service,
the job leaves the system. Hence, the \recall{departure time of the
  system} is
\begin{equation*}
  D_k = D_{Q,k} + S_k.
\end{equation*}

The \recall{sojourn time}, or \emph{waiting time in the system}, is the
time a job spends in the entire system. With the above relations we
see that
\begin{equation}
  W_k = D_k - A_k = D_{Q,k} + S_k -A_k = W_{Q,k} + S_k,
\end{equation}
where each of these equations has its own interpretation. 

A bit of similar reasoning gives another recursion for $W_k$:
\begin{align}
  \label{eq:59}
  W_{Q,k} &= [W_{k-1} - X_k]^+, &
  W_{k} &= W_{Q,k} + S_k = [W_{k-1} - X_k]^+ + S_k,
\end{align}
from which follows a recursion for $D_k$
\begin{equation}
  D_k = A_k + W_k.
\end{equation}
This in turn specifies the departure process $\{D(t)\}$ as
\begin{equation*}
  D(t) = \max\{k; D_k \leq t\} = \sum_{k=1}^\infty \1{D_k \leq t}.
\end{equation*}

\begin{exercise}
 Assume that $X_1=10$, $X_2=5$, $X_3=6$ and $S_1 = 17$,
    $S_2=20$ and $S_3=5$, compute the arrival times, waiting times in
    queue, the sojourn times and the departure times for these three
    customers.
  \begin{hint}
     The intent of this exercise is
      to make you familiar with the notation.

      BTW, such simple test cases are also very useful to test
      computer code. The numbers in the exercise are one such simple
      case. You can check the results by hand; if the results of the
      simulator are different, there is a problem.
    \end{hint}
  \end{exercise}

\begin{exercise}
  Suppose that $X_k\in\{1,3\}$ such that $\P{X_k=1}=\P{X_k=3}$ and
  $S_k\in\{1,2\}$ with $\P{S_k=1}=\P{S_k=2}$. If $W_{Q,0}=3$, what are
  the distributions of $W_{Q,1}$ and $W_{Q,2}$?  
  \begin{hint}
Use Eq.~(\ref{eq:59}).
\end{hint}
\begin{solution}  First find the distribution of $Y_k:=S_{k-1}-X_k$ so that we can write
  $W_{Q,k}=[W_{Q,k-1}+Y_k]^+$.  Use independence of $\{S_k\}$ and $\{X_k\}$:
\begin{align*}
  \P{Y_k=-2} &=\P{S_{k-1}-X_k=-2} = \P{S_{k-1}=1, X_k=3} = \P{S_{k-1}=1}\P{X_k=3} = \frac14.
\end{align*}
Dropping the dependence on $k$ for ease, we get
\begin{align*}
  \P{Y=-2} &=\P{S-X=-2} = \P{S=1, X=3} = \P{S=1}\P{X=3} = \frac14\\
  \P{Y=-1} &=\P{S=2}\P{X=3} = \frac14\\
  \P{Y=0} &=\P{S=1}\P{X=1} = \frac14\\
  \P{Y=1} &=\P{S=2}\P{X=1} = \frac14.
\end{align*}
With this
  \begin{align*}
    \P{W_{Q,1} = 1} &=\P{W_{Q,0} + Y_0= 1} = \P{3 + Y_0 =1} = \P{Y_0=-2} =\frac14\\
    \P{W_{Q,1} = 2} &= \P{3 + Y =2} = \P{Y=-1}  =\frac14\\
    \P{W_{Q,1} = 3} &= \P{3 + Y = 3} = \P{Y=0}  =\frac14\\
    \P{W_{Q,1} = 4} &= \P{3 + Y = 4} = \P{Y=1}  =\frac14.\\
  \end{align*}
And, then
  \begin{equation*}
    \begin{split}
    \P{W_{Q,2} = 1} 
&=\P{W_{Q,1} + Y = 1} = \sum_{i=1}^4 \P{W_{Q,1} + Y = 1\given W_{Q,1}=i}\P{W_{Q,1}=i}\\
&=\sum_{i=1}^4 \P{i + Y = 1\given W_{Q,1}=i}\frac14
=\sum_{i=1}^4 \P{Y = 1-i\given W_{Q,1}=i}\frac14\\
&=\frac14\sum_{i=1}^4 \P{Y = 1-i} = \frac14(\P{Y = 0} + \P{Y=-1} +\P{Y=-2}) = \frac{3}{16}.
    \end{split}
  \end{equation*}

Typing the solution becomes  boring\ldots, let's use the computer.

%<<term=True>>=
\begin{pyconsole}
from lea import Lea

W = 3
S = Lea.fromVals(1,  2)
S
X = Lea.fromVals(1,  3)
X

# This is WQ1
W = Lea.fastMax(W + S - X, 0)
W

# This is WQ2
W = Lea.fastMax(W + S - X, 0)
W
\end{pyconsole}
%@
Great! Our handiwork matches with the computer's results. 

\end{solution}
\end{exercise}


Another set of recursions to compute the arrival and departure times for the single server queue is the following:
\begin{equation}
  \label{eq:45}
  \begin{split}
    A_k &= A_{k-1} + X_k, \\
    D_k &= \max\{A_k, D_{k-1}\} + S_k,\\
    W_k &= D_k - A_k.
  \end{split}
\end{equation}

\begin{exercise}  Why do the recursions Eq.~(\ref{eq:45}) work? 
  \begin{solution}
 Of course, the service of job $k$ cannot start before it
      arrives. Hence, it cannot leave before $A_k + S_k$. Therefore it
      must be that $D_k \geq A_k +S_k$. But the service of job $k$ can
      also not start before the previous job, i.e. job $k-1$, left the
      server. Thus job $k$ cannot start before $D_{k-1}$. To clarify
      it somewhat further, define $S_k'$ as the earliest start of job
      $k$. Then it must be that $S_k' = \max\{A_k, D_{k-1}\}$---don't
      confuse the earliest start $S_k'$ and the service time
      $S_k$---and $D_k = S_k' + S_k$.
    \end{solution}
\end{exercise}

\begin{exercise}
  Implement the above recursions in excel or some other computer
  program such as R, python, or julia, and check the results of the
  previous exercise.
    \begin{solution}
      You can check the files on \texttt{github} in the \texttt{progs}
      directory.
\begin{itemize}
\item \texttt{mm1.py} is the python implementation
\item \texttt{mm1.R} is the R implementation
\item \texttt{mm1.jl} is the julia implementation
\end{itemize}
Take your pick, and start playing with it. These examples are meant to
be simple to understand, not necessarily super efficient.
\end{solution}
\end{exercise}

Once we have the arrival and departure processes it is easy to compute
the \recall{number of jobs in the system} at time $t$ as
\begin{equation}\label{eq:14}
  L(t) = A(t) - D(t) + L(0),
\end{equation}
where $L(0)$ is the number of jobs in the system at time $t=0$;
typically we assume that $L(0)=0$. Thus, if we were to plot $A(t)$ and
$D(t)$ as functions of $t$, then the difference $L(t)$ between the
graphs of $A(t)$ and $D(t)$ tracks the number in the system, see
Figure~\ref{fig:atltdt}. 

The number in the system as seen by the $k$th arrival is defined as
\begin{equation}
  \label{eq:Lk}
  L_k = L(A_k), 
\end{equation}
since the $k$th job arrives at time $A_k$ and $\{L(s)\}$ is right-continuous.


\begin{figure}[t]
  \centering
\begin{tikzpicture}[scale=0.9,
  open/.style={shape=circle, fill=white, inner sep=1pt, draw, node contents=},
  closed/.style={shape=circle, fill=black, inner sep=1pt, draw, node contents=},
  soldot/.style={color=blue,only marks,mark=*}
]

\def\rightend{14}
\def\top{7}
\path [clip] (-1,-1) rectangle (\rightend,\top);

\draw[->] (-1,0) -- (\rightend,0);
\draw[->] (-0.5,-0.5) -- (-0.5, \top);

% arrivals
\def\lastx{0}
\foreach \x [count=\y, remember=\x as \lastx] in {1,3,4, 7, 9, 18} { 
  \node at (\lastx, 0) [below] {$A_{\y}$};
  \node (a) at (\lastx,\y) [closed] {};
  \draw[dotted] (\lastx,0) -- (a); 
  \draw (a)-- (\x,\y) node[open, label={}]; 
}
% draw first arrival. Since I want the circle to be open, I draw it at
% the end.
\node at (0, 0) [open] {};
\node at (5.5, 4) [fill=white] {$A(t)$};


% departures
\def\lastx{5}
\foreach \x [count=\y, remember=\x as \lastx] in {6, 8, 10, 11, 13,15} { 
  \draw[dotted] (\lastx,\y) -- (\lastx,0); 
  \draw (\lastx,\y) node[closed, label={}] -- (\x,\y) node[open, label={}]; 
  \node at (\lastx, -0.5) [below] {$D_{\y}$};
}
\node at (5, 0) [open] {};
\node at (12, 5) [fill=white] {$D(t)$};
  
\draw[dashed, <->] (3,2.5)--node[midway, fill=white] {$W_3$} (8,2.5);
\draw[dashed, <->] (7.5,2)--node[midway, fill=white,rotate=90] {$L(t)$} (7.5,5);

\end{tikzpicture}

  
  \caption{Relation between the arrival process $\{A(t)\}$, the
    departure process $\{D(t)\}$, the number in the system $\{L(t)\}$
    and the waiting times $\{W_k\}$.}
  \label{fig:atltdt}
\end{figure}

Observe that in a queueing system, jobs can be in queue or in
service. For this reason we distinguish between the number in the
system $L(t)$, the number in queue $L_Q(t)$, and the number of jobs in
service $L_s(t)$. If we know $D_Q(t)$, i.e. the number of jobs that
departed from the queue up to time $t$, then
\begin{equation*}
  L_Q(t) = A(t) - D_Q(t)
\end{equation*}
must be the number of jobs in queue. The above expressions for $L(t)$
and $L_Q(t)$ then show that the number in service must be
\begin{equation*}
L_s(t) = D_Q(t) - D(t) = L(t) - L_Q(t).
\end{equation*}



\begin{exercise}
Why don't we need separate notation for $D_s(t)$, the number
    of jobs that departed from the server? 
\begin{solution}
 Because $D_s(t) = D(t)$. Once customers leave the server,
    their service is completed, and they leave the queueing system.
  \end{solution}
\end{exercise}



\begin{exercise}
 Is $D_Q(t) \leq D(t)$ or $D_Q(t) \geq D(t)$?
\begin{solution}
 All customers that left the system must have left the
    queue. Thus, $D_Q(t) \geq D(t)$.
  \end{solution}
\end{exercise}



\begin{exercise}
 Why is $L(t) = L_Q(t) + L_s(t)$?
\begin{solution}
 $L_Q(t) = A(t) - D_Q(t)$. $L_S(t) = D_Q(t) - D(t)$. This is in
    line with the fact that $L(t) = L_Q(t) + L_s(t) = A(t) - D(t)$.
\end{solution}
\end{exercise}



\begin{exercise}
 Express $L_Q(t)$ and $L_s(t)$ in terms of $A(t)$, $D_Q(t)$ and $D(t)$.
\begin{solution}
 $L_Q(t) = A(t) - D_Q(t)$. $L_S(t) = D_Q(t) - D(t)$. This is in
    line with the fact that $L(t) = L_Q(t) + L_s(t) = A(t) - D(t)$.
\end{solution}
\end{exercise}

\begin{exercise}
 Consider a multi-server queue with $m$ servers. Suppose that
    at some time~$t$ it happens that $D_Q(t) - D_s(t) < m$ even though
    $A(t) - D_s(t) > m$. How can this occur? 
\begin{solution}
 In this case, there are servers idling while there are still
    customers in queue. If such events occur, we say that the server
    is not work-conservative.
\end{solution}
\end{exercise}


\begin{exercise} Show that $L(t) = A(t)-D(t)$ implies that 
$L(t) = \sum_{k=1}^\infty \1{A_k \leq t < D_k}$.
  \begin{hint}
  Use Boolean algebra.  Write, for notational ease,
    $A=\1{A_k \leq t}$ and $\bar A = 1- A = \1{A_k > t}$, and define
    something similar for $D$.  Then show that
    $A - D = A \bar D - \bar A D$, and show that $\bar A D
    =0$. Finally sum over $k$.
  \end{hint}

\begin{solution}
  \begin{equation*}
    \begin{split}
      L(t)
&= A(t) - D(t) \\
&= \sum_{k=1}^\infty \1{A_k \leq t} -  \sum_{k=1}^\infty \1{D_k \leq t} \\
&= \sum_{k=1}^\infty [\1{A_k \leq t} -  \1{D_k \leq t}].
    \end{split}
  \end{equation*}
  Write for the moment $A=\1{A_k \leq t}$ and
  $\bar A = 1- A = \1{A_k > t}$, and likewise for $D$. Now we can use
  Boolean algebra to see that
  $\1{A_k \leq t} - \1{D_k \leq t} = A-D = A(D+\bar D) -D = AD +
  A\bar D - D = A\bar D - D(1-A) = A\bar D - D \bar A$.
  But $D \bar A = 0$ since
  $D \bar A = \1{D_k \leq t} \1{A_k > t} = \1{D_k \leq t < A_k}$
  which would mean that the arrival time $A_k$ of the $k$th job would
  be larger than its departure time $D_k$. As $A \bar D = \1{A_k \leq t < D_k}$
  \begin{equation*}
    \begin{split}
      L(t)
&= \sum_{k=1}^\infty [\1{A_k \leq t} -  \1{D_k \leq t}] \\
&= \sum_{k=1}^\infty \1{A_k \leq t < D_k}.
    \end{split}
  \end{equation*}

  Boolean algebra is actually a really nice way to solve logical
  puzzles. If you are interested you can find some examples on my
  homepage. 
\end{solution}

\end{exercise}

\begin{exercise} Show  that $L(A_k)>0 \implies A_k \leq D_{k-1}$.
  \begin{hint} Use that $L(A_k)>0$ means that the system contains at least one
  job at the time of the $k$th arrival, and that $A_k \leq D_{k-1}$ means that
  job $k$ arrives before job $k-1$ departs.
  \end{hint}
\begin{solution}  In a sense, the claim is evident, for, if the system contains a
  job when job $k$ arrives, it cannot be empty. But if it is not
  empty, then at least the last job that arrived before job $k$, i.e.,
  job $k-1$, must still be in the system. That is, $D_{k-1} \geq A_k$. A more formal proof proceeds along the following lines. Using that $A(A_k) = k$ and $D(D_{k-1})= k-1$, 
  \begin{equation*}
    \begin{split}
      L(A_k) &> 0 \Leftrightarrow A(A_k) - D(A_k) > 0   \Leftrightarrow k - D(A_k) > 0 \Leftrightarrow k > D(A_k) \\
      &\Leftrightarrow k-1 \geq D(A_k) \Leftrightarrow D(D_{k-1}) \geq D(A_k) \Leftrightarrow D_{k-1} \geq A_k, 
    \end{split}
  \end{equation*}
  where the last relation follows from the fact that $D(t)$ is a
  counting process, hence monotone non-decreasing.
\end{solution}
\end{exercise}



\begin{exercise} Can you derive the following (algorithmic efficient) procedure to
    compute the number of jobs in the system as seen by arrivals,
  \begin{equation*}
    L_k = L_{k-1}+1 - \sum_{i= k-1 - L_{k-1}}^{k-1} \1{D_i< A_k}?
  \end{equation*} Why do we take $i=k-1-L_{k-1}$ in  the sum, and not $i=k-2-L_{k-1}$?
% \hint{ Realize that $L_k = k - 1 - D(A_k)$. 
% what happens to job k-1?
% }
\begin{solution}
  Let $L_{k} = L(A_{k}-)$, i.e., the number of jobs in the system as
  `seen by' job $k$. It must be that $L_{k}=k-1 - D(A_{k})$. To see
  this, assume first that no job has departed when job $k$
  arrives. Then job $k$ must see $k-1$ jobs in the system. In general,
  if at time $A_k$ the number of departures is $D(A_k)$, then the
  above relation for $L_k$ must hold. Applying this to job $k-1$ we get that $L_{k-1} = k-2 - D(A_{k-1})$. 

  For the computation of $L_k$ we do not have to take the departures
  before $A_{k-1}$ into account as these have already been
  `incorporated in' $L_{k-1}$.  Therefore,
  \begin{equation*}
    L_k = L_{k-1} + 1 - \sum_{i= k-1 - L_{k-1}}^{k-1} \1{D_i< A_k}.
  \end{equation*}

    Suppose $L_{k-1}=0$, i.e., job $k-1$ finds an empty system at its
    arrival and $D_{k-1}>A_{k}$, i.e., job $k-1$ is still in the
    system when job $k$ arrives. In this case, $L_{k}=1$, which checks
    with the formula.  Also, if $L_{k-1}=0$ and $D_{k-1}< A_k$ then
    $L_k = 0$. This also checks with the formula. 

    The reason to start at $k-1-L_{k-1}$ is that the number in the
    system as seen by job $k$ is $k-1 - D(A_k)$ (not
    $k-2-D(A_k)$). Hence, the jobs with index from
    $k-1-L_{k-1}, k-L_{k-1}, \ldots, k-1$, could have left the system
    between the arrival of job $k-1$ and job $k$.
\end{solution}
\end{exercise}


Finally, the \recall{virtual waiting time process} $\{V(t)\}$ is the
amount of waiting that an arrival would see if it would arrive at time
$t$. To construct $\{V(t)\}$, we simply draw lines that start at
points $(A_k, W_k)$ and have slope -1, unless the line hits the
$x$-axis, in which case the virtual waiting time remains zero until
the next arrival occurs.  Thus, the lines with slope $-1$ in
Figure~\ref{fig:waitingtimegg1} show (a sample path of) the virtual
waiting time.


Observe that, just as in Section~\ref{sec:constr-discr-time}, we have
obtained a set of recursions by which we can run a simulation of a
queueing process of whatever length we need, provided we have a
sequence of have inter-arrival times $\{X_k\}$ and service times
$\{S_k\}$.  A bit of experimentation with computer programs such as
$R$ or python will reveal that this is easy.





\begin{exercise}
  Provide a specification of the virtual waiting time process $\{V(t)\}$ for
    all $t$.
    \begin{hint}Make a plot of the function $A_{A(t)}-t$. What is the meaning of $V(A_{A(t)})?$ What is
$V(A_{A(t)} + A_{A(t)}-t$?
    \end{hint}
    \begin{solution}
      There is a funny way to do this. Recall from a previous exercise
      that if $A(t)=n$, then $A_n$ is the arrival time of the $n$th
      job. Thus, the function $A_{A(t)}$ provides us with arrival
      times as a function of $t$. When $t=A_{A(t)}$, i.e., when $t$ is
      the arrival time of the $A(t)$th job, we set
      $V(t) = V(A_{A(t)}) = W_{A(t)}$ i.e., the virtual waiting time
      at the arrival time $t=A_{A(t)}$ is equal to the waiting time of
      the $A(t)$th job. Between arrival moments, the virtual waiting
      time decreases with slope $1$, until it hits 0.  Thus,
      \begin{equation*}
        V(t) 
= [V(A_{A(t)}) + (A_{A(t)}-t)]^+= [W_{A(t)} + (A_{A(t)}-t)]^+
      \end{equation*}
      The notation may be a bit confusing, but it is in fact very
      simple. Take some $t$, look back at the last arrival time before
      time $t$, which is written as $A_{A(t)}$. (In computer code these
      times are easy to find.) Then draw a line with slope $-1$ from
      the waiting time that the last arrival saw.
    \end{solution}
\end{exercise}




\begin{exercise}  Try to extend the recursions~(\ref{eq:45})  to a situation in which one queue is
  served by two servers.  (This is a hard problem, only for the interested.)
    \begin{hint} The problem is
      that in a multi-server queueing systems, unlike for
      single-server queues, jobs can overtake each other: a small job
      that arrives after a very large job can still leave the system
      earlier. After trying for several hours, I obtained an inelegant
      method. A subsequent search on the web helped a lot. The
      solution below is a modification of N. Krivulin, `Recursive
      equations based models of queueing systems'. 
    \end{hint}
  \begin{solution}
The recursions for the two-server system are this: 
      \begin{equation*}
        \begin{split}
          A_k &= A_{k-1} + X_k, \\
          C_k &= \max\{A_k, D_{k-2}\} + S_k,\\
          M_k &= \max\{M_{k-1}, C_k\}, \\
          D_{k-1} &= \min\{M_{k-1}, C_k\}.
        \end{split}
      \end{equation*}
      Here, $C_k$ is the completion time of job $k$, and $\{D_k\}$ is
      a sorted list of departure times. Thus, $D_k$ is the $k$th
      departure time; recall this is not necessarily equal to the
      completion time $C_k$ of the $k$th job (as jobs may overtake
      each other). To understand the other equations, we reason like
      this.  By construction, $C_k > D_{k-m}$ (as $S_k >0$).
      Therefore, when we arrived at time $C_k$, $(k-m)$ jobs must have
      departed. Moreover, by construction, $M_k$ tracks the latest
      completion time of all $k$ jobs, hence, $M_{k-m+1}$ is the latest
      completion time of the first $k-m+1$ jobs. Therefore, if
      $C_k>M_{k-1}$, job $k$ must leave later than the latest of the
      jobs in $\{1,2,\ldots, k-1\}$.  Hence, the latest departure time
      of the jobs in $\{1, 2, \ldots, k-1\}$ jobs must be
      $M_{k-1}$. If however, $C_k<M_{k-1}$, then job $k$ leaves
      earlier than the latest of the jobs in $\{1,2,\ldots, k-1\}$. As
      $C_k>D_{k-2}$, it must be that $C_k > M_{k-2}$, because
      $D_{k-2}$ is latest departure of the jobs in
      $\{1,2,\ldots, k-2\}$, and this is also equal to $M_{k-2}$. As a
      consequence, if $C_k < M_{k-1}$, job $k$ is also the first job
      that leaves after $D_{k-2}$ ( provided of course that
      $C_{k+1} < C_k$). Thus, all in all
      $D_{k-1} = \min\{M_{k-1}, C_k\}$.

In an attempt to   extend the above to $m>2$ servers, I came up with this scheme:
      \begin{equation*}
        \begin{split}
          A_k &= A_{k-1} + X_k, \\
          C_k &= \max\{A_k, D_{k-m}\} + S_k,\\
          M_k &= \max\{M_{k-m+1}, C_k\}, \\
          D_{k-m+1} &= \min\{M_{k-m+1}, C_k\},
        \end{split}
      \end{equation*}
      but it is not correct. Can you find a counter example?
  \end{solution}
\end{exercise}



\begin{exercise} Suppose one server serves two
  queues, such that jobs in queue A are served with priority over jobs
  in queue B. Assume that service is not-preemptive.  Can you develop
  a similar set of recursions for each of the queues, or, otherwise,
  an algorithm? Can you extend this to cover the LIFO (Last-In-First-Out) queue?
(This is  a hard problem; only for the interested.)
  \begin{solution}
    This question came up in class. I don't think this can be
    constructed as a straightforward recursion, and a search on the
    web lead to nothing. In case you can find a recursion, please let
    me know.  

    While a straightforward recursion may not exist, it is not really
    difficult to code this queueing discipline. The key idea is to put
    all jobs into one queue, but sort the elements in the queue in
    order of priority. Every time the server becomes empty, it checks
    the head of the queue. If the queue is empty, it wait until the
    next arrival occurs. Otherwise, it starts the service of the first
    job in line. When a new job arrives, then identify its priority
    first, and then put it at the end of the jobs of the same
    priority.

    This type of sorting is known as lexicographic sorting, which is
    what we do when we build a dictionary. First we sort words in
    order of first letter, then the second letter, and so on. In case
    of sorting the queue, we first have to sort in order of priority,
    then, within the class of jobs with the same priority, sort in
    ascending order of arrival time.

    The python code is like this, where I include the FIFO case for
    comparison. As always, it is not obligatory to memorize the  code.

%<<evaluate=False>>=
\begin{pyverbatim}
class Fifo(Queue):
    def __init__(self):
        self.queue = SortedSet(key = lambda job: job.arrivalTime)

class Priority(Queue): # a priority queue
    def __init__(self, numServers = 1):
        self.queue = SortedSet(key = lambda job: job.p)
\end{pyverbatim}
%@

Interestingly, once we have a proper environment to carry out
simulations, changing the queueing discipline is a one-liner!.

With the above, an algorithm for the LIFO is
    straightforward. Rather then sorting the jobs in queue in
    ascending order of arrival time, just sort them in descending
    order of arrival time. 

% <<evaluate=False>>=
% class Lifo(Queue):
%     def __init__(self):
%         self.queue = SortedSet(key = lambda job: -job.arrivalTime)
% @

% Another interesting queueing rule is to sort in increasing job size;
% <<evaluate=False>>=
%   class SPTF(Queue): # shortest processing time first
%     def __init__(self):
%         self.queue = SortedSet(key = lambda job: job.serviceTime)
% @

Do you see how sort in descending order of job size (it just a matter
of putting a minus sign at the right place)?
  \end{solution}
\end{exercise}

\Closesolutionfile{hint}
\Closesolutionfile{ans}
\subsection*{Hints}
\input{hint}
\subsection*{Solutions}
\input{ans}
\clearpage
