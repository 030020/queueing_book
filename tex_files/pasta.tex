\section{Poisson Arrivals See Time Averages}
\label{sec:poisson-arrivals-see}

Suppose the following limit exists:
\begin{equation*}
  \pi(n) 
= \lim_{m\to\infty} 
\frac1m\sum_{k=1}^m \1{L(A_k-) = n},
\end{equation*}
where $\pi(n)$ is the long-run fraction of jobs that observe $n$
customers in the system at the moment of arrival.  It is natural to
ask whether $\pi(n)$ and $p(n)$ are related, that is, whether what
customers see upon arrival is related to the time-average behavior of
the system.


We can make some progress with this question by rewriting $\pi(n)$ in
the following way. Since $A(t)\to \infty$ as $t\to\infty$, it is reasonable 
that\footnote{See below for the proof.}
\begin{equation}\label{eq:132}
  \begin{split}
  \pi(n) &= \lim_{t\to\infty} \frac1{A(t)}\sum_{k=1}^{A(t)} \1{L(A_k-) = n} \\
  &= \lim_{t\to\infty} \frac1{A(t)}\sum_{k=1}^\infty \1{A_k \leq t, L(A_k-) = n} \\
  &= \lim_{t\to\infty} \frac{A(n,t)}{A(t)},
  \end{split}
\end{equation}
where we use~\eqref{eq:19} in the last row. But, 
\begin{equation}\label{eq:1333}
 \frac{A(n,t)}{t} 
=   \frac{A(n,t)}{A(t)} \frac{A(t)}t 
= \frac{A(t)}t \frac{A(n,t)}{A(t)}
\to \lambda  \pi(n), \quad\text{as } t \to \infty, 
\end{equation}
where we use Eq.~\eqref{eq:3} (i.e., $A(t)/t \to \lambda$). that
Next, by  Eq.~\eqref{eq:21}, 
\begin{equation*}
\frac{A(n,t)}t = \frac{A(n,t)}{Y(n,t)}\frac{Y(n,t)}t \to \lambda(n) p(n), \quad\text{as } t \to \infty.
\end{equation*}
Thus
\begin{equation}\label{eq:13}
\lambda  \pi(n) = \lambda(n) p(n), 
\end{equation}
from which  follows our final result:
\begin{equation*}
  \lambda(n) = \lambda \iff \pi(n) = p(n), 
\end{equation*}

So, why is this useful? Well, in words, it means that if the arrival
rate does not depend on the state of the system, i.e.,
$\lambda=\lambda(n)$, the sample average $\pi(n)$ is equal to the
time-average $p(n)$, i.e., $\pi(n)=p(n)$. But, when $\pi(n)=p(n)$, the
customer perception at arrival moments, i.e., $\pi(n)$, is the same as
the server perception, i.e., $p(n)$.

Of course, this property is not satisfied for any general queueing
system, see the exercise below in which $X_k=1$ hour for all $k$ and
$S_k=59$ minutes.  However, it is true when the arrival process is
Poisson. This fact is typically called \recall{PASTA}: Poisson
Arrivals See Time Averages.
% Interestingly, the result can be generalized to general sets
% $B$ of the state space, not just the number $n$ of jobs in the system.

Thus, when customers arrive in accordance to a Poisson process (so
that the inter-arrival times are exponentially distributed), it must
be that $\pi(n) = p(n)$, hence, for the $M/M/1$ queue, we see that
\begin{equation*}
  \pi(n) = p(n) = (1-\rho)\rho^n.
\end{equation*}


With the above reasoning, we can also establish a relation between
$\pi(n)$ and the statistics of the system as obtained by the
departures, i.e., $\delta(n)$, to be defined presently. For this we
turn again to Eq.~\eqref{eq:15}, i.e., $|A(n,t) - D(n,t)| \leq 1$. To
obtain Eq.~\eqref{eq:12} we divided both sides of this equation by the
time the system spends in a certain state. We can also use another
form:
\begin{equation*}
\frac{A(t)}t \frac{A(n,t)}{A(t)} = \frac{A(n,t)}t \approx \frac{D(n,t)}t 
= \frac{D(t)}t \frac{D(n,t)}{D(t)}.
\end{equation*}
Taking limits at the left and right, we see again that the left hand
becomes $\lambda \pi(n)$. For the right hand side, we use
Eq.~\eqref{eq:28} and define, analogous to \eqref{eq:132}, 
\begin{equation}
  \label{eq:33}
  \delta(n) = \lim_{t\to\infty} \frac{D(n,t)}{D(t)}.
\end{equation}
Thus, $\delta(n)$ is the long-run fraction of jobs that leave $n$ jobs
\emph{behind}. Clearly, then, if the limits exist, the right hand side
tends to $\gamma \delta(n)$ as $t\to\infty$. Hence, for (queueing)
systems in which customers arrive and leave as single units, we have
\begin{equation}
  \label{eq:36}
  \lambda \pi(n) = \gamma \delta(n).
\end{equation}
Moreover, if the system is rate-stable, i.e., the output rate $\gamma$ is equal to the input rate $\lambda$, we obtain
\begin{equation}
  \label{eq:39}
\lambda = \gamma \iff  \pi(n) = \delta(n).
\end{equation}
This means that the system as seen by arrivals, i.e., $\pi(n)$, is
the same as what jobs leave behind, i.e., $\delta(n)$.

There is a subtle problem in~\eqref{eq:132} and the derivation
of~\eqref{eq:1333}: $\pi(n)$ is defined as a limit over arrival epochs
while in $A(n,t)/t$ we take the limit over time. Now the observant
reader might ask why these limits should relate at all.  The
resolution lies in the \recall{renewal reward theorem}. As this
theorem is intuitive and very useful in its own right we state this
theorem first, and then apply it to show~\eqref{eq:1333}. For the (simple) proof we refer to \citet{el-taha98:_sampl_path_analy_queuein_system}.

\begin{theorem}[Renewal Reward Theorem, $Y=\lambda X$]
  Suppose the counting process $\{N(t), t\geq 0\}$ is such that
  $N(t)/t\to\lambda$ as $t\to\infty$, where $0<\lambda < \infty$. Let
  $\{Y(t), t\geq 0\}$ be a non-decreasing right-continuous
  (deterministic) process. Define $X_k = Y(T_k)-Y(T_{k-1})$,
  $k\geq 1$, where $T_k$ are the epochs at which $N$ increases, i.e.,
  $N(t) = \{k : T_k \leq t\}$. Then $Y(t)/t$ has a limit iff
  $n^{-1}\sum_k^n X_k$ has a limit in which case $Y=\lambda X$, i.e.,
  \begin{equation*}
  \lim_t \frac{Y(t)}t=Y \iff \lim_n \frac 1n\sum_k^n X_k =X \implies Y=\lambda X.
  \end{equation*}
\end{theorem}


Now, to show that~(\ref{eq:132}) is valid, define
\begin{align*}
  Y(t) &:= A(n,t) = \sum_{k=1}^{A(t)} \1{L(A_k-) = n} \\
X_k &:= Y(A_k) - Y(A_{k-1}) = A(n, A_k) - A(n, A_{k-1}) = \1{L(A_k-)=n}
\end{align*}
Then, 
\begin{equation*}
X= \lim_{m\to\infty} \frac 1 m \sum_{k=1}^m X_k =\lim_{m\to\infty} \frac 1 m \sum_{k=1}^m \1{L(A_k-)=n} = \pi(n).
\end{equation*}
Since $Y=\lim_t Y(t)/t = \lim_t A(n,t)/t$ it follows from the renewal reward theorem that
\begin{equation*}
  Y=\lambda X \implies \lim_t \frac{A(n,t)} t = \lambda X = \lambda \pi(n).
\end{equation*}
Thus, Eq.~\eqref{eq:1333} follows from the renewal reward theorem.

In many (stochastic) systems the relation $Y=\lambda X$ is useful. As
a further example, suppose customers pay on average an amount $X$ and
customers arrive at rate $\lambda$, then the system earns money at
rate $\lambda X = Y$. 

\begin{question} 
Check that the conditions of the renewal reward theorem are satisfied in the above proof of~\eqref{eq:1333}. 
  \begin{solution}
    The counting process here is $\{A(t)\}$ and the epochs at which
    $A(t)$ increases are $\{A_k\}$. By assumption, $A_k\to\infty$,
    hence $A(t)\to\infty$ as $t\to\infty$. Moreover, by assumption
    $A(t)/t \to \lambda$. Also $A(n,t)$ is evidently non-decreasing and
    $A(n,t)\to\infty$ as $t\to\infty$.
  \end{solution}
\end{question}

\begin{question}\label{ex:8}
  To see that $p(n)$ and $\pi(n)$ can be very different, consider a
  queueing system in which the inter-arrival times $\{X_i\}$ are all
  identical to 1 hour, and the service times are all identical to
  $S_i=59$ minutes.
\begin{enumerate}
\item What are $\rho$, $p(0)$, $p(n)$ for $n\geq 2$. What is $\pi(n)$?
  What is the time average queue length $\E L$? What is the queue
  length as observed by customers?
\item Check Eq.~(\ref{eq:13}) for this case.
\item Consider another (theoretical) queueing system in which each
  customer requires precisely 1 minute. At the start of each hour, 59
  customers arrive. What is $p(n)$, what is $\pi(n)$?
\end{enumerate}
\hint{Check the definitions of $Y(0,t)$, $Y(1,t)$, $A(0,t)$ and so on.}
  \begin{solution}
\begin{enumerate}
\item $A(t)=\lfloor t\rfloor$ provided the unit of $t$ is in
  hours. $A(0,t)=A(t)$ and $A(n,t)=0$ for $n>0$.  Thus, 
  \begin{equation*}
    \begin{split}
A(0,t) &\approx t \\
A(1,t) & = 0 \\
    Y(0,t)&= \frac 1{60} \lfloor t \rfloor + (t-\lfloor t \rfloor) 1\{t-\lfloor t \rfloor \in[59/60, 1)\} \approx \frac{1}{60}t, \\
    Y(1,t)&= \frac{59}{60} \lfloor t \rfloor + (t-\lfloor t \rfloor) 1\{t-\lfloor t \rfloor     \in[0,59/60)\} \approx \frac{59}{60}t. \\
\lambda(0) &= \lim_t \frac{A(0,t)}{Y(0,t)} = \lim_t \frac{t}{t/60} = 60, \\
\lambda(1) &= \lim_t \frac{A(1,t)}{Y(1,t)} = \lim_t\frac{0}{59/60 t} = 0 \\
\lambda(n) &= 0, \quad n\geq 1. \\
\lambda &= \lim_t \frac{A(t)}t  = 1 \\
p(0)  &= \frac{1}{60}, \\
p(1)  &= \frac{59}{60}, \\
\rho  &= \frac{59}{60}, \\
\pi(0)  &= 1. \\
    \end{split}
  \end{equation*}
  There is no queue so $\E{L_Q} = 0$, but $\E L = \rho$.  The queue
  length as observed by customers is equal to 0, because jobs only
  arrive when the server is free.
\item We have to check that $\lambda(n) p(n) = \lambda \pi(n)$.  First
  take $n=0$. By the above: $\lambda(0) p(0) = 60\cdot 1/60 = 1$,
  $\lambda \pi(0) = 1\cdot 1 = 1$.  For $n=1$: $\lambda(1) p(1) = 0$. Since $\pi(1)=0$ the result is again OK for $n=1$. And so on .
\item Again $\rho=59/60$.

  Observe that the average queue length is very different from the
  previous example, even though $\rho = 59/60$ in both cases. Thus,
  knowledge of the load is not sufficient to make a statement about the
  average queue length.

  To compute the average queue length, as perceived by the customers,
  note that the first customer sees no queue, the second sees one
  customer in front, and so on. Thus, the average queue length as seen
  by customers is
  \begin{equation*}
   \frac{0+1+2+\cdots+58}{59}=\frac{58\cdot 59}{2\cdot 59} = 29.
  \end{equation*}
The time average queue length is $\E L = 59/60 + 58/60 + \cdots 1/60 + 0/60 = 59/2$. 
\end{enumerate}

These two examples show that the way customers arrive have an enormous
impact on the queue length. Both queueing systems have the same
utilization, i.e. 59/60, but the average queue lengths are not nearly
the same, i.e. 0 versus 29.
  \end{solution}
\end{question}

\begin{question}
  Is in general $\pi(n)\geq \delta(n)$?  \hint{Use that
    $\lambda \geq \gamma$ always, and $\lambda=\gamma$ when the system
    is rate-stable.}
  \begin{solution}
    We have shown for one-step transition processes that
    $\lambda \pi(n) = \gamma \delta(n)$. Thus,
    $\pi(n) = \gamma/\lambda \delta(n)$. Since $\lambda\geq \gamma$,
    we have that $\pi(n) \geq \delta(n)$.
  \end{solution}
\end{question}

\begin{question}
  Why is $\mu(n) = \mu$ for the $M/M/1$ queue?  \hint{Think about the
    construction of the $M/M/1$ queue as a random walk, see
    Section~\ref{sec:queu-proc-as}.}

 \begin{solution}
   The $M/M/1$-queue can be constructed as a reflection of the random
   walk $Z(t) = Z(0) + N_\lambda(t) - N_\mu(t)$. Clearly,
   downcrossings can only occur when $N_\mu$ fires. The rate at which
   the transitions of $N_\mu$ occur is constant, and, in particular,
   independent of the history of $Z$. 

   More specifically, for the interested, define
   $\sigma\{X(t) : t\in I\}$ as the $\sigma$-algebra generated by the
   stochastic procsses $\{X(t), t\in I\}$ on the index set $I$. Then,
   by construction of $\{N_\lambda(t)\}$ and $\{Z(t)\}$, we have that
   $\sigma\{Z(s) : s\in[0,t]\}$  and $\sigma\{N_\lambda(u) : u > t\}$ are independent.
 \end{solution}
\end{question}


\begin{question}
Show that 
\begin{equation*}
\lambda  \pi(n) = \lambda(n) p(n) = \mu(n+1) p(n+1) = \mu \delta(n).
\end{equation*}
What is the important condition for this to be true?
\hint{Check all definitions of $Y(n,t)/t$ and so on.}
\begin{solution}
  The important condition is that transitions occur as single
  steps. In other words, the relation is true for processes with
  \recall{one-step} transitions, i.e, when $|A(n,t) - D(n,t)|\leq 1$.
  In  that case, 
\begin{align*}
  \frac{A(n,t)}{t} &=   \frac{A(n,t)}{A(t)} \frac{A(t)}{t} \to \pi(n) \lambda\\
  \frac{A(n,t)}{t} &=   \frac{A(n,t)}{Y(n,t)} \frac{Y(n,t)}{t} \to \lambda(n)p(n)\\
  \frac{D(n,t)}{t} &=   \frac{D(n,t)}{Y(n+1,t)} \frac{Y(n+1,t)}{t} \to \mu(n+1)p(n+1)\\
  \frac{D(n,t)}{t} &=   \frac{D(n,t)}{D(t)} \frac{D(t)}{t} \to \delta(n)\mu\\
\end{align*}
\end{solution}
\end{question}




%%% Local Variables:
%%% mode: latex
%%% TeX-master: "book"
%%% End:
