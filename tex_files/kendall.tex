
\section{Kendall's Notation to Characterize Queueing Processes}
\label{sec:kendalls-notation}

\subsection*{Theory and Exercises}

As will become apparent in Sections~\ref{sec:constr-discr-time}
and~\ref{sec:constr-gg1-queu}, the construction of any single-station queueing
process involves three main elements: 
the distribution of the
interarrival times between consecutive jobs, the distribution of the
service times of the individual jobs, and the number of servers
present to process jobs. In this characterization it is implicit that
the interarrival times form a set of i.i.d. (independent and
identically distributed) random variables, the service times are also
i.i.d., and finally, the interarrival times and service times are
mutually independent.

To characterize the type of queueing process it is common to use 
Kendall's abbreviation $A/B/c/K$, where $A$ is the distribution of the
interarrival times, $B$ the distribution of the service times, $c$ the
number of servers, and $K$ system size, i.e., the total number of customers that can be simultaneously present, whether in queue or in service. In this notation it
is assumed that jobs are served in first-in-first-out (FIFO) order;
FIFO scheduling is also often called first-come-first-serve (FCFS).

Let us illustrate the shorthand $A/B/c/K$ with some examples:
\begin{itemize}
\item $M/M/1$: the distribution of the interarrival times is
  \emph{M}emory-less, hence exponential, the service times are also
  \emph{M}emoryless, and there is 1 server. As $K$ is unspecified, it
  is assumed to be infinite.
\item $M/M/c$: A \recall{multi-server} queue with $c$ servers in which
  all servers have the same service capacity. Jobs arrive according to a
  Poisson process and have exponentially distributed processing times.
\item $M(n)/M(n)/1$: the interarrival times are exponential, just as
  the service times, but the rates of the arrival and service processes
  may depend on the queue length $n$. 
\item $M/M/c/K$: interarrival times and process times are exponential,
  and the \recall{system capacity} is $K$ jobs. Thus, the queue can
  contain at most $K-c$ jobs.\footnote{Sometimes, the $K$ stands for
    the capacity in the queue, not the entire system. The notation differs among
    authors.}
\item $M/M/c/c$: in this system the number of servers is the same as
  the system capacity, thus the queue length is always zero. This
  queueing system is useful to determine the number of beds
  in a hospital; the beds act as servers.
\item $M^X/M/1$: Customers arrive with exponentially distributed
  interarrival times. However, each customer brings in a number of
  jobs, known as a batch. The number of jobs in each batch is
  distributed as the random variable $X$. Thus, the arrival process of
  work is \recall{compound Poisson}.
\item $M/G/1$: the interarrival times are exponentially distributed,
  the service times can have any \emph{G}eneral distribution (with
  finite mean), and there is 1 server.
\item $M/G/\infty$: exponential interarrival times, service times can
  have any distribution, and there is an unlimited supply of
  servers. This is also known as an \recall{ample} server. Observe
  that in this queueing process, jobs actually never have to wait in
  queue; upon arrival there is always a free server available.
\item $M/D/1-LIFO$.  Now job service times are \emph{D}eterministic, and the service sequence is last-in-first-out (LIFO).
\item $G/G/1$: generally distributed interarrival and service times, 1 server.
\end{itemize}

In the sequel we will use Kendall's notation to distinguish between the different queueing models. Ensure that you familiarize yourself with this notation.

\Opensolutionfile{hint}
\Opensolutionfile{ans}

\begin{exercise}
  Is the $M/D/1$ queue a specific type of  $M/G/c$ queue? 
  \begin{solution}
    Yes, take $G=D$ and $c=1$. 
  \end{solution}
\end{exercise}

\begin{exercise}
What is the $D/D/1$ queue?  
\begin{solution}
  A queueing process with deterministic interarrival times and deterministic service times and 1 server.
\end{solution}
\end{exercise}

\begin{exercise}
  By how many parameters is the $M/M/1$ queue characterized?
  \begin{solution}
    The interarrival times are exponentially distributed with rate $\lambda$; the service times are also exponential, but with parameter $\mu$. Thus, if we know $\lambda$ and $\mu$, we have fully characterized the parameters of both distributions. Since the number of servers is 1, only $\lambda$ and $\mu$ remain.
  \end{solution}
\end{exercise}


\begin{exercise}
  What are some advantages and disadvantages of using the Shortest
  Processing Time First (SPTF) rule to serve jobs? 
  \begin{hint}
Look up the relevant
  definitions on wikipedia or
  \citet{hall91:_queuein_method_servic_manuf}
  \end{hint}
  \begin{solution}
  \begin{itemize}
  \item Advantage: SPTF minimizes the number of jobs in queue. Thus,
    if you want to keep the shop floor free of jobs (Work In Progress,
    WIP), then this is certainly a good rule. 
  \item Disadvantage: large jobs get near to terrible waiting times,
    and the variance of the waiting time increases. Thus, the $C_s^2$
    is larger than under FIFO. Also, SPTF does not take duedates into
    account, thus giving a reliable duedate quotation to a customer is
    hard (near to impossible.)
  \end{itemize}
  \end{solution}
\end{exercise}


\Closesolutionfile{hint}
\Closesolutionfile{ans}
\subsection*{Hints}
\input{hint}
\subsection*{Solutions}
\input{ans}
\clearpage

%%% Local Variables:
%%% mode: latex
%%% TeX-master: "book"
%%% End:
