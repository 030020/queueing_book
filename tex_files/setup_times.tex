\section{Setups and Batch Processing}


\subsection*{Theory and Exercises}

\Opensolutionfile{hint}
\Opensolutionfile{ans}

Setups are often required when a machine has to undergo some changeovers when changing from process one type of job to  another type. For instance, a painting machine requires a clean up when changing color, or an oven has to warm up or cool down when items require a different temperature. In service settings, when people have to move from part of a building to another, like the customs at Schiphol, the time they spend walking from one place to another they  cannot serve customers.  In this section we develop a simple model to incorporate the effects  of setups on the averate time jobs spend in the system.

Setups differ from random interruptions in that they occur, often, between any $B$ jobs, and service starts once $B$ jobs have been assembled into a batch. As we will see, this has considerable consequences on the system times. 

Suppose that we produce in batch sizes of $B$ jobs and the setup time is a random variable $T$ independent of the normal job service times. Let the average normal (or net) service time of a job be $\E{S_0}$. 
\begin{exercise}
  Show that the average \recall{effective} processing time $S_e$, i.e., the average time the server is occupied with processing a job including setups, is 
  \begin{equation*}
    \E{S_e} = \E{S_0} + \frac{\E{T}} B.
  \end{equation*}
  \begin{hint}
    What fraction of the setup time $\E T$ `belongs' to one job?
  \end{hint}
  \begin{solution}
    The total service time spent on a batch of size $B$ is $B \E{S_0} + \E T$. The expected effective time per job is then the average time per job, i.e.,  $(B \E{S_0}+\E T)/B$. 
  \end{solution}
\end{exercise}

Observing that
\begin{equation*}
  \lambda_B = \frac \lambda B,
\end{equation*}
the load is given by 
\begin{equation*}
\rho = \lambda_B (B \E{S_0} + \E T) = \lambda (\E{S_0} + \E{T}/B).
\end{equation*}
The first equality has the interpretation of the batch arrival rate times the work per batch, while the second means the job arrival rate times the effective work per job. 

\begin{exercise} 
Show that  the minimal batch size~$B$ must satisfy the constraint
  \begin{equation*}
 B>\frac{\lambda \E T}{1-\lambda \E{S_0}}.
  \end{equation*}

\begin{solution}
This follows right away from requiring that $\rho < 1$.
\end{solution}
\end{exercise}

\begin{exercise}
Show that the effective variance of a job is  
\begin{equation*}
  \V{S_e} = \V{S_0} + \V{T}/B.
\end{equation*}
\begin{hint}
  What is the variance of a batch?
\end{hint}
\begin{solution}
  The variance of a batch is $\V{\sum_{i=1}^B S_{0,i} + T} = B\V{S_0} + \V T$, since the normal service times $S_{0,i}, i=1,\ldots, B$ of the job are independent, and also independent of the setup time $T$  of the batch.
\end{solution}
\end{exercise}

With the above, we can characterize the SCV $C_e^2$ of the service times. We also need the SCV $C_a^2$ of the interarrival times of the batches; recall, jobs are first assembled into batches, and these batches are sent to the server for processing.

\begin{exercise}
  Why is $C_{a,e}^2$ of the batch equal to $C_{a}^2/B$, with $C_a^2$ the SCV of interarrival times of individual jobs?
  \begin{solution}
The variance of the interarrival time of batches is $B$ times the variance of job interarrival times. The interarrival times of batches is also $B$ times the interarrival times of jobs. Thus, 
\begin{equation*}
  C_{a,e}^2 = \frac{B \V{X}}{(B \E X)^2} = \frac{\V X}{(\E X)^2} \frac 1 B =  \frac{C_a^2}{B}.
\end{equation*}
  \end{solution}
\end{exercise}

\begin{exercise}
  Jobs arrive at $\lambda=3$ per hour at a machine with $C_a^2=1$; service times are exponential with average 15 minutes.  Between any two batches, the machine requires a cleanup of 4 hours, with a standard deviation of $1$ hour, during which it is unavailable for service.  Suppose the batch size $B=60$ jobs. What is the minimal batch size?  What is the average time, batches spend in queue? 
  \begin{solution}
First check the load.
\begin{pyconsole}
labda = 3 # per hour
ES0 = 15./60 # hour
ES0
ET = 4.
B = 60
ESe = ES0+ ET/B
ESe

rho = labda*ESe
rho
\end{pyconsole}

The minimal batch size is
\begin{pyconsole}
Bmin = labda*ET/(1-labda*ES0)
Bmin
\end{pyconsole}

Now the time in queue
\begin{pyconsole}
Cae = 1.
CaB = Cae/B
Ce = 1 # SCV of service times
VS0 = Ce*ES0*ES0
VT = 1*1. # Var is sigma squared
VSe = B*VS0 + VT
CeB = VSe/ESe/ESe
CeB
ESb = B*ES0 + ET
ESb
EWq = (CaB+CeB)/2 * rho/(1-rho) * ESb
EWq
\end{pyconsole}

This is terrible.  The batch size is way too small, because $\rho$ is very high.
  \end{solution}
\end{exercise}

Now that we can compute the average time a batch spends in queue, we can also compute the total time a job spends in the system. When a job arrives, it first needs to wait until a batch is formed. Then it enters (as part of a batch) a queue of batches. And, finally,  the batch is processed. 

First we consider the time it takes to form a batch. 
\begin{exercise}
  Show that the total time to form a batch is $(B-1)/\lambda$. Hence, the average time a job spends 
waiting until the batch is complete is
\begin{equation*}
  \E{W_f} = \frac{B-1}{2\lambda}.
\end{equation*}
\begin{solution}
  Suppose a batch is just finished. The first job of a new batch needs to wait, on average, $B-1$  interarrival times until the batch is complete, the second $B-2$ interarrival times, and so on. The last job does not have to wait at all. Thus, the total time to form a batch is $(B-1)/\lambda$. 

An arbitrary job can be anywhere in the batch, hence the average time of a job until the batch is complete is half the total time. 
\end{solution}
\end{exercise}

Finally, when the batch is taken into service, there can be various rules to determine when the job's service finished. If the job has to wait until all jobs in the batch are served, the time a job spends at the server is $B \E{S_0} + \E T$. If, however, it can leave right after being served, the expected time is $B \E{S_0} + \E T$. Yet other rules apply when the job has to wait for transport. Then it needs to wait until a transport batch has formed, and the size of the transport batch may be larger or smaller than $B$. 

Observe now that the times to form and process batches are linear functions of the batch size $B$. 

Clearly, batch sizes need to be tuned to minimize the time jobs spend in the system. When the batchsizes are small, the load $\rho$ is near to one (in other words, the server spends a relatively large fraction of its time on setups), so that the queueing times are long, but the times to form a batch are small. If, however, the batch sizes are large, the queueing times will be relatively short, but the times to form and process batches will be large. 



\Closesolutionfile{hint}
\Closesolutionfile{ans}
\subsection*{Hints}
\input{hint}
\subsection*{Solutions}
\input{ans}
\clearpage

%%% Local Variables:
%%% mode: latex
%%% TeX-master: "booktest"
%%% End:
