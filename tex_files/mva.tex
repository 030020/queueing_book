
\section{MVA Algorithm}
\label{sec:mva}

\subsection*{Theory and Exercises}

\Opensolutionfile{hint}
\Opensolutionfile{ans}

\begin{comment}
\begin{exercise}[use=false]
  Assume that there are $M$ stations, each with one server. The
  service time at station $i$, $i=0,\ldots,M$ is deterministic and
  equal to $1/\mu_i$. Suppose that the routing is such that station
  $i$ sends all its output to station $i+1$, and station $M$ sends the
  jobs (cards) to station 0. Thus, the stations in this network are in
  tandem. Sketch the cycle time (also called lead time or sojourn
  time), throughput and number of jobs in queue as a function of the total number of jobs $N$ in the network.
  \begin{solution}
    TBD. See Factory Physics.
  \end{solution}
\end{exercise}
\end{comment}

\begin{exercise}\label{ex:mva_numerical}
  Consider two stations in tandem, stations 0 and 1. The service times
  are $\E{ S_0} = 2=1/\mu_0$ and $\E{ S_1} = 3=1/\mu_1$ hours. The routing matrix is
  \begin{equation*}
    P=
    \begin{pmatrix}
      0 & 1\\
1/2 & 1/2\\
    \end{pmatrix}.
  \end{equation*}
Apply the MVA algorithm to this case.
\begin{hint}
Start with $n=1$, then consider $n=2$ and so on.
\end{hint}
\begin{solution}
  From $V P = V$ we conclude that $V_1=2V_0$.  Take $n=1$.
  \begin{align*}
\E{ W_0(1)} &= \E{ L_0(0)+1}\E{ S_0} = \E{ S_0 }= 2,\\
 \E{ W_1(1)} &= \E{ L_1(0)+1}\E{ S_1} = \E{ S_1} = 3,\\
 \E{ W(1)} &= \sum_{i=0}^1 V_i \E{ W_i(1)} = V_0 2 + V_1  3 = 1\cdot 2 + 2 \cdot 3 = 8\\
 \TH_0(1) &= \frac{1}{\E{ W(1)}} = \frac18, \\
 \E{ L_0(1)} &= \TH_0(1) \E{ W_0(1)} = \frac28=\frac14 \\
 \E{ L_1(1)} &= \TH_1(1) \E{ W_1(1)} = V_1 \TH_0(1) \E{ W_1(1)}= 2 \cdot \frac18 \cdot 3 = \frac34.\\
  \end{align*}
Now take $n=2$.     
  \begin{align*}
\E{ W_0(2) }&= (\E{ L_0(1)+1}\E{ S_0}= (1/4+1)\E{ S_0} = \frac54\cdot2=\frac52,\\
\E{ W_1(2) }&= (\E{ L_1(1)+1}\E{ S_1} = (3/4+1)\E{ S_1} = \frac74\cdot3=\frac{21}4,\\
\E{ W(2) }&= \sum_{i=0}^1 V_i \E{ W_i(2)} = \frac52 + 2 \cdot \frac{21}4=13\\
\TH_0(2) &= \frac{2}{\E{ W(2)}} = \frac2{13}, \\
\TH_1(2) &= V_1 \TH_0(2) = \frac4{13}, \\
\E{ L_0(2)} &= \TH_0(2) \E{ W_0(2)} = \frac2{13}\cdot \frac52=\frac5{13} \\
\E{ L_1(2)} &= \TH_1(2) \E{ W_1(2)} = \frac{4}{13} \cdot \frac{21}{4} = \frac{21}{13}.\\
  \end{align*}
And so on.
\end{solution}
\end{exercise}


\begin{exercise}
Zijm.Ex.3.1.13. Assume that all stations have just one server.
\begin{solution}
  Since there are quite a lot of jobs, and station 0 is the
  bottleneck, i.e., the station with the highest load, most of the
  time there will be a queue at station 0. In fact, most of the jobs
  will be in queue in front of station 0. Therefore the rate out of
  station 0 will be approximately equal to its service rate, i.e.,
  $\mu_0$. Thus, station 1 will receive jobs at a rate
  $\lambda_1\approx\mu_0$, i.e., $\lambda_1 \approx 2$. Now I simply
  approximate the queueing process at station 1 as an $M/M/1$ queue
  with $\mu_1=2.5$. Hence, $\rho_1 = \lambda_1/\mu_1=2/2.5 = 4/5$,
  hence $L_1 = \rho_1/(1-\rho_1) = (4/5)/(1/5) = 4$.
\end{solution}
\end{exercise}



\begin{exercise}
Zijm.Ex.3.1.14
\begin{solution}
  I use the MVA algorithm to compute the expected number of jobs at
  each of the stations. Again, as in an earlier question in which I
  implemented the convolution algorithm, nearly as a one-liner,
  studying the code in detail is very rewarding. I include the
  numerical results at the end.


 Start with computing the visit ratios.

\begin{pyconsole}
from functools import lru_cache
import numpy as np
from numpy.linalg import eig

mu = np.array([2, 2.5, 3, 3, 3])
c = np.array([1, 1, 1, 1, 1])  # single server stations
P = np.matrix([
    [0, 1, 0, 0, 0],
    [0, 0, 1, 0, 0],
    [0, 0, 0, 1, 0],
    [0, 0, 0, 0, 1],
    [1, 0, 0, 0, 0],
]
)

v, w = eig(P.T)
x = v.real.argmax()
V = w[:, x].real
V = V / V[0]
print(V)
  
\end{pyconsole}
The visit ratios are according to expectation.


The following implements the MVA algorithm. Note how easy the code becomes with
memoization; I only have to specify the recursions and the `boundary
conditions', i.e., what to do when $n=0$. For the rest I don't have to
think about in what exact sequence each of the functions needs to be
called. The memoization takes care of all these problems.

\begin{pyconsole}
@lru_cache(maxsize=None)
def EL(j, n):
    if n <= 0:
        return 0
    else:
        return TH(j, n) * EW(j, n)


@lru_cache(maxsize=None)
def EW(j, n):
    return (EL(j, n - 1) + 1) / mu[j]


@lru_cache(maxsize=None)
def TH(j, n):
    if j == 0:
        return n / sum(V[j] * EW(j, n) for j in range(len(mu)))
    else:
        return V[j] * TH(0, n)
  
\end{pyconsole}
This is all!


Now I print the expected jobs at the stations. 

\begin{pyconsole}
print("  n   EL0   EL1   EL2  EL3   EL4   TH0")
for n in range(1,30):
    res = "{:3d}".format(n)
    for j in range(len(mu)):
        res += "{:6.2f}".format(float(EL(j, n)))
    res += "{:6.2f}".format(float(TH(0,n)))
    print(res)
  
\end{pyconsole}

This is interesting. Using the insights of the previous question,
approximation stations 3, 4 and 5 as $M/M/1$ queues, we have according
to this model that $\E{ L_3} = \rho_3/(1-\rho_3)$. Taking
$\lambda_3 = 2$ and $\mu_3 = 3$, we see that $\rho_3 = 2/3$ so that
$\E{ L_3} = 2$.  This is very close to $1.98$. 

For the last line, with $n=29$, the number of jobs at station 0 must
therefore be approximately $29-4-2-2-2 = 19$. The algorithm supports
our intuition!
\end{solution}
\end{exercise}

\begin{exercise}
  Implement the MVA algorithm in your preferred computer language and
  make Figure 3.2.
  \begin{solution}
    See the answer of the previous question.
  \end{solution}
\end{exercise}





\Closesolutionfile{hint}
\Closesolutionfile{ans}

\opt{solutionfiles}{
\subsection*{Hints}
\input{hint}
\subsection*{Solutions}
\input{ans}
}

%\clearpage
