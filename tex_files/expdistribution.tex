\section{Exponential Distribution}
\label{sec:expon-distr}
As we will see in the sections to come, the modeling and analysis of
any queueing system involves the specification of the (probability)
distribution of the time between consecutive arrival epochs of jobs,
or the specification of the distribution of the number of jobs that
arrive in a certain interval.  For the first case, the most common
distribution is the exponential distribution, while for the second it
is the Poisson distribution.  For these reasons we start our
discussion of the analysis of queueing system with these two
exceedingly important distributions. In the ensuing sections we will
use these distributions time and again.

As mentioned, one of the most useful models for the inter-arrival
times of jobs assumes that the sequence $\{X_i\}$ of inter-arrival
times is a set of \recall{independent and identically distributed
  (i.i.d.)}  random variables.  Let us write $X$ for the generic
random time between two successive arrivals. For many queueing
systems, measurements of the inter-arrival times between consecutive
arrivals show that it is reasonable to model an inter-arrival $X$ as
an \recall{exponentially distributed} random variable, i.e.,
\begin{equation*}
  \P{X \leq t} = 1- e^{-\lambda t} := G(t)
\end{equation*}
The constant $\lambda$ is often called the \recall{rate}. The reason
behind this will be clarified once we relate the exponential
distribution to the Poisson process in
Section~\ref{sec:poisson-distribution}.  In the sequel we often write
$X\sim \exp(\lambda)$ to mean that $X$ is exponentially distributed
with rate $\lambda$.


Let us show with simulation how the exponential distribution
originates. Consider $N$ people that regularly visit a shop. We assume
that we can characterize the interarrival times
$\{X_k^i, k=1,2, \ldots\}$ of customer $i$ by some distribution
function, for instance the uniform distribution. Then, with
$A_{0}^i=0$ for all~$i$,
\begin{equation*}
A_k^i = A_{k-1}^i + X_k^i = \sum_{j=1}^n X_j^i,
\end{equation*}
is the arrival moment of the $k$th visit of customer $i$.  Now the
shop owner `sees' the superposition of the arrivals of all
customers. One way to compute the arrival moments of all customers is
to put all the numbers $\{A_k^i, k=1,\ldots,n, i=1,\ldots,N\}$ into
one set, and sort these numbers in increasing order. This results in
the (sorted) set of arrivals $\{A_k, k=1,2,\ldots\}$ at the shop, and
then
\begin{equation*}
X_k = A_k - A_{k-1},
\end{equation*}
with $A_0=0$, must be the inter-arrival time between the $k-1$th and
$k$th visit to the shop.  Thus, starting from interarrival times of
individual customers we can construct interarrival times as seen by
the shop.

To plot the \recall{empirical distribution function}, or the
histogram, of the interarrival times at the shop, we need to count the
number of interarrivals smaller than time $t$ for any $t$.  For this,
we introduce the \recall{indicator function}:
\begin{equation*}
  \1{A} =
  \begin{cases}
    1, \quad\text{if the event $A$ is true},\\
    0, \quad\text{if the event $A$ is false}.
  \end{cases}
\end{equation*}
With the indicator function we  define the counting function for a
simulation with a total of $n\cdot N$ visits as
\begin{equation*}
  \PP{nN}{X \leq t} = \frac1{nN}\sum_{i=1}^{nN} \1{X_i\leq t},
\end{equation*}
where $\1{X_i\leq t}=1$ if $X_i\leq t$ and $\1{X_i\leq t}=0$
if $X_i> t$.  

We now compare the probability density of $\mathbb{P}_n$ to the
density of the exponential distribution, i.e., to
$\lambda e^{-\lambda t}$ several simulation scenarios.  As a first
example, take $N=1$ and let the computer generate $n=100$ uniformly
distributed numbers on the set $[4, 6]$.  Thus, the time between two
visits of an individual customer is somewhere between $4$ and $6$
hours. In a second simulation we take $N=3$, and in the third,
$N=10$. The emperical distributions are shown, from left to right, in
the three panels in Figure~\ref{fig:uniformfew}. The continuous curve
is the graph of $\lambda e^{-\lambda x}$ where $\lambda = N/5$.
Recall from Eq.~(\ref{eq:54}) that when $N$ persons visits the shop,
each with an average interarrival time of $5$ hours, it must be that
the arrival rate is $N/5$.  As a second example, we extend the
simulation to $n=1000$ visits to the shop, see
Figure~\ref{fig:uniformmany} In the third example we take the
inter-arrival times to be normally distributed times with mean $5$ and
$\sigma=1$, see Figure~\ref{fig:normal}.

\begin{figure}[ht]
  \centering
  \begin{tabular}[h]{c}
% see progs/convergence_to_exp.py
 % This file was created by matplotlib2tikz v0.5.15.
\begin{tikzpicture}

\begin{groupplot}[group style={group size=3 by 1}]
\nextgroupplot[
title={$N=1, n=100, b=10$},
xmin=0, xmax=8,
ymin=0, ymax=2,
width=5cm,
height=5cm,
tick align=outside,
xmajorgrids,
x grid style={white},
ymajorgrids,
y grid style={white},
axis line style={white},
axis background/.style={fill=white!89.803921568627459!black},
legend entries={{simulation},{theory}},
legend cell align={left},
legend style={draw=white!80.0!black, fill=white!89.803921568627459!black}
]
\addplot [semithick, black, mark=*, mark size=1, mark options={solid}, only marks]
table {%
4.14316511876391 0.477456921988451
4.33356782678306 0.530507691098278
4.52397053480221 0.74271076753759
4.71437324282137 0.424406152878623
4.90477595084052 0.636609229317934
5.09517865885968 0.583558460208106
5.28558136687883 0.583558460208106
5.47598407489798 0.583558460208106
5.66638678291714 0.212203076439311
5.85678949093629 0.477456921988451
};
\addplot [semithick, black]
table {%
0 0.201666243229445
0.1 0.197640049681036
0.2 0.193694237629449
0.3 0.189827202287196
0.4 0.18603737090579
0.5 0.182323202136099
0.6 0.178683185401466
0.7 0.175115840283355
0.8 0.171619715919244
0.9 0.168193390412562
1 0.164835470254385
1.1 0.16154458975669
1.2 0.158319410496923
1.3 0.15515862077365
1.4 0.152060935073083
1.5 0.149025093546251
1.6 0.14604986149661
1.7 0.143134028877887
1.8 0.140276409801946
1.9 0.137475842056476
2 0.134731186632315
2.1 0.132041327260207
2.2 0.129405169956808
2.3 0.126821642579754
2.4 0.124289694391617
2.5 0.12180829563256
2.6 0.119376437101529
2.7 0.116993129745803
2.8 0.114657404258743
2.9 0.112368310685563
3 0.110124918036983
3.1 0.107926313910584
3.2 0.105771604119734
3.3 0.103659912329913
3.4 0.101590379702301
3.5 0.0995621645444846
3.6 0.0975744419681357
3.7 0.0956264035535224
3.8 0.0937172570207203
3.9 0.0918462259073866
4 0.0900125492529686
4.1 0.0882154812892152
4.2 0.0864542911368687
4.3 0.084728262508411
4.4 0.0830366934167454
4.5 0.0813788958896931
4.6 0.0797541956901909
4.7 0.0781619320420749
4.8 0.0766014573613375
4.9 0.0750721369927518
5 0.0735733489517529
5.1 0.0721044836714721
5.2 0.0706649437548231
5.3 0.0692541437315359
5.4 0.0678715098200426
5.5 0.0665164796941166
5.6 0.0651885022541708
5.7 0.063887037403122
5.8 0.0626115558267297
5.9 0.0613615387783206
6 0.060136477867811
6.1 0.0589358748549412
6.2 0.0577592414466378
6.3 0.0566060990984221
6.4 0.0554759788197825
6.5 0.0543684209834333
6.6 0.0532829751383812
6.7 0.0522191998267238
6.8 0.051176662404106
6.9 0.0501549388637608
7 0.0491536136640626
7.1 0.048172279559524
7.2 0.0472105374351664
7.3 0.0462679961441972
7.4 0.0453442723489281
7.5 0.0444389903648687
7.6 0.0435517820079337
7.7 0.0426822864446998
7.8 0.0418301500456523
7.9 0.0409950262413616
};
\nextgroupplot[
title={$N=3, n=100, b=10$},
xmin=0, xmax=8,
ymin=0, ymax=2,
width=5cm,
height=5cm,
tick align=outside,
xmajorgrids,
x grid style={white},
ymajorgrids,
y grid style={white},
axis line style={white},
axis background/.style={fill=white!89.803921568627459!black},
legend entries={{simulation},{theory}},
legend cell align={left},
legend style={draw=white!80.0!black, fill=white!89.803921568627459!black}
]
\addplot [semithick, black, mark=*, mark size=1, mark options={solid}, only marks]
table {%
0.270772192547602 0.495388011520489
0.804119872907879 0.344890387767429
1.33746755326816 0.232017169952634
1.87081523362843 0.169309826722192
2.40416291398871 0.144226889430016
2.93751059434898 0.125414686460883
3.47085827470926 0.181851295368281
4.00420595506954 0.125414686460883
4.53755363542981 0.0438951402613091
5.07090131579009 0.0125414686460883
};
\addplot [semithick, black]
table {%
0 0.601992175105371
0.1 0.566821947903445
0.2 0.533706473126197
0.3 0.502525705841803
0.4 0.473166614511137
0.5 0.445522771243889
0.6 0.419493965993162
0.7 0.394985843290015
0.8 0.371909560201082
0.9 0.350181464269352
1 0.32972279027064
1.1 0.310459374686473
1.2 0.292321386858342
1.3 0.275243075848751
1.4 0.259162532091412
1.5 0.244021462966575
1.6 0.229764981487927
1.7 0.216341407335053
1.8 0.203702079510191
1.9 0.191801179940153
2 0.180595567383956
2.1 0.170044621044093
2.2 0.160110093314493
2.3 0.150755971131416
2.4 0.141948345424635
2.5 0.133655288195697
2.6 0.125846736777634
2.7 0.118494384856589
2.8 0.111571579860279
2.9 0.105053226341354
3 0.0989156950053789
3.1 0.0931367370536968
3.2 0.0876954035306303
3.3 0.0825719693826748
3.4 0.0777478619543833
3.5 0.0732055936617415
3.6 0.0689286985989698
3.7 0.0649016728489515
3.8 0.0611099182809073
3.9 0.0575396896315823
4 0.0541780446781129
4.1 0.0510127973219461
4.2 0.0480324734137415
4.3 0.0452262691591164
4.4 0.0425840119544556
4.5 0.0400961235108133
4.6 0.0377535851322289
4.7 0.0355479050225908
4.8 0.0334710875025322
4.9 0.0315156040247718
5 0.0296743658828256
5.1 0.0279406985141602
5.2 0.0263083173046345
5.3 0.0247713048065195
5.4 0.0233240892875122
5.5 0.0219614245329808
5.6 0.0206783708282251
5.7 0.0194702770518117
5.8 0.018332763815071
5.9 0.0172617075866368
6 0.0162532257444784
6.1 0.0153036625012395
6.2 0.0144095756518614
6.3 0.0135677240954511
6.4 0.0127750560861592
6.5 0.0120286981704788
6.6 0.0113259447708601
6.7 0.0106642483778831
6.8 0.0100412103154327
6.9 0.00945457204540162
7 0.008902206980398
7.1 0.00838211277478084
7.2 0.0078924040660761
7.3 0.00743130564046165
7.4 0.00699714599754563
7.5 0.00658835129111004
7.6 0.00620343962385475
7.7 0.00584101567546005
7.8 0.00549976564449414
7.9 0.00517845248583
};
\nextgroupplot[
title={$N=10, n=100, b=10$},
xmin=0, xmax=8,
ymin=0, ymax=2,
width=5cm,
height=5cm,
tick align=outside,
xmajorgrids,
x grid style={white},
ymajorgrids,
y grid style={white},
axis line style={white},
axis background/.style={fill=white!89.803921568627459!black},
legend entries={{simulation},{theory}},
legend cell align={left},
legend style={draw=white!80.0!black, fill=white!89.803921568627459!black}
]
\addplot [semithick, black, mark=*, mark size=1, mark options={solid}, only marks]
table {%
0.271945769998786 1.17566289181205
0.815161553270047 0.447783828699573
1.35837733654131 0.162160398870627
1.90159311981257 0.0350119043016127
2.44480890308383 0.0147418544427843
2.98802468635509 0.00368546361069608
3.53124046962636 0
4.07445625289762 0
4.61767203616888 0
5.16088781944014 0.00184273180534804
};
\addplot [semithick, black]
table {%
0 1.98849289405249
0.1 1.62991476612804
0.2 1.33599780657406
0.3 1.09508188787737
0.4 0.897609513470848
0.5 0.735746657480649
0.6 0.603071977146027
0.7 0.49432206850139
0.8 0.405182659230617
0.9 0.332117455000358
1 0.272227849349136
1.1 0.223137931612706
1.2 0.182900230977249
1.3 0.149918457385334
1.4 0.122884174310277
1.5 0.100724891112679
1.6 0.0825615157249139
1.7 0.0676734797476211
1.8 0.0554701524183586
1.9 0.0454674094015994
2 0.0372684268487507
2.1 0.0305479387996938
2.2 0.0250393333933042
2.3 0.0205240759742298
2.4 0.0168230394946777
2.5 0.0137893982752179
2.6 0.0113028032094164
2.7 0.00926460733391029
2.8 0.00759395235511458
2.9 0.00622456087919384
3 0.00510210708824046
3.1 0.00418206155343201
3.2 0.00342792468566663
3.3 0.0028097787420081
3.4 0.00230310094386025
3.5 0.00188779062148469
3.6 0.00154737179022311
3.7 0.00126833952342404
3.8 0.0010396241917061
3.9 0.000852152314123872
4 0.000698486599542265
4.1 0.000572530898119694
4.2 0.00046928835788196
4.3 0.000384663192094664
4.4 0.000315298193247901
4.5 0.000258441547588795
4.6 0.000211837666534221
4.7 0.000173637704081793
4.8 0.000142326210310317
4.9 0.000116661011203849
5 9.56239297416181e-05
5.1 7.8380393285398e-05
5.2 6.42463248286676e-05
5.3 5.26610046336628e-05
5.4 4.31648256366757e-05
5.5 3.53810601450924e-05
5.6 2.90009144836442e-05
5.7 2.37712786852234e-05
5.8 1.94846852380895e-05
5.9 1.59710785378752e-05
6 1.30910685261852e-05
6.1 1.07304008774899e-05
6.2 8.79542435831932e-06
6.3 7.20937554208252e-06
6.4 5.90933348856738e-06
6.5 4.84372357567835e-06
6.6 3.97027145666644e-06
6.7 3.25432597325968e-06
6.8 2.66748449213716e-06
6.9 2.18646612977897e-06
7 1.79218816482806e-06
7.1 1.46900899785466e-06
7.2 1.20410762559912e-06
7.3 9.86975012503906e-07
7.4 8.08997181479021e-07
7.5 6.63113484484907e-07
7.6 5.43536496013265e-07
7.7 4.45522417219224e-07
7.8 3.6518288229171e-07
7.9 2.99330701137896e-07
};
\end{groupplot}

\end{tikzpicture}\\
  \end{tabular}
  \caption{The interarrival process as seen by the shop owner. Observe
    that the `theory' line intersects the $y$-axis at level $N/5$,
    which is equal to the arrival rate when $N$ persons visit the
    shop. The parameter $n=100$ is the simulation length, i.e., the
    number of visits per customer, and $b=10$ is number of bins.}
  \label{fig:uniformfew}
\end{figure}

\begin{figure}[ht]
  \centering
  \begin{tabular}[h]{c}
% see progs/convergence_to_exp.py
 % This file was created by matplotlib2tikz v0.5.15.
\begin{tikzpicture}

\begin{groupplot}[group style={group size=3 by 1}]
\nextgroupplot[
title={$N=1, n=1000, b=50$},
xmin=0, xmax=8,
ymin=0, ymax=2,
width=5cm,
height=5cm,
tick align=outside,
xmajorgrids,
x grid style={white},
ymajorgrids,
y grid style={white},
axis line style={white},
axis background/.style={fill=white!89.803921568627459!black},
legend entries={{simulation},{theory}},
legend style={draw=white!80.0!black, fill=white!89.803921568627459!black},
legend cell align={left}
]
\addplot [semithick, black, mark=*, mark size=1, mark options={solid}, only marks]
table {%
4.02179055881267 0.452021556491336
4.06165152951034 0.452021556491336
4.10151250020802 0.477133865185299
4.14137347090569 0.753369260818894
4.18123444160336 0.426909247797373
4.22109541230104 0.276235395633594
4.26095638299871 0.652920026043041
4.30081735369638 0.452021556491336
4.34067832439406 0.577583099961152
4.38053929509173 0.602695408655115
4.42040026578941 0.602695408655115
4.46026123648708 0.552470791267189
4.50012220718475 0.627807717349078
4.53998317788243 0.376684630409447
4.5798441485801 0.502246173879263
4.61970511927777 0.602695408655115
4.65956608997545 0.40179693910341
4.69942706067312 0.527358482573226
4.73928803137079 0.652920026043041
4.77914900206847 0.477133865185299
4.81900997276614 0.502246173879263
4.85887094346382 0.426909247797373
4.89873191416149 0.502246173879263
4.93859288485916 0.577583099961152
4.97845385555684 0.477133865185299
5.01831482625451 0.452021556491336
5.05817579695218 0.602695408655115
5.09803676764986 0.527358482573226
5.13789773834753 0.602695408655115
5.17775870904521 0.376684630409447
5.21761967974288 0.602695408655115
5.25748065044055 0.477133865185299
5.29734162113823 0.326460013021521
5.3372025918359 0.426909247797373
5.37706356253357 0.426909247797373
5.41692453323125 0.502246173879263
5.45678550392892 0.527358482573226
5.49664647462659 0.502246173879263
5.53650744532427 0.477133865185299
5.57636841602194 0.351572321715484
5.61622938671962 0.452021556491336
5.65609035741729 0.502246173879263
5.69595132811496 0.40179693910341
5.73581229881264 0.652920026043041
5.77567326951031 0.652920026043041
5.81553424020798 0.351572321715484
5.85539521090566 0.728256952124931
5.89525618160333 0.301347704327558
5.935117152301 0.351572321715484
5.97497812299868 0.577583099961152
};
\addplot [semithick, black]
table {%
0 0.200699398877691
0.1 0.196711526050194
0.2 0.192802891774368
0.3 0.188971921589757
0.4 0.185217072320243
0.5 0.181536831452434
0.6 0.177929716526394
0.7 0.174394274538487
0.8 0.170929081356085
0.9 0.167532741143904
1 0.164203885801736
1.1 0.160941174413362
1.2 0.157743292706404
1.3 0.154608952522918
1.4 0.151536891300504
1.5 0.148525871563724
1.6 0.145574680425627
1.7 0.142682129099179
1.8 0.139847052418403
1.9 0.137068308369026
2 0.134344777628464
2.1 0.131675363114934
2.2 0.129058989545541
2.3 0.126494603003125
2.4 0.123981170511737
2.5 0.121517679620535
2.6 0.11910313799595
2.7 0.116736573021966
2.8 0.114417031408326
2.9 0.112143578806536
3 0.109915299433495
3.1 0.107731295702601
3.2 0.10559068786219
3.3 0.103492613641156
3.4 0.101436227901617
3.5 0.0994207022984773
3.6 0.0974452249457605
3.7 0.0955090000895648
3.8 0.0936112477875228
3.9 0.091751203594628
4 0.0899281182553035
4.1 0.0881412574015906
4.2 0.0863899012573332
4.3 0.0846733443482405
4.4 0.0829908952177106
4.5 0.081341876148301
4.6 0.0797256228887329
4.7 0.0781414843863202
4.8 0.076588822524715
4.9 0.0750670118668643
5 0.0735754394030739
5.1 0.0721135043040781
5.2 0.0706806176790161
5.3 0.0692762023382177
5.4 0.0678996925607016
5.5 0.066550533866294
5.6 0.0652281827922754
5.7 0.0639321066744648
5.8 0.0626617834326542
5.9 0.0614167013603065
6 0.0601963589184315
6.1 0.0590002645335581
6.2 0.057827936399721
6.3 0.0566789022843812
6.4 0.0555526993382031
6.5 0.0544488739086116
6.6 0.0533669813570535
6.7 0.05230658587989
6.8 0.0512672603328481
6.9 0.0502485860589597
7 0.0492501527199202
7.1 0.0482715581307977
7.2 0.0473124080980263
7.3 0.0463723162606185
7.4 0.0454509039345332
7.5 0.0445477999601354
7.6 0.0436626405526872
7.7 0.0427950691558095
7.8 0.0419447362978556
7.9 0.041111299451138
};
\nextgroupplot[
title={$N=3, n=1000, b=50$},
xmin=0, xmax=8,
ymin=0, ymax=2,
width=5cm,
height=5cm,
tick align=outside,
xmajorgrids,
x grid style={white},
ymajorgrids,
y grid style={white},
axis line style={white},
axis background/.style={fill=white!89.803921568627459!black},
legend entries={{simulation},{theory}},
legend style={draw=white!80.0!black, fill=white!89.803921568627459!black},
legend cell align={left}
]
\addplot [semithick, black, mark=*, mark size=1, mark options={solid}, only marks]
table {%
0.0593827380619275 0.388369678550795
0.177866082636755 0.368669767319958
0.296349427211582 0.348969856089121
0.414832771786409 0.396812497649726
0.533316116361236 0.346155583056144
0.651799460936063 0.334898490924237
0.77028280551089 0.393998224616749
0.888766150085717 0.326455671825306
1.00724949466054 0.368669767319958
1.12573283923537 0.354598402155074
1.2442161838102 0.306755760594469
1.36269952838503 0.303941487561492
1.48118287295985 0.222327569605165
1.59966621753468 0.233584661737073
1.71814956210951 0.264541665099817
1.83663290668433 0.25328457296791
1.95511625125916 0.205441931407305
2.07359959583399 0.185742020176467
2.19208294040881 0.236398934770049
2.31056628498364 0.222327569605165
2.42904962955847 0.230770388704096
2.5475329741333 0.208256204440282
2.66601631870812 0.205441931407305
2.78449966328295 0.151970743780746
2.90298300785778 0.171670655011583
3.0214663524326 0.135085105582885
3.13994969700743 0.146342197714792
3.25843304158226 0.129456559516932
3.37691638615709 0.129456559516932
3.49539973073191 0.129456559516932
3.61388307530674 0.112570921319071
3.73236641988157 0.118199467385025
3.85084976445639 0.0900567370552569
3.96933310903122 0.0816139179563266
4.08781645360605 0.0984995561541872
4.20629979818087 0.0675425527914427
4.3247831427557 0.0478426415606052
4.44326648733053 0.0422140954946517
4.56174983190536 0.025328457296791
4.68023317648018 0.0281427303297678
4.79871652105501 0.0140713651648839
4.91719986562984 0
5.03568321020466 0.00281427303297678
5.15416655477949 0
5.27264989935432 0.00281427303297678
5.39113324392915 0
5.50961658850397 0.00281427303297678
5.6280999330788 0.00281427303297678
5.74658327765363 0
5.86506662222845 0.00281427303297678
};
\addplot [semithick, black]
table {%
0 0.600671883787071
0.1 0.565653469855566
0.2 0.532676585330344
0.3 0.501622211619503
0.4 0.472378268765078
0.5 0.444839210929425
0.6 0.418905645464235
0.7 0.394483974187357
0.8 0.371486055572717
0.9 0.349828886634144
1 0.329434303354973
1.1 0.310228698582217
1.2 0.292142756367157
1.3 0.275111201793541
1.4 0.259072565390485
1.5 0.243968961279797
1.6 0.229745878257047
1.7 0.216351983052333
1.8 0.203738935060704
1.9 0.191861211873565
2 0.180675944981382
2.1 0.170142765054708
2.2 0.160223656245132
2.3 0.150882818980307
2.4 0.14208654075784
2.5 0.133803074471747
2.6 0.126002523832323
2.7 0.118656735465876
2.8 0.111739197304929
2.9 0.105224942902129
3 0.0990904613225452
3.1 0.0933136122891272
3.2 0.0878735462750786
3.3 0.0827506292547428
3.4 0.0779263718414188
3.5 0.0733833625563521
3.6 0.0691052049880598
3.7 0.0650764586151878
3.8 0.0612825830793205
3.9 0.0577098857066161
4 0.0543454720888638
4.1 0.0511771995456038
4.2 0.048193633299346
4.3 0.0453840052057196
4.4 0.0427381748896017
4.5 0.0402465931469616
4.6 0.0379002674803306
4.7 0.0356907296435115
4.8 0.0336100050783894
4.9 0.0316505841335392
5 0.029805394960751
5.1 0.0280677779916545
5.2 0.0264314619023238
5.3 0.0248905409791156
5.4 0.0234394538040503
5.5 0.0220729631828086
5.6 0.0207861372429008
5.7 0.019574331633789
5.8 0.018433172764719
5.9 0.0173585420197669
6 0.0163465608931273
6.1 0.0153935769909958
6.2 0.0144961508495247
6.3 0.0136510435212747
6.4 0.0128552048853607
6.5 0.0121057626391019
6.6 0.0114000119314446
6.7 0.0107354056007428
6.8 0.0101095449816654
6.9 0.00952017124804706
7 0.00896515726044197
7.1 0.00844249988895347
7.2 0.00795031278363384
7.3 0.0074868195663603
7.4 0.00705034741961631
7.5 0.00663932104903867
7.6 0.0062522569979405
7.7 0.00588775829329056
7.8 0.00554450940382499
7.9 0.00522127149209498
};
\nextgroupplot[
title={$N=10, n=1000, b=50$},
xmin=0, xmax=8,
ymin=0, ymax=2,
width=5cm,
height=5cm,
tick align=outside,
xmajorgrids,
x grid style={white},
ymajorgrids,
y grid style={white},
axis line style={white},
axis background/.style={fill=white!89.803921568627459!black},
legend entries={{simulation},{theory}},
legend style={draw=white!80.0!black, fill=white!89.803921568627459!black},
legend cell align={left}
]
\addplot [semithick, black, mark=*, mark size=1, mark options={solid}, only marks]
table {%
0.0586122871830139 1.60197593818587
0.175729378272817 1.33896496325983
0.292846469362621 1.13914493685498
0.409963560452425 0.927369866135315
0.527080651542228 0.748044201413016
0.644197742632032 0.600314010951313
0.761314833721835 0.485033226486979
0.878431924811639 0.410741165387741
0.995549015901442 0.294606449186633
1.11266610699125 0.225437978508032
1.22978319808105 0.185303186879708
1.34690028917085 0.129797623989473
1.46401738026066 0.11698864793788
1.58113447135046 0.0922246275714678
1.69825156244026 0.0742920610992379
1.81536865353007 0.0495280407328253
1.93248574461987 0.0315954742605954
2.04960283570967 0.0222022251560941
2.16671992679948 0.0153707712619113
2.28383701788928 0.0153707712619113
2.40095410897909 0.00939324910450135
2.51807120006889 0.0085393173677285
2.63518829115869 0.0034157269470914
2.7523053822485 0.0034157269470914
2.8694224733383 0.0017078634735457
2.9865395644281 0
3.10365665551791 0.0017078634735457
3.22077374660771 0.00085393173677285
3.33789083769751 0.00085393173677285
3.45500792878732 0.00085393173677285
3.57212501987712 0.00085393173677285
3.68924211096692 0
3.80635920205673 0
3.92347629314653 0
4.04059338423634 0
4.15771047532614 0.00085393173677285
4.27482756641594 0
4.39194465750575 0.00085393173677285
4.50906174859555 0
4.62617883968535 0
4.74329593077516 0
4.86041302186496 0
4.97753011295476 0
5.09464720404457 0
5.21176429513437 0
5.32888138622418 0
5.44599847731398 0
5.56311556840378 0
5.68023265949359 0
5.79734975058339 0.00085393173677285
};
\addplot [semithick, black]
table {%
0 2.00318630389195
0.1 1.63954773817039
0.2 1.34192050959861
0.3 1.0983215750039
0.4 0.898943173973758
0.5 0.73575794960701
0.6 0.602195751725809
0.7 0.492879110025666
0.8 0.403406726805179
0.9 0.330176272277382
1 0.270239347862031
1.1 0.221182778002726
1.2 0.18103145108972
1.3 0.148168797677574
1.4 0.121271704298152
1.5 0.0992572423742113
1.6 0.0812390674374569
1.7 0.0664917331999384
1.8 0.0544214836948427
1.9 0.0445423475222447
2 0.0364565717082873
2.1 0.0298386074074282
2.2 0.0244220026814051
2.3 0.0199886746330553
2.4 0.0163601289705191
2.5 0.013390273484636
2.6 0.0109595360963501
2.7 0.00897005065542657
2.8 0.00734171666150316
2.9 0.00600897426428523
3 0.00491816469820689
3.1 0.00402536987759353
3.2 0.00329464417028266
3.3 0.00269656715751715
3.4 0.00220705911144761
3.5 0.00180641149909613
3.6 0.00147849347901082
3.7 0.0012101024426446
3.8 0.000990432452007938
3.9 0.000810639171875927
4 0.000663483779885699
4.1 0.000543041517661561
4.2 0.000444463148677085
4.3 0.000363779718690066
4.4 0.000297742758030922
4.5 0.000243693491982133
4.6 0.000199455793407671
4.7 0.000163248567699956
4.8 0.0001336140424942
4.9 0.000109359074956507
5 8.95071135645195e-05
5.1 7.32588802697722e-05
5.2 5.99601900301728e-05
5.3 4.90756120652565e-05
5.4 4.01669123858281e-05
5.5 3.28754096528732e-05
5.6 2.69075339787769e-05
5.7 2.20230072404819e-05
5.8 1.80251690213183e-05
5.9 1.4753058685367e-05
6 1.20749347934805e-05
6.1 9.88297094021753e-06
6.2 8.08891445591241e-06
6.3 6.62053318489556e-06
6.4 5.41870728034042e-06
6.5 4.43504892581813e-06
6.6 3.62995414898381e-06
6.7 2.97100828967611e-06
6.8 2.43168092351668e-06
6.9 1.99025769613035e-06
7 1.62896606158242e-06
7.1 1.33325972558559e-06
7.2 1.09123298378713e-06
7.3 8.9314137527251e-07
7.4 7.31009351875753e-07
7.5 5.98309167310451e-07
7.6 4.89698057581853e-07
7.7 4.0080313105918e-07
7.8 3.28045307469881e-07
7.9 2.68495217261961e-07
};
\end{groupplot}

\end{tikzpicture}\\
  \end{tabular}
  \caption{Each of the $N$ customer visits the shop with exponentially
    distributed interarrival times, but now the number of visits is
    $n=1000$.}  
  \label{fig:uniformmany}
\end{figure}


\begin{figure}[ht]
  \centering
  \begin{tabular}[h]{c}
% see progs/convergence_to_exp.py
 % This file was created by matplotlib2tikz v0.5.15.
\begin{tikzpicture}

\begin{groupplot}[group style={group size=3 by 1}]
\nextgroupplot[
title={$N=1, n=1000, b=50$},
xmin=0, xmax=8,
ymin=0, ymax=2,
width=5cm,
height=5cm,
tick align=outside,
xmajorgrids,
x grid style={white},
ymajorgrids,
y grid style={white},
axis line style={white},
axis background/.style={fill=white!89.803921568627459!black},
legend entries={{simulation},{theory}},
legend style={draw=white!80.0!black, fill=white!89.803921568627459!black},
legend cell align={left}
]
\addplot [semithick, black, mark=*, mark size=1, mark options={solid}, only marks]
table {%
1.80383749056507 0.00772836331907271
1.9333605162917 0.00772836331907271
2.06288354201833 0
2.19240656774497 0.00772836331907271
2.3219295934716 0
2.45145261919823 0.00772836331907271
2.58097564492486 0.0386418165953636
2.71049867065149 0.00772836331907271
2.84002169637812 0.0463701799144363
2.96954472210475 0.0618269065525817
3.09906774783138 0.0618269065525817
3.22859077355801 0.0927403598288726
3.35811379928464 0.0927403598288726
3.48763682501127 0.154567266381454
3.6171598507379 0.139110539743309
3.74668287646453 0.208665809614963
3.87620590219116 0.270492716167545
4.00572892791779 0.247307626210327
4.13525195364442 0.224122536253109
4.26477497937105 0.347776349358272
4.39429800509768 0.347776349358272
4.52382103082431 0.316862896081981
4.65334405655094 0.409603255910854
4.78286708227757 0.386418165953636
4.9123901080042 0.301406169443836
5.04191313373083 0.355504712677345
5.17143615945746 0.37096143931549
5.30095918518409 0.417331619229927
5.43048221091072 0.417331619229927
5.56000523663735 0.262764352848472
5.68952826236398 0.278221079486618
5.81905128809061 0.347776349358272
5.94857431381724 0.262764352848472
6.07809733954387 0.154567266381454
6.2076203652705 0.154567266381454
6.33714339099713 0.185480719657745
6.46666641672376 0.1700239930196
6.5961894424504 0.162295629700527
6.72571246817703 0.100468723147945
6.85523549390366 0.0618269065525817
6.98475851963029 0.0695552698716544
7.11428154535692 0.0618269065525817
7.24380457108355 0.054098543233509
7.37332759681018 0.0154567266381454
7.50285062253681 0.0154567266381454
7.63237364826344 0.00772836331907271
7.76189667399007 0
7.8914196997167 0
8.02094272544333 0
8.15046575116996 0.00772836331907271
};
\addplot [semithick, black]
table {%
0 0.200034071183864
0.1 0.19607246314734
0.2 0.192189313436151
0.3 0.188383068209343
0.4 0.184652204399261
0.5 0.1809952291021
0.6 0.177410678980515
0.7 0.173897119678068
0.8 0.170453145245274
0.9 0.167077377577007
1 0.163768465861052
1.1 0.160525086037582
1.2 0.157345940269331
1.3 0.154229756422268
1.4 0.151175287556552
1.5 0.148181311427569
1.6 0.145246629996852
1.7 0.142370068952685
1.8 0.139550477240206
1.9 0.13678672660081
2 0.134077711120675
2.1 0.131422346788235
2.2 0.128819571060411
2.3 0.126268342437436
2.4 0.123767640046098
2.5 0.121316463231236
2.6 0.118913831155335
2.7 0.116558782406036
2.8 0.114250374611432
2.9 0.111987684062978
3 0.109769805345868
3.1 0.107595850976734
3.2 0.10546495104852
3.3 0.103376252882389
3.4 0.101328920686524
3.5 0.0993221352216871
3.6 0.0973550934733978
3.7 0.0954270083306111
3.8 0.0935371082707534
3.9 0.091684637050998
4 0.0898688534056552
4.1 0.0880890307495549
4.2 0.0863444568873039
4.3 0.0846344337283011
4.4 0.0829582770073969
4.5 0.0813153160110845
4.6 0.0797048933091144
4.7 0.0781263644914241
4.8 0.0765790979102769
4.9 0.0750624744275091
5 0.0735758871667817
5.1 0.0721187412707389
5.2 0.0706904536629767
5.3 0.0692904528147245
5.4 0.0679181785161482
5.5 0.0665730816521829
5.6 0.0652546239828041
5.7 0.0639622779276513
5.8 0.0626955263549173
5.9 0.0614538623744176
6 0.0602367891347586
6.1 0.059043819624523
6.2 0.057874476477392
6.3 0.0567282917811274
6.4 0.0556048068903369
6.5 0.0545035722429471
6.6 0.0534241471803114
6.7 0.0523660997708808
6.8 0.0513290066373664
6.9 0.0503124527873252
7 0.0493160314471012
7.1 0.0483393438990548
7.2 0.0473819993220164
7.3 0.0464436146348992
7.4 0.0455238143434093
7.5 0.0446222303897922
7.6 0.0437385020055545
7.7 0.0428722755671021
7.8 0.0420232044542387
7.9 0.0411909489114648
};
\nextgroupplot[
title={$N=3, n=1000, b=50$},
xmin=0, xmax=8,
ymin=0, ymax=2,
width=5cm,
height=5cm,
tick align=outside,
xmajorgrids,
x grid style={white},
ymajorgrids,
y grid style={white},
axis line style={white},
axis background/.style={fill=white!89.803921568627459!black},
legend entries={{simulation},{theory}},
legend style={draw=white!80.0!black, fill=white!89.803921568627459!black},
legend cell align={left}
]
\addplot [semithick, black, mark=*, mark size=1, mark options={solid}, only marks]
table {%
0.0635620497560896 0.349465945565713
0.185693817308201 0.352196148265445
0.307825584860312 0.401339796860623
0.429957352412423 0.384958580662231
0.552089119964535 0.33308472936732
0.674220887516646 0.346735742865981
0.796352655068757 0.33308472936732
0.918484422620868 0.28121087807241
1.04061619017298 0.297592094270802
1.16274795772509 0.335814932067052
1.2848797252772 0.330354526667588
1.40701149282931 0.308512905069731
1.52914326038142 0.259369256474553
1.65127502793354 0.25663905377482
1.77340679548565 0.300322296970535
1.89553856303776 0.199304797080446
2.01767033058987 0.262099459174285
2.13980209814198 0.270290067273481
2.26193386569409 0.221146418678303
2.3840656332462 0.215686013278838
2.50619740079831 0.221146418678303
2.62832916835043 0.210225607879374
2.75046093590254 0.177463175482589
2.87259270345465 0.144700743085803
2.99472447100676 0.172002770083124
3.11685623855887 0.158351756584464
3.23898800611098 0.111938310689017
3.36111977366309 0.163812161983928
3.4832515412152 0.106477905289553
3.60538330876732 0.117398716088482
3.72751507631943 0.0873664863914282
3.84964684387154 0.0791758782922318
3.97177861142365 0.0791758782922318
4.09391037897576 0.0737154728927676
4.21604214652787 0.0354926350965177
4.33817391407998 0.062794662093839
4.46030568163209 0.0327624323967856
4.58243744918421 0.0354926350965177
4.70456921673632 0.0191114188981249
4.82670098428843 0.0191114188981249
4.94883275184054 0.0081906080991964
5.07096451939265 0.0081906080991964
5.19309628694476 0.0081906080991964
5.31522805449687 0.00273020269973213
5.43735982204898 0.00273020269973213
5.55949158960109 0.00273020269973213
5.68162335715321 0.00273020269973213
5.80375512470532 0
5.92588689225743 0.00273020269973213
6.04801865980954 0.00273020269973213
};
\addplot [semithick, black]
table {%
0 0.599134510418468
0.1 0.564292469570393
0.2 0.531476631168261
0.3 0.500569163527934
0.4 0.47145908734363
0.5 0.444041877195027
0.6 0.418219086228306
0.7 0.393897992663467
0.8 0.370991266858656
0.9 0.349416657736007
1 0.329096697443062
1.1 0.309958423189307
1.2 0.291933115259
1.3 0.274956050259597
1.4 0.258966268719757
1.5 0.24390635620244
1.6 0.229722237147146
1.7 0.216362980701041
1.8 0.20378061784177
1.9 0.191929969135305
2 0.180768482510354
2.1 0.170256080466828
2.2 0.160355016169738
2.3 0.15102973791181
2.4 0.142246761458122
2.5 0.133974549814415
2.6 0.126183399987349
2.7 0.118845336330095
2.8 0.111934010090309
2.9 0.105424604799782
3 0.0992937471660584
3.1 0.0935194231460618
3.2 0.0880808989003667
3.3 0.082958646344295
3.4 0.0781342730285096
3.5 0.0735904560973291
3.6 0.0693108800876267
3.7 0.0652801783449706
3.8 0.061483877846646
3.9 0.0579083472334375
4 0.0545407478635686
4.1 0.0513689877130478
4.2 0.0483816779568928
4.3 0.0455680920753266
4.4 0.0429181273381118
4.5 0.0404222685287219
4.6 0.0380715537780962
4.7 0.0358575423852963
4.8 0.0337722845095189
4.9 0.0318082926246389
5 0.0299585146337815
5.1 0.0282163085473906
5.2 0.0265754186338661
5.3 0.0250299529571366
5.4 0.0235743622205108
5.5 0.0222034198408428
5.6 0.0209122031814631
5.7 0.0196960758764897
5.8 0.0185506711830492
5.9 0.0174718763016334
6 0.0164558176082878
6.1 0.015498846745608
6.2 0.0145975275225985
6.3 0.0137486235763577
6.4 0.0129490867512837
6.5 0.0121960461540756
6.6 0.01148679784523
6.7 0.0108187951300175
6.8 0.0101896394140769
6.9 0.00959707159079385
7 0.00903896392953629
7.1 0.00851331243562182
7.2 0.00801822965458291
7.3 0.00755193789489258
7.4 0.00711276284481544
7.5 0.00669912756046395
7.6 0.00630954680347313
7.7 0.00594262170796176
7.8 0.00559703475763095
7.9 0.00527154505496423
};
\nextgroupplot[
title={$N=10, n=1000, b=50$},
xmin=0, xmax=8,
ymin=0, ymax=2,
width=5cm,
height=5cm,
tick align=outside,
xmajorgrids,
x grid style={white},
ymajorgrids,
y grid style={white},
axis line style={white},
axis background/.style={fill=white!89.803921568627459!black},
legend entries={{simulation},{theory}},
legend style={draw=white!80.0!black, fill=white!89.803921568627459!black},
legend cell align={left}
]
\addplot [semithick, black, mark=*, mark size=1, mark options={solid}, only marks]
table {%
0.0446994018784085 1.73660414063981
0.134078160065044 1.44232135134324
0.223456918251679 1.27112261840646
0.312835676438315 1.09209126370132
0.40221443462495 0.913059908996188
0.491593192811586 0.773191663132801
0.580971950998221 0.647869714839207
0.670350709184856 0.544926685883754
0.759729467371492 0.434151035159952
0.849108225558127 0.396106872285111
0.938486983744763 0.347992195708106
1.0278657419314 0.265190194156981
1.11724450011803 0.22714603128214
1.20662325830467 0.191339760341113
1.2960020164913 0.147700867631736
1.38538077467794 0.1577713813339
1.47475953286458 0.135392461995758
1.56413829105121 0.0883967313856603
1.65351704923785 0.0559472983453546
1.74289580742448 0.0738504338158681
1.83227456561112 0.0514715144777263
1.92165332379775 0.0324494330403057
2.01103208198439 0.0346873249741199
2.10041084017102 0.0391631088417482
2.18978959835766 0.0234978653050489
2.27916835654429 0.0201410274043277
2.36854711473093 0.0134273516028851
2.45792587291756 0.0100705137021638
2.5473046311042 0.00559472983453546
2.63668338929084 0.00223789193381419
2.72606214747747 0.00335683790072128
2.81544090566411 0.00111894596690709
2.90481966385074 0.00111894596690709
2.99419842203738 0.00223789193381419
3.08357718022401 0.00111894596690709
3.17295593841065 0
3.26233469659728 0.00111894596690709
3.35171345478392 0
3.44109221297055 0
3.53047097115719 0
3.61984972934383 0.00111894596690709
3.70922848753046 0
3.7986072457171 0
3.88798600390373 0
3.97736476209037 0
4.066743520277 0
4.15612227846364 0
4.24550103665027 0
4.33487979483691 0.00111894596690709
4.42425855302354 0.00111894596690709
};
\addplot [semithick, black]
table {%
0 2.005024722117
0.1 1.64075076185297
0.2 1.34265828886074
0.3 1.09872341525568
0.4 0.899106759513194
0.5 0.73575656418881
0.6 0.60208391942245
0.7 0.492696992009538
0.8 0.403183539876145
0.9 0.329932939440215
1 0.269990547136672
1.1 0.220938520618273
1.2 0.18079829242422
1.3 0.147950762284638
1.4 0.121070988929722
1.5 0.0990747471258075
1.6 0.0810747942576107
1.7 0.0663450824211259
1.8 0.0542914724825564
1.9 0.0444277688226309
2 0.0363561081013472
2.1 0.0297509110023898
2.2 0.0243457496331771
2.3 0.0199226008626671
2.4 0.0163030521184796
2.5 0.013341104919485
2.6 0.0109172858664274
2.7 0.00893382755091141
2.8 0.00731072499941254
2.9 0.00598250858464163
3 0.00489560323609312
3.1 0.00400616743062828
3.2 0.00327832479640135
3.3 0.00268271699992691
3.4 0.00219531954539618
3.5 0.00179647272020485
3.6 0.00147008859881388
3.7 0.00120300211857163
3.8 0.000984440052426422
3.9 0.000805586458959868
4 0.000659227081689674
4.1 0.000539458354096706
4.2 0.000441449272773839
4.3 0.000361246533588054
4.4 0.000295615070808466
4.5 0.000241907567170589
4.6 0.00019795767142166
4.7 0.000161992616159261
4.8 0.000132561711307594
4.9 0.000108477828937102
5 8.87695191532507e-05
5.1 7.26418255961627e-05
5.2 5.94442200011634e-05
5.3 4.86443624254588e-05
5.4 3.98066287308183e-05
5.5 3.25745392046479e-05
5.6 2.66563795585542e-05
5.7 2.18134343115533e-05
5.8 1.78503579384904e-05
5.9 1.4607295393347e-05
6 1.19534341800733e-05
6.1 9.7817278866301e-06
6.2 8.0045783501767e-06
6.3 6.55030228879033e-06
6.4 5.36023987741774e-06
6.5 4.38638863928912e-06
6.6 3.5894672132012e-06
6.7 2.93733089659254e-06
6.8 2.40367505359729e-06
6.9 1.96697408861505e-06
7 1.609613187728e-06
7.1 1.31717780580023e-06
7.2 1.07787223993962e-06
7.3 8.82043836842971e-07
7.4 7.21793642404461e-07
7.5 5.90657788710618e-07
7.6 4.83346766815968e-07
7.7 3.95532068579746e-07
7.8 3.23671591527453e-07
7.9 2.64866764250223e-07
};
\end{groupplot}

\end{tikzpicture}
  \end{tabular}
  \caption{Each of the $N$ customer visits the shop with normally
    distributed interarrival times with $\mu=5$ and
    $\sigma=1$.}  \label{fig:normal}
\end{figure}

As the graphs show, even when the customer population consists of 10
members each visiting the shop with an inter-arrival time that is
quite `far' from exponential, the distribution of the inter-arrival
times as observed by the shop is very well approximated by an
exponential distribution. Thus, for a real shop, with many thousands
of customers, or a hospital, call center, in fact for nearly every
system that deals with random demand, it seems reasonable to use the
exponential distribution to model inter-arrival times.  In conclusion,
the main conditions to use an exponential distribution are: 1)
arrivals have to be drawn from a large population, and 2) each of the
arriving customers decides, independent of the others, to visit the
system.

Another, but theoretical, reason to use the exponential distribution
is that an exponentially distributed random variable is
\recall{memoryless}, that is, $X$ is memoryless if it satisfies the
property that
\begin{equation*}
  \P{X > t+h | X>t} = \P{X>h}.
\end{equation*}
In words, the probability that $X$ is larger than some time $t+h$,
conditional on it being larger than a time~$t$, is equal to the
probability that $X$ is larger than $h$. Thus, no matter how long we
have been waiting for the next arrival to occur, the probability that
it will occur in the next $h$ seconds remains the same.  This property
seems to be vindicated also in practice: suppose that a patient with a
broken arm just arrived at the emergency room of a hospital, what does
that tell us about the time the next patient will be brought in? Not
much, as most of us will agree.

It can be shown that only exponential random variables have the
memoryless property. The proof of this fact requires quite some work;
we refer the reader to the literature if s/he want to check this.

Finally, the reader should realize that it is simple, by means of
computers, to generate exponentially distributed inter-arrival
times. Thus, it is easy to use such exponentially distributed random
variables to simulate queueing systems. 



\begin{question}(Conditional probability)
  We have to give one present to one of three children. As we cannot
  divide the present into parts, we decide to let `fate decide'. That
  is, we choose a random number in the set $\{1, 2, 3\}$. The first
  child that guesses the number wins the present. Show that the
  probability of winning the present is the same for each child.
\hint{For the second child, condition on the event that the first does not chose the right number.}
\begin{solution}
    Use the definition of conditional probability
    ($\P{A|B} = \P{AB}/\P{B}$, provided $\P{B}>0$)

    The probability that the first child to guess also wins is
    $1/3$. What is the probability for child number two? Well, for
    him/her to win, it is necessary that child one does not win and
    that child two guesses the right number of the remaining
    numbers. Assume, without loss of generality that child 1 chooses
    $3$ and that this is not the right number. Then 
    \begin{equation*}
      \begin{split}
&\P{\text{Child  2 wins}} \\
&= \P{\text{Child 2 guesses the right number and child 1 does not win}} \\
&= \P{\text{Child 2 guesses the right number} \given \text{ child 1 does not win}}
\cdot \P{\text{Child 1 does not win}} \\
&= \P{\text{Child 2 makes the right guess in the set $\{1,2\}$}}\cdot \frac 23 \\
&= \frac 1 2\cdot \frac 23  = \frac 1 3.
      \end{split}
    \end{equation*}
    Similar conditional reasoning gives that child 3 wins with probability $1/3$. 
  \end{solution}
\end{question}

\begin{question}
  Assume that the time $X$ to fail of a machine is uniformly
  distributed on the interval $[0,10]$. If the machine fails at time
  $t$, the cost to repair it is $h(t)$. What is the expected repair
  cost? 
  \begin{solution}
    Write for $F(x) = \P{X\leq x}$ and $f(x) = \d F(x)/\d x$ for the
    density of $F$.
    \begin{equation*}
      \begin{split}
\E{h(X)}
&= \int_0^{10} \E{h(X) \given X = x} \P{X\in \d x} \\
&= \int_0^{10} \E{h(x) \given X = x} \d F(x) \\
&= \int_0^{10} \E{h(x) \given X = x} F(\d x) \\
&= \int_0^{10} \E{h(x) \given X = x} f(x) \d x \\
&= \int_0^{10} h(x)\frac{\d x}{10}.
      \end{split}
    \end{equation*}
    Here we introduce some notation that is commonly used in the
    probabity literature to indicate the same conceptual idea, i.e,
    $\P{X\in \d x} = \d F(x) = F(\d x) = f(x) \d x$, where the last
    equality follows from the fact that $F$ has a density $f$
    everywhere on $[0,10]$. 

    The concept of conditional expectation is of fundamental
    importance in probability theory. Any \emph{good} probability book
    defines this concept as a random variable measurable with respect
    to some $\sigma$-algebra. In this course we will not deal with
    this elegant idea, due to lack of time. 

  \end{solution}
\end{question}


\begin{question}
  Show that an exponentially distributed random variable is
  memoryless.  
   \hint{Condition on the event ${X>t}$. }
  \begin{solution}
This is easy, but be sure you can do it. 

To see that an exponentially
distributed is memoryless, use the definition of conditional
probability ($\P{A|B} = \P{AB}/\P{B}$, provided $\P{B}>0$):
\begin{equation*}
  \P{X>t+h|X>t} = \frac{\P{X>t+h, X>t}}{\P{X>t}} = \frac{\P{X>t+h}}{\P{X>t}} = \frac{e^{-\lambda(t+h)}}{e^{-\lambda t}} = e^{-\lambda h} = \P{X>h}.
\end{equation*}
  \end{solution}
\end{question}

\begin{question} 
  If the random variable $X\sim\exp(\lambda)$, show that
 \begin{equation*}
    \E X = \int_0^\infty t \,\d F(t) = \int_0^\infty t f(t)\, \d t =
    \int_0^\infty t \lambda e^{-\lambda t}\, \d t = \frac{1}\lambda,
  \end{equation*}
  where~$f$ is the density function of the distribution function $F$ of $X$.
  \begin{solution}
    \begin{equation*}
      \begin{split}
\E{X} 
&= \int_0^\infty \E{X\given X=t} f(t) \d t \\
&= \int_0^\infty t \lambda e^{-\lambda t} \d t, \quad\text{density is } \lambda e^{-\lambda t} \\
&=   \lambda^{-1} \int_0^\infty u e^{-u}\, \d u, \quad \text{ by  change of variable $u=\lambda t$},   \\
&=  -\lambda^{-1}\left. t e^{- t}\right|_0^\infty + \lambda^{-1} \int_0^\infty e^{- t} \d t\\
&=  - \lambda^{-1} \left. e^{- t} \right|_0^\infty =  \frac1\lambda.
      \end{split}
    \end{equation*}
  \end{solution}
\end{question}

\begin{question} 
  If the random variable $X\sim\exp(\lambda)$, show that
  \begin{equation*}
  \E{X^2}= \int_0^\infty t^2 \lambda e^{-\lambda t}\, \d t =  \frac{2}{\lambda^2}.
  \end{equation*}
  \begin{solution}
    \begin{equation*}
      \begin{split}
\E{X^2} 
&= \int_0^\infty \E{X^2\given X=t} f(t) \d t \\
&= \int_0^\infty t^2 \lambda e^{-\lambda t} \d t \\
&=   \lambda^{-2} \int_0^\infty u^2 e^{-u}\, \d u, \quad \text{ by  change of variable $u=\lambda t$},   \\
&= -\lambda^{-2}\left. t^2 e^{- t}\right|_0^\infty + 2\lambda^{-2}\int_0^\infty t e^{- t} \d t \\
&=  -2\lambda^{-2}\left. t e^{- t}\right|_0^\infty + 2\lambda^{-2} \int_0^\infty e^{- t} \d t\\
&=  - 2\lambda^{-2} \left. e^{- t} \right|_0^\infty \\
&=  2/\lambda^2.
      \end{split}
    \end{equation*}
  \end{solution}
\end{question}

\begin{question} 
  If the random variable $X\sim\exp(\lambda)$, show that the \recall{variance}
  \begin{equation*}
  \V X = \E X^2 - (\E X)^2 = \frac{1}{\lambda^2}.
  \end{equation*}
  Recall in particular this middle term to compute $\V X$; it is very
  practical.
  \begin{solution}
    By the previous problems, $\E{X^2}=2/\lambda^2$ and $\E X = \lambda$. 
  \end{solution}
\end{question}

\begin{question}
 Define the \recall{square coefficient of variation (SCV)} as
  \begin{equation}
C_a^2 = \frac{\V X}{(\E X)^2}.
  \end{equation}
  Prove that when $X$ is exponentially distributed, $C_a^2 =1$.  As
  will become clear later, the SCV is a very important concept in
  queueing theory. Memorize it as a measure of \emph{relative
  variability}.
\begin{solution}
  By the previous problems, $\V X = 1/\lambda^2$ and $\E X=1/\lambda$.
\end{solution}
\end{question}

\begin{question}\label{ex:33}
 If $X$ is an exponentially distributed random variable with
    parameter $\lambda$, show that its moment generating function
    \begin{equation*}
    M_X(t) = \E{e^{tX}} = \frac{\lambda}{\lambda-t}.
    \end{equation*}
    \begin{solution}
    \begin{equation*}
      \begin{split}
      M_X(t) &= \E{\exp(t X)} = \int_0^\infty e^{tx} \,\d F(x) 
=\int_0^\infty e^{tx} f(x) \,\d x 
=\int_0^\infty e^{tx} \lambda e^{-\lambda x} \,\d x  \\
&=\lambda \int_0^\infty e^{(t-\lambda)x} \,\d x 
=\frac{\lambda}{\lambda -t}.
      \end{split}
    \end{equation*}
    \end{solution}
  \end{question}

  \begin{question}
 Let $A_i$ be the arrival time of customer $i$ and set $A_0=0$.
    Assume that the inter-arrival times $\{X_i\}$ are i.i.d.  with
    exponential distribution with mean $1/\lambda$ for some
    $\lambda>0$.  Prove that the density of
    $A_i=X_1+X_2+\cdots+X_i=\sum_{k=1}^i X_k$ with $i\geq 1$ is
\begin{equation*}
f_{A_i}(t) = \lambda e^{-\lambda t} \frac{(\lambda t)^{i-1}}{(i-1)!}. 
\end{equation*}
 \hint{ Check the result for $i=1$ by filling in $i=1$ (just to be
     sure that you have read the formula right), and compare the result
     to the exponential density. Then write $A_i =\sum_{i=1}^k X_k$, and compute the moment
     generating function for $A_i$ and use that the inter-arrival times
     $X_i$ are independent. Use the moment generating function  of $X_i$.}
\begin{solution}
 One way to find the distribution of $A_i$ is by using the
    moment generating function $M_{A_i}(t) = \E{e^{t A_i}}$ of
    $A_i$. Let $X_i$ be the inter-arrival time between customers $i$
    and $i-1$, and $M_X(t)$ the associated moment generating
    function. Using the i.i.d. property of the $\{X_i\}$,
\begin{equation*}
  \begin{split}
  M_{A_i}(t) &= \E{e^{t A_i}} = \E{\exp\left(t\sum_{j=1}^{i} X_j\right)} \\
& = \prod_{j=1}^{i} \E{e^{tX_j}} = 
\prod_{j=1}^{i} M_{X_j}(t) = 
\prod_{j=1}^{i} \frac{\lambda}{\lambda -t }
 = \left(\frac{\lambda}{\lambda -t }\right)^i.
  \end{split}
\end{equation*}
From a table of moment generation functions it follows immediately that
$A_i \sim \Gamma(n,\lambda)$, i.e., $A_i$ is Gamma distributed.
\end{solution}

\end{question}

  \begin{question}
    Assume that the inter-arrival times $\{X_i\}$ are i.i.d. and
    $X_i\sim\exp(\lambda)$. Let
    $A_i=X_1+X_2+\cdots+X_i=\sum_{k=1}^i X_k$ with $i\geq 1$. Use the
    density of $A_i$ or the moment generating function of $A_i$ to
    show that
 \begin{equation*}
\E{A_i} = \frac i\lambda,
 \end{equation*}
 that is, the expected time to see $i$ jobs is $i/\lambda$.
  \hint{Use that
    $\int_0^\infty (\lambda t)^i e^{- \lambda t} \d t =
    \frac{i!}\lambda$.
    Another way would be to use that, once you have the moment
    generating function of some random variable $X$,
    $\E X = \frac{\d}{\d t} M(t) |_{t=0}$. }
  \begin{solution}
  \begin{equation*}
\E{A_i} = \int_0^\infty t \P{A_i \in \d t} = \int_0^\infty t f_{A_i} (t) \, \d t  = 
\int_0^\infty t  \lambda e^{-\lambda t} \frac{(\lambda t)^{i-1}}{(i-1)!}\, \d t.
  \end{equation*}
Thus, 
  \begin{equation*}
\E{A_i} = \frac{1}{(i-1)!} \int_0^\infty   e^{-\lambda t} (\lambda t)^i\,\d t = \frac{i!}{(i-1)!\lambda}=\frac{i}\lambda,
  \end{equation*}
  where we used the hint.

What if we would use the moment generating function? 
\begin{equation*}
  \begin{split}
    \E{A_i} 
&= \left.\frac{\d}{\d t} M_{A_i}(t)\right|_{t=0} \\
&= \left.\frac{\d}{\d t} \left(\frac{\lambda}{\lambda-t}\right)^i\right|_{t=0} \\
&= i \left.\left(\frac{\lambda}{\lambda-t}\right)^{i-1}\frac{\lambda}{(\lambda-t)^2}\right|_{t=0} \\
&= \frac i\lambda \left.\left(\frac{\lambda}{\lambda-t}\right)^{i-1}\right|_{t=0} \\
&= \frac i\lambda.
  \end{split}
\end{equation*}

And indeed, by the fact that $\E{X+Y}= \E X + \E Y$ for any r.v. $X$ and $Y$,
\begin{equation*}
\E{A_i} = \E{\sum_{k=1}^i X_k} = i \E{X} = \frac 1 \lambda.
\end{equation*}
We get the same answer. 
  \end{solution}
\end{question}


\begin{question}\label{ex:10}
  If $X\sim\exp(\lambda)$ and $Y\sim\exp(\mu)$ and $X$ and $Y$ are
  independent, show that 
  \begin{equation*}
Z=\min\{X,Y\}\sim\exp(\lambda+\mu),
  \end{equation*}
hence $\E Z = (\lambda+\mu)^{-1}$.

\hint{Use that if $Z=\min\{X, Y\}>x$ that then it must be that $X>x$
  and $Y>x$. Then use independence of $X$ and $Y$.}

  \begin{solution}
Use that $X$ and $Y$ are independent to get
    \begin{equation*}
      \begin{split}
      \P{Z>x} 
&= \P{\min{X,Y}>x} = \P{X>x\text{ and } Y>x} = \P{X>x}\P{Y>x} \\
&= e^{-\lambda x} e^{-\mu x} = e^{-(\lambda+\mu)x}.
      \end{split}
    \end{equation*}
  \end{solution}

\end{question}

\begin{question}\label{ex:3}
   If $A\sim \exp(\lambda)$, $S\sim\exp(\mu)$ and independent, show that 
    \begin{equation*}
      \P{A\leq S} = \frac{\lambda}{\lambda+\mu}.
    \end{equation*}

    \hint{ Define the joint distribution of $A$ and $S$ and carry out
      the computations, or use conditioning, or use the result of the
      previous exercise.}

\begin{solution}
There is more than one way to show that $\P{A\leq S} = \lambda/(\lambda+\mu)$.  

Method 1. (I admit that, although the simplest, least technical,
method, I did not think of this right away. I am `conditioned' to use
conditioning\ldots) Observe first that $A$ and $S$, being
exponentially distributed, both have a density. Moreover, as they are
independent, we can sensibly speak of the joint density
$f_{A,S}(x,y) = f_A(x)f_S(y) = \lambda \mu e^{-\lambda x} e^{-\mu
  y}$. With this,
\begin{equation*}
  \begin{split}
    \P{A\leq S} 
&= \E{\1{A\leq S}} \\
&= \int_0^\infty \int_0^\infty \1{x\leq y} f_{A,S}(x,y)\, \d y\,\d x\\
&= \lambda \mu \int_0^\infty \int_0^\infty \1{x\leq y} e^{-\lambda x} e^{-\mu y} \, \d y\,\d x\\
&= \lambda \mu \int_0^\infty e^{-\mu y} \int_0^y e^{-\lambda x}\, \d x \, \d y \\
&= \mu \int_0^\infty e^{-\mu y} (1-e^{-\lambda y})\,\d y\\
&= \mu \int_0^\infty (e^{-\mu y} - e^{-(\lambda +\mu)y} ) \,\d y\\
&= \mu \int_0^\infty (e^{-\mu y} - e^{-(\lambda +\mu)y} ) \,\d y\\
&= 1 - \frac{\mu}{\lambda + \mu} 
  \end{split}
\end{equation*}
This argument is provided in the probability book you use in the first year.

Method 2. Applying a standard conditioning argument
\begin{equation*}
\P{A\leq S} = \int_0^\infty \P{A\leq S| S=s} \mu e^{-\mu s} \,  \d s.
\end{equation*}
Now, $\P{A\leq S| S=s}$ is a conditional probability
distribution. This is a bit of tricky object, but very useful once you
get used to it. The tricky part is that $\P{S=s} = 0$. Therefore
$\P{A\leq S\given S=s}$ cannot be defined as $\frac{\P{A\leq s,
  S=s}}{\P{S=s}}$. However, if we proceed nonetheless and use the
independence of $S$ and $A$, we get
\begin{equation*}
   \P{A\leq S| S=s} = \frac{\P{A\leq s, S=s}}{\P{S=s}} = \frac{\P{A\leq s} \P{S=s}}{\P{S=s}} = \
\P{A\leq s}
 \end{equation*}
 and thus, indeed, $\P{A\leq S| S=s} = \P{A\leq s}$. Then,
\begin{equation*}
  \begin{split}
\P{A\leq S} 
&= \int_0^\infty \P{A\leq S| S=s} \mu e^{-\mu s} \,  \d s \\
&= \int_0^\infty \P{A\leq s} \mu e^{-\mu s} \,  \d s \\
&= \int_0^\infty (1-e^{-\lambda s})\mu e^{-\mu s} \,  \d s \\
  \end{split}
\end{equation*}
and we arrive at the integral we have seen above. 

So we get the correct answer, but by the wrong method.  How can we
repair this?  As a first step, let's not fix $S$ to a set of measure
zero, but let's assume that $S\in [s,t]$ for $s<t$. Then it follows
that
\begin{equation*}
  1\{A\leq s\} 1\{S\in[s,t]\} \leq  1\{A\leq S\} 1\{S\in[s,t]\} \leq 1\{A\leq t\} 1\{S\in[s,t]\}
\end{equation*}
As a second step,  using that $\P{S\in[s,t]} > 0$ if $s<t$ and the independence of $A$ and $S$,
\begin{equation*}
  \begin{split}
    \P{A\leq s} &= \frac{\P{A\leq s} \P{S\in[s,t]}}{\P{S\in[s,t]}} = \frac{\P{A\leq s,  S\in[s,t]}}{\P{S\in[s,t]}} \\
    \P{A\leq t} &= \frac{\P{A\leq t} \P{S\in[s,t]}}{\P{S\in[s,t]}} = \frac{\P{A\leq t,  S\in[s,t]}}{\P{S\in[s,t]}}
  \end{split}
\end{equation*}
Now with the result of the first step
\begin{equation*}
  \begin{split}
    \P{A\leq s} 
&= \frac{\P{A\leq s,  S\in[s,t]}}{\P{S\in[s,t]}} \\
&\leq \frac{\P{A\leq S,  S\in[s,t]}}{\P{S\in[s,t]}} \\
&= \P{A\leq S| S\in[s,t]} \\
&\leq \frac{\P{A\leq t,  S\in[s,t]}}{\P{S\in[s,t]}} \\
&= \P{A\leq t}.
  \end{split}
\end{equation*}
Hence, 
\begin{equation*}
  \begin{split}
    \P{A\leq s} \leq \P{A\leq S| S\in[s,t]} \leq \P{A\leq t}.
  \end{split}
\end{equation*}
Finally, taking the limit $t\downarrow s$, and defining $\P{A\leq S|
S = s} = \lim_{t\downarrow s} \P{A\leq S| S\in[s,t]}$, it follows
that
\begin{equation*}
    \P{A\leq s} = \P{A\leq S| S = s}
\end{equation*}

A more direct way to properly define $\P{A\leq S| S=s}$ is as follows. For any $y$ such that $f_S(y) > 0$, we can define the conditional probability density function of $A$, given that $S=s$, as 
\begin{equation*}
  f_{A|S}(x|s) = \frac{f_{A, S}( x, s)}{f_Y(s)},
\end{equation*}
where, as before, $f_{A, S}( x, s)$ is the joint density of $A$ and $S$. Now that the conditional probability density is defined, we can properly define
\begin{equation*}
  \E {A| S=s } = \int_0^\infty x   f_{A|S}(x|s)\, \d x
\end{equation*}
and also
\begin{equation*}
  \begin{split}
  \P{A\leq S| S=s } = \E{\1{A \leq S}| S=s } = \int_0^\infty \1{x \leq s}   f_{A|S}(x|s)\, \d x.
  \end{split}
\end{equation*}
Using the definition of $f_{A|S}(x|s)$ and the independence of $A$ and $S$ it follows that
\begin{equation*}
  f_{A|S}(x|s) = \frac{f_{A, S}( x, s)}{f_Y(s)} = \frac{\lambda e^{-\lambda x} \mu e^{-\mu s}}{\mu e^{-\mu s}} = \lambda e^{-\lambda x}
\end{equation*}
from which we get that 
\begin{equation*}
  \begin{split}
  \E{\1{A \leq S}| S=s } 
&= \int_0^\infty \1{x \leq s}   f_{A|S}(x|s)\, \d x \\
&= \int_0^\infty \1{x \leq s} \lambda e^{-\lambda x}\, \d x \\
&= \int_0^s \lambda e^{-\lambda x}\, \d x \\
&= 1 - e^{-\lambda s},
  \end{split}
\end{equation*}
that is,
\begin{equation*}
  \begin{split}
  \P{A\leq S| S=s} = \E{\1{A \leq S}| S=s } = 1 - e^{-\lambda s}  = \P{A \leq s}.
  \end{split}
\end{equation*}

All of these problems can be put on solid ground by using measure
theory. We do not pursue these matters any further, but trust on our
intuition that all is well.

\end{solution}
\end{question}


\begin{question}
  A machine serves two types of jobs. The processing time of jobs of
  type $i$, $i=1,2$, is exponentially distributed with parameter
  $\mu_i$. The type $T$ of job is random and independent of anything
  else, and such that $\P{T=1} = p = 1-q = 1-\P{T=2}$. (An example
  is a desk serving men and women, both requiring different average
  service times, and $p$ is the probability that the customer in
  service is a man.) What is the expected processing time and what is the variance? 
  \begin{solution}
    Let $X$ be the processing (or service) time at the server, and
    $X_i$ the service time of a type $i$ job. Then, 
    \begin{equation*}
      X = \1{T=1} X_1 + \1{T=2} X_2,
    \end{equation*}
    where $\1{}$ is the indicator function, that is, $\1{A}=1$ if the
    event $A$ is true, and $\1{A}=0$ if $A$ is not true. With this,
\begin{equation*}
  \begin{split}
  \E X 
&= \E{\1{T=1} X_1} + \E{\1{T=2} X_2} \\
&= \E{\1{T=1}} \E{ X_1} + \E{\1{T=2}} \E{X_2}, \text{ by the independence of $T$}, \\
&= \P{T=1} /\mu_1 + \P{T=2}/ \mu_2 \\
&= p /\mu_1 + q/ \mu_2 \\
&= p \E{X_1}  + q \E{X_2}.
  \end{split}
\end{equation*}
(The next derivation may seem a bit long, but the algebra is
standard. I include all steps so that you don't have to use pen and
paper yourself if you want to check the result.) Next, using that
\begin{equation*}
\1{T=1}\1{T=2} = 0 \text{ and } \1{T=1}^2 = \1{T=1},
\end{equation*}
we get
\begin{equation*}
  \begin{split}
  \V X 
&= \E{X^2} - (\E X)^2 \\
&= \E{\left(\1{T=1} X_1 + \1{T=2} X_2\right)^2} - \left(\frac{p}{\mu_1}+\frac{q}{\mu_2}\right)^2 \\
&= \E{\1{T=1} X_1^2 + \1{T=2} X_2^2} - \left(\frac{p}{\mu_1}+\frac{q}{\mu_2}\right)^2 \\ 
&= p \E{X_1^2} + q \E{X_2^2} - \left(\frac{p}{\mu_1}+\frac{q}{\mu_2}\right)^2 \\ 
&= p \V{X_1} + p (\E{X_1})^2 + q \V{X_2} + q(\E{ X_2})^2 - \left(\frac{p}{\mu_1}+\frac{q}{\mu_2}\right)^2 \\ 
&= p \V{X_1} + \frac{p}{\mu_1^2} + q \V{X_2} + \frac{q}{\mu_2^2} - \left(\frac{p}{\mu_1}+\frac{q}{\mu_2}\right)^2 \\ 
&= p \V{X_1} + q \V{X_2}
+ \frac{p}{\mu_1^2} + \frac{q}{\mu_2^2}
- \frac{p^2}{\mu_1^2}-\frac{q^2}{\mu_2^2}  -\frac{2pq}{\mu_1\mu_2}\\ 
&= p \V{X_1} + q \V{X_2}
+ \frac{p(1-p)}{\mu_1^2} + \frac{q(1-q)}{\mu_2^2}
-\frac{2pq}{\mu_1\mu_2}\\ 
&= p \V{X_1} + q \V{X_2}
+ \frac{pq}{\mu_1^2} + \frac{qp}{\mu_2^2}
-\frac{2pq}{\mu_1\mu_2}\\ 
&= p \V{X_1} + q \V{X_2}
+ pq(\E{X_1} - \E{X_2})^2.
\end{split}
\end{equation*}

Interestingly, we see that even if $\V X_1 = \V X_2 = 0$, $\V X > 0$
if $\E{X_1} \neq \E{X_2}$. Bear this in mind; we will use these ideas
later when we discuss the effects of failures on the variance of
service times of jobs.

\end{solution}
\end{question}

\begin{question}
  Try to make Figure~\ref{fig:uniformmany} with simulation.
  \begin{solution}
    The source code can be found in \texttt{progs/converge\_to\_exp.py}.
\end{solution}
\end{question}
  


%%% Local Variables:
%%% mode: latex
%%% TeX-master: "book"
%%% End:
