\section{Poisson Distribution}
\label{sec:poisson-distribution}

In this section we provide a derivation of, and motivation for, the
Poisson process, and clarify its relation with the exponential
distribution at the end.

Consider a machine that fails occasionally. Let us write $N(s, t)$ for
the number of failures occurring during a time interval of $(s,t]$. We
assume, without loss of generality, that repairs are
instantaneous. Clearly, as we do not know in advance how often the
machine will fail, we model $N(s,t)$ as a random variable for all
times $s$ and $t$.

Our first assumption is that the failure behavior of the machine does not
significantly change over time. Then it is reasonable to assume that
the expected number of failure is proportional to the  length of
the interval $T$. Thus, it is reasonable to assume that there exists some
constant $\lambda$ such that
\begin{equation}
  \label{eq:6}
 \E{N(s,t)} = \lambda (t-s)
\end{equation}
The constant $\lambda$ is often called the \recall{arrival rate}, or
failure rate in this case.


The second assumption is that $\{N(s,t), s\leq t\}$ has
\recall{stationary} and \emph{independent increments}. Stationarity
means that the distribution of the number of arrivals are the same for
all intervals of equal length. Formally, $N(s_1,t_1)$ has the same
distribution as $N(s_2, t_2)$ if $t_2-s_2 = t_1-s_1$. Independence
means, roughly speaking, that knowing that $N(s_1,t_1)= n$, does not
help to make any predictions about $N(s_2, t_2)$ if the intervals
$(s_1,t_1)$ and $(s_2, t_2)$ have no overlap.

To find the distribution of $N(0,t)$, let us split the interval
$[0,t]$ into $n$ sub-intervals, all of equal length, and ask: `What is
the probability that the machine will fail in some given
sub-interval.'  By our second assumption, the failure behavior is
constant over time. Therefore, the probability $p$ to fail in each
interval should be equal. Moreover, if $n$ is large, $p$ must be
small, for otherwise \eqref{eq:6} could not be true. As a consequence,
if the time intervals are very small, we can safely neglect the
probability that two or more failures occur in one such tiny interval.

As a consequence, then, we can model the occurrence of a failure in
some period $i$ as a Bernoulli distributed random variable $B_i$ such
that $\P{B_i =1}$ and $\P{B_i=0} = 1-\P{B_i = 1}$, and we assume that
$\{B_i\}$ are independent. The total number of failures $N_n(t)$ that
occur in $n$ intervals is then binomially distributed
\begin{equation}\label{eq:bin}
  \P{N_n(t) = k} = {n \choose k} p^k (1-p)^{n-k}.
\end{equation}
In the exercises we ask you to use this to motivate that, in some
appropriate sense, $N_n(t)$ converges to $N(t)$ for $n\to \infty$ such
that
\begin{equation}\label{eq:pois}
  \P{N(t) = k} = 
e^{-\lambda t} \frac{(\lambda t)^k}{k!}.
\end{equation}
We say that $N(t)$ is \recall{Poisson distributed} with rate
$\lambda$, and write $N(t)\sim P(\lambda t)$. 

Moreover, if there are no failures in some interval
$[0,t]$, then it must be that $N(t) = 0$ and $A_1$, i.e., the
occurance of the first failure, must be larger than $t$.  Therefore,
\begin{equation*}
  \P{A_1>t} = \P{X_1> t} = P(N(t) = 0) = e^{-\lambda t} \frac{(\lambda t)^0}{0!}= e^{-\lambda t}.
\end{equation*}

The relations we discussed above are of paramount importance in the
analysis of queueing process. We summarize this by a theorem.
\begin{theorem}\label{thr:1}
  A counting process $\{N(t)\}$ is a Poisson process with rate $\lambda$ if and only if
  the inter-arrival times $\{X_i\}$, i.e., the times between consecutive
  arrivals, are i.i.d. and $\P{ X_1\leq t} = 1 - e^{-\lambda t}$. In other words, $X_i\sim \exp(\lambda) \Leftrightarrow N(t) \sim P(\lambda t)$ 
\end{theorem}

\begin{question}
Show that $\E{N_n(t)} = \sum_{i=1}^n \E{B_i} = n p$.
\begin{solution}
  \begin{equation*}
    \E{N_n(t)} = \E{\sum_{i=1}^n {B_i}} = \sum_{i=1}^n \E{B_i} = n \E{B_i} = n p.
  \end{equation*}
\end{solution}
\end{question}


\begin{question}
What is the difference between $N_n(t)$ and $N(t)$?
\begin{solution}
  $N_n(t)$ is a binomially distributed random variable with parameters
  $n$ and $p$. The maximum value of $N_n(t)$ is $n$. The random
  variable $N(t)$ models the number of failures that can occur during
  $[0,t]$. As such it is not necessarily bounded by $n$. Thus, $N_n(t)$ and $N(t)$ cannot represent the same random variable. 
\end{solution}
\end{question}

\begin{question}
  Show how the binomial distribution~\eqref{eq:bin} converges to the
  Poisson distribution~\eqref{eq:pois} if $n\to\infty$, $p\to0$ such
  that $np=\lambda$.

  \begin{solution} Now we like to relate $N_n(t)$ and $N(t)$ It is
    clear that we at least want the expectations to be the same, that
    is, $np = \lambda t$. This implies that
\begin{equation*}
  p = \frac{\lambda t}n,
\end{equation*}
so that 
\begin{equation*}
  \P{N_n(t) = k} = {n \choose k} \left(\frac{\lambda t}{n}\right)^k \left(1-\frac{\lambda t}n\right)^{n-k}.
\end{equation*}
To see that 
\begin{equation*}\label{eq:52}
  \lim_{n\to\infty} {n \choose k} \left(\frac{\lambda t}{n}\right)^k \left(1-\frac{\lambda t}n\right)^{n-k} = e^{-\lambda t} \frac{(\lambda t)^k}{k!}.
\end{equation*}
use that
    \begin{equation*}
      \begin{split}
      {n \choose k} \left(\frac{\lambda t}{n}\right)^k \left(1-\frac{\lambda t}n\right)^{n-k} 
&= \frac{n!}{k!(n-k)!} \left(\frac{\lambda t}{n}\frac{n}{n-\lambda t}\right)^k \left(1-\frac{\lambda t}n\right)^{n} \\
&= \frac{(\lambda t)^k}{k!} \left(\frac n{n-\lambda t} \right)^k  \frac{n!}{n^k(n-k)!}\left(1-\frac{\lambda t}n\right)^{n}\\
&= \frac{(\lambda t)^k}{k!} \left(\frac n{n-\lambda t} \right)^k \frac{n}{n}\frac{n-1}{n}\cdots\frac{n-k+1}{n} \left(1-\frac{\lambda t}n\right)^{n}  \\
&\to \frac{(\lambda t)^k}{k!} e^{-\lambda t} \text{ when } n \to \infty, 
      \end{split}
    \end{equation*}
    where we use that $(1+x/n)^n\to e^{x}$ so that ,
    $\left(1-\frac{\lambda t}n\right)^{n} \to e^{-\lambda t}$.  The
    rest is easy.
  \end{solution}
\end{question}


\begin{question}
  If the inter-arrival times are i.i.d. and exponentially distributed
  with mean $1/\lambda$, prove that the number $N(t)$ of arrivals
  during interval $[0,t]$ is Poisson distributed.
    \begin{solution}
      We want to show that
    \begin{equation*}
      \P{N(t)=k} = e^{-\lambda t}\frac{(\lambda t)^k}{k!}.
    \end{equation*}
    Now observe that
    $\P{N(t)=k} = \P{A_k \leq t} - \P{A_{k+1} \leq t}$.  Using the
    density of $A_{k+1}$ as obtained previously and applying partial
    integration leads to
\begin{equation*}
  \begin{split}
\P{A_{k+1} \leq t} 
&= \lambda \int_0^t \frac{(\lambda s)^{k}}{k!}e^{-\lambda s}\, \d s \\
&= \lambda \frac{(\lambda s)^{k}}{k!}\frac{e^{-\lambda s}}{-\lambda} \Big|_{0}^t + \lambda \int_0^t \frac{(\lambda s)^{k-1}}{(k-1)!}e^{-\lambda s}\, \d s \\
&= - \frac{(\lambda t)^{k}}{k!} e^{-\lambda t} + \P{A_k \leq t}
  \end{split}
\end{equation*}
We are done.
    \end{solution}
\end{question}



\begin{question}
 Show from the above that  for small $h$, 
  \begin{enumerate}
  \item $\P{N(h) = n \given N(0) = n} = 1-\lambda h + o(h)$ 
  \item $\P{N(h) = n+1 \given N(0) = n} = \lambda h + o(h)$ 
  \item $\P{N(h) \geq n+2 \given N(0) = n} = o(h)$,
  \end{enumerate}
 where $o(h)$ is a function
    $f(h)$ such $f(h)/h \to 0$ as $h\to 0$.
\begin{solution}
\begin{enumerate}
  \item 
  $\P{N(h) = n | N(0) = n} = \P{N(h) = 0} = e^{-\lambda h} (\lambda
  h)^0/0! = e^{-\lambda h} = 1-\lambda h + o(h)$,
  as follows from a standard argument in analysis that
  $e^{-\lambda h} = 1 -\lambda h + o(h)$  for $h$ small. 
\item 
  $\P{N(h) = n +1 | N(0) = n} = \P{N(h) = 1} = e^{-\lambda h} (\lambda h)^1/1! = 
(1-\lambda h + o(h))\lambda h  = \lambda h - \lambda^2 h^2 + o(h) = \lambda h + o(h)$. 
\item 
  $\P{N(h) \geq n+2 | N(0) = n} = \P{N(h) \geq 2} = e^{-\lambda h} \sum_{i=2}^\infty \frac{(\lambda h)^i}{i!} = e^{-\lambda h}(e^{\lambda h} - 1 - \lambda h).$ Hence,
  \begin{equation*}
    \begin{split}
\P{N(h) \geq 2} &= 1 - e^{-\lambda h}(1 + \lambda h) = 1 - (1-\lambda h + o(h))(1+\lambda h) \\
&= 1 - (1-\lambda^2 h^2 + o(h)) = \lambda^2 h^2 + o(h) = o(h).
    \end{split}
  \end{equation*}
\end{enumerate}
\end{solution}
\end{question}


\begin{question}
  Assume a timer fires at times $0=T_0<T_1<T_2< \cdots$, such that
  $T_{k}-T_{k-1}\sim\exp(\lambda)$. Define
  $N(t) = \sum_{k=0}^\infty k \1{T_k \leq t < T_{k+1}}$, where $\1{}$
  is the \recall{indicator function}, that is, $\1{A}=1$ if the event
  $A$ is true, and $\1{A}=0$ if $A$ is not true. What is the
  distribution of $N(t)$?
\begin{solution}
$N(t) \sim \text{P}(\lambda t)$. 
\end{solution}
\end{question}


\begin{question} (Merged Poisson streams form a new Poisson process
  with the sum of the rates.) Assume that
  $N_a(t)\sim \text{P}(\lambda t)$ , $N_s(t) \sim \text{P}(\mu t)$ and
  independent.  Show that
  $N_a(t) + N_s(t) \sim \text{P}((\lambda + \mu)t)$.  

\hint{ Use a
    conditioning argument or use moment generating functions.}

    \begin{solution}
 We use conditioning.
  \begin{equation*}
    \begin{split}
\P{N_a(t) + N_s(t) = n} 
&= \sum_{i=0}^n \P{N_a(t) + N_s(t) = n|N_a(t) = i}\P{N_a(t)=i} \\
&= \sum_{i=0}^n \P{ N_s(t) = n-i}\P{N_a(t)=i} \\
&= \sum_{i=0}^n \frac{(\mu t)^{n-i}}{(n-i)!} \frac{(\lambda t)^i}{i!} e^{-(\mu+\lambda)t} \\
&= e^{-(\mu+\lambda)t} \sum_{i=0}^n \frac{(\mu t)^{n-i}}{(n-i)!} \frac{(\lambda t)^i}{i!}  \\
&= e^{-(\mu+\lambda)t}\frac 1{n!} \sum_{i=0}^n {n \choose i }(\mu t)^{n-i}(\lambda t)^i  \\
&= \frac{((\mu+\lambda)t))^n}{n!}e^{-(\mu+\lambda)t}.
    \end{split}
  \end{equation*}

  With moment generating functions.
      \begin{equation*}
        \begin{split}
          M_a(z) 
&= \E{z^{N_a(t)}} = \sum_{k=0}^\infty z^k \P{N_a(t) = k} \\
&= \sum_{k=0}^\infty z^k e^{-\lambda t} \frac{(\lambda t)^k}{k!} \\
&= e^{-\lambda t} \sum_{k=0}^\infty  \frac{(z \lambda t)^k}{k!} \\
&=\exp(\lambda t (z-1)). 
        \end{split}
      \end{equation*}
      Use this, and the similar expression for $N_s(t)$ and
      independence to see that
\begin{equation*}
M(z) = \E{z^{N(t)}} = \E{z^{N_a(t)}} \E{z^{N_s(t)}} = \exp((\lambda +\lambda)t (z-1)). 
\end{equation*}
Finally, since the moment generating function uniquely characterizes
the distribution, and since the above expression has the same form as
$M_1(z)$ but with $\lambda+\mu$ replacing $\lambda$, we can conclude that $N(t)\sim P((\lambda +\mu) t)$.
    \end{solution}
\end{question}


\begin{question} Assume that
  $N_a(t)\sim \text{P}(\lambda t)$ , $N_s(t) \sim \text{P}(\mu t)$ and
  independent. Use that  $N_a(t) + N_s(t) \sim \text{P}((\lambda + \mu)t)$ to conclude
    \begin{equation*}
    \P{N_a(h) = 1,  N_s(h) = 0 | N_a(h) + N_s(h) = 1} =
\frac{\lambda}{\lambda+\mu}.
    \end{equation*}
    Note that the right-hand-side does not depend on $h$, hence it
    holds for any time $h$, whether it is small or not.
    \begin{solution}
  With the above:
  \begin{equation*}
    \begin{split}
&    \P{N_a(h) = 1,  N_s(h) = 0 | N_a(h) + N_s(h) = 1} \\
&= \frac{\P{N_a(h) = 1,  N_s(h) = 0, N_a(h) + N_s(h) = 1} }{ \P{N_a(h) + N_s(h) = 1}} \\ 
&= \frac{\P{N_a(h) = 1,  N_s(h) = 0}}{ \P{N_a(h) + N_s(h) = 1}} \\ 
&= \frac{\P{N_a(h) = 1}\P{N_s(h) = 0}}{ \P{N_a(h) + N_s(h) = 1}} \\ 
&= \frac{\lambda h \exp{-\lambda h} \exp{-\mu h}}{((\lambda+\mu)h)\exp{(-(\lambda+\mu)h)}}\\
&= \frac{\lambda h \exp{(-(\lambda + \mu)h)}}{((\lambda+\mu)h)\exp{(-(\lambda+\mu)h)}}\\
&= \frac{\lambda}{\lambda+\mu}
    \end{split}
  \end{equation*}
    \end{solution}
\end{question}


\begin{question}
 Assume that $N_a(t)\sim \text{P}(\lambda t)$ ,
  $N_s(t) \sim \text{P}(\mu t)$ and independent.  What is actually the
  meaning of the event
  $\{N_a(h) = 1, N_s(h) = 0\}\cap\{N_a(h) + N_s(h) = 1\}$?
    \begin{solution}
 This means that, given that an event occurred, the event was an arrival, i.e., $N_a$ was the first.
    \end{solution}
\end{question}


\begin{question} (A Bernouilli-thinned Poisson process is still a
  Poisson process) Consider a Poisson process. Split the process in
  the following way. When a job arrives, throw a coin that lands heads
  with probability $p$ and tails with $q=1-p$. When the coin lands
  heads, call the job of type 1, otherwise of type 2.  Another way of
  thinning is by modeling the stream of people passing a shop as a
  Poisson process with rate $\lambda$. With probability $p$ a person
  decides, independent of anything else, to enter the shop.

  Show that the Poisson process obtained by thinning the original
  process is $\sim P(\lambda t p)$ for any $t$.
 
\hint{ Condition on the total number of arrivals $N(t)=m$ up to time
      $t$. Realize that the probability that a job is of type 1 is
      Bernouilli distributed, hence when you consider $m$ jobs in
      total, the number of type 1 jobs is binomially distributed.}

    \begin{solution}
\begin{equation*}
  \begin{split}
    \P{N(t) = n}
&= \sum_{m=0}^n \P{N(t) =n \given N_1(t) = m}\P{N_1(t) = m} \\
&= \sum_{m=0}^n \P{N_2(t) =n-m}\P{N_1(t) = m} \\
&= \sum_{m=0}^n 
e^{-\lambda_2 t}\frac{(\lambda_2 t)^{n-m}}{(m-n)!} 
e^{-\lambda_1 t}\frac{(\lambda_1 t)^{m}}{m!} \\
&= e^{-(\lambda_2+\lambda_1) t} \frac{1}{n!}\sum_{m=0}^n 
{n \choose m} (\lambda_2 t)^{n-m} (\lambda_1 t)^{m}\\
&= e^{-(\lambda_2+\lambda_1) t} \frac{1}{n!}(\lambda_2 t + \lambda_1 t)^n.
  \end{split}
\end{equation*}
We see that the thinned stream is Poisson with parameter $\lambda t p$. 
\end{solution}
\end{question}    


\begin{question}\label{ex:3}
   If $A\sim \exp(\lambda)$, $S\sim\exp(\mu)$ and independent, show that 
    \begin{equation*}
      \P{A\leq S} = \frac{\lambda}{\lambda+\mu}.
    \end{equation*}

    \hint{ Define the joint distribution of $A$ and $S$ and carry out
      the computations, or use conditioning, or use the result of the
      previous exercise.}

\begin{solution}
There is more than one way to show that $\P{A\leq S} = \lambda/(\lambda+\mu)$.  

Method 1. (I admit that, although the simplest, least technical,
method, I did not think of this right away. I am `conditioned' to use
conditioning\ldots) Observe first that $A$ and $S$, being
exponentially distributed, both have a density. Moreover, as they are
independent, we can sensibly speak of the joint density
$f_{A,S}(x,y) = f_A(x)f_S(y) = \lambda \mu e^{-\lambda x} e^{-\mu
  y}$. With this,
\begin{equation*}
  \begin{split}
    \P{A\leq S} 
&= \E{\1{A\leq S}} \\
&= \int_0^\infty \int_0^\infty \1{x\leq y} f_{A,S}(x,y)\, \d y\,\d x\\
&= \lambda \mu \int_0^\infty \int_0^\infty \1{x\leq y} e^{-\lambda x} e^{-\mu y} \, \d y\,\d x\\
&= \lambda \mu \int_0^\infty e^{-\mu y} \int_0^y e^{-\lambda x}\, \d x \, \d y \\
&= \mu \int_0^\infty e^{-\mu y} (1-e^{-\lambda y})\,\d y\\
&= \mu \int_0^\infty (e^{-\mu y} - e^{-(\lambda +\mu)y} ) \,\d y\\
&= \mu \int_0^\infty (e^{-\mu y} - e^{-(\lambda +\mu)y} ) \,\d y\\
&= 1 - \frac{\mu}{\lambda + \mu} 
  \end{split}
\end{equation*}
This argument is provided in the probability book you use in the first year.

Method 2. Applying a standard conditioning argument
\begin{equation*}
\P{A\leq S} = \int_0^\infty \P{A\leq S| S=s} \mu e^{-\mu s} \,  \d s.
\end{equation*}
Now, $\P{A\leq S| S=s}$ is a conditional probability
distribution. This is a bit of tricky object, but very useful once you
get used to it. The tricky part is that $\P{S=s} = 0$. Therefore
$\P{A\leq S\given S=s}$ cannot be defined as $\frac{\P{A\leq s,
  S=s}}{\P{S=s}}$. However, if we proceed nonetheless and use the
independence of $S$ and $A$, we get
\begin{equation*}
   \P{A\leq S| S=s} = \frac{\P{A\leq s, S=s}}{\P{S=s}} = \frac{\P{A\leq s} \P{S=s}}{\P{S=s}} = \
\P{A\leq s}
 \end{equation*}
 and thus, indeed, $\P{A\leq S| S=s} = \P{A\leq s}$. Then,
\begin{equation*}
  \begin{split}
\P{A\leq S} 
&= \int_0^\infty \P{A\leq S| S=s} \mu e^{-\mu s} \,  \d s \\
&= \int_0^\infty \P{A\leq s} \mu e^{-\mu s} \,  \d s \\
&= \int_0^\infty (1-e^{-\lambda s})\mu e^{-\mu s} \,  \d s \\
  \end{split}
\end{equation*}
and we arrive at the integral we have seen above. 

So we get the correct answer, but by the wrong method.  How can we
repair this?  As a first step, let's not fix $S$ to a set of measure
zero, but let's assume that $S\in [s,t]$ for $s<t$. Then it follows
that
\begin{equation*}
  1\{A\leq s\} 1\{S\in[s,t]\} \leq  1\{A\leq S\} 1\{S\in[s,t]\} \leq 1\{A\leq t\} 1\{S\in[s,t]\}
\end{equation*}
As a second step,  using that $\P{S\in[s,t]} > 0$ if $s<t$ and the independence of $A$ and $S$,
\begin{equation*}
  \begin{split}
    \P{A\leq s} &= \frac{\P{A\leq s} \P{S\in[s,t]}}{\P{S\in[s,t]}} = \frac{\P{A\leq s,  S\in[s,t]}}{\P{S\in[s,t]}} \\
    \P{A\leq t} &= \frac{\P{A\leq t} \P{S\in[s,t]}}{\P{S\in[s,t]}} = \frac{\P{A\leq t,  S\in[s,t]}}{\P{S\in[s,t]}}
  \end{split}
\end{equation*}
Now with the result of the first step
\begin{equation*}
  \begin{split}
    \P{A\leq s} 
&= \frac{\P{A\leq s,  S\in[s,t]}}{\P{S\in[s,t]}} \\
&\leq \frac{\P{A\leq S,  S\in[s,t]}}{\P{S\in[s,t]}} \\
&= \P{A\leq S| S\in[s,t]} \\
&\leq \frac{\P{A\leq t,  S\in[s,t]}}{\P{S\in[s,t]}} \\
&= \P{A\leq t}.
  \end{split}
\end{equation*}
Hence, 
\begin{equation*}
  \begin{split}
    \P{A\leq s} \leq \P{A\leq S| S\in[s,t]} \leq \P{A\leq t}.
  \end{split}
\end{equation*}
Finally, taking the limit $t\downarrow s$, and defining $\P{A\leq S|
S = s} = \lim_{t\downarrow s} \P{A\leq S| S\in[s,t]}$, it follows
that
\begin{equation*}
    \P{A\leq s} = \P{A\leq S| S = s}
\end{equation*}

A more direct way to properly define $\P{A\leq S| S=s}$ is as follows. For any $y$ such that $f_S(y) > 0$, we can define the conditional probability density function of $A$, given that $S=s$, as 
\begin{equation*}
  f_{A|S}(x|s) = \frac{f_{A, S}( x, s)}{f_Y(s)},
\end{equation*}
where, as before, $f_{A, S}( x, s)$ is the joint density of $A$ and $S$. Now that the conditional probability density is defined, we can properly define
\begin{equation*}
  \E {A| S=s } = \int_0^\infty x   f_{A|S}(x|s)\, \d x
\end{equation*}
and also
\begin{equation*}
  \begin{split}
  \P{A\leq S| S=s } = \E{\1{A \leq S}| S=s } = \int_0^\infty \1{x \leq s}   f_{A|S}(x|s)\, \d x.
  \end{split}
\end{equation*}
Using the definition of $f_{A|S}(x|s)$ and the independence of $A$ and $S$ it follows that
\begin{equation*}
  f_{A|S}(x|s) = \frac{f_{A, S}( x, s)}{f_Y(s)} = \frac{\lambda e^{-\lambda x} \mu e^{-\mu s}}{\mu e^{-\mu s}} = \lambda e^{-\lambda x}
\end{equation*}
from which we get that 
\begin{equation*}
  \begin{split}
  \E{\1{A \leq S}| S=s } 
&= \int_0^\infty \1{x \leq s}   f_{A|S}(x|s)\, \d x \\
&= \int_0^\infty \1{x \leq s} \lambda e^{-\lambda x}\, \d x \\
&= \int_0^s \lambda e^{-\lambda x}\, \d x \\
&= 1 - e^{-\lambda s},
  \end{split}
\end{equation*}
that is,
\begin{equation*}
  \begin{split}
  \P{A\leq S| S=s} = \E{\1{A \leq S}| S=s } = 1 - e^{-\lambda s}  = \P{A \leq s}.
  \end{split}
\end{equation*}

All of these problems can be put on solid ground by using measure
theory. We do not pursue these matters any further, but trust on our
intuition that all is well.


Finally, using the results of the previous exercise, note that the
event $\{A<S\}$ is equal to the event
$\{N_a(t) = 1, N_s(t)=0\}\cap\{N_a(t) + N_s(t)=1\}$. 

\end{solution}
\end{question}


\begin{question}
  Show that $\E{N(t)} = \lambda t$ and $\V{N(t)} = \lambda t$. Why is
  the SCV of $N(t)$ equal to $1/\lambda t$? Conclude that the relative
  variability of $N(t)$ becomes smaller as $t$ becomes larger. Observe
  that the SCV of a Poisson distributed random variable is not the
  same as the SCV of an exponentially distribued random variable.

  \hint{You might use generating functions here.}

  \begin{solution} I expect that this is easy for you by now. The
    exercise is just meant to help you recall this.

    When a random variable $N$ is Poisson distributed with parameter
    $\lambda$,
    \begin{equation*}
      \begin{split}
      \E N 
&= \sum_{n=0}^\infty n e^{-\lambda}\frac{\lambda^n}{n!}  \\
&= \sum_{n=1}^\infty n e^{-\lambda}\frac{\lambda^n}{n!}, \text{ since the term with $n=0$ cannot contribute} \\
&= e^{-\lambda} \lambda \sum_{n=1}^\infty \frac{\lambda^{n-1}}{(n-1)!} \\
&= e^{-\lambda} \lambda \sum_{n=0}^\infty \frac{\lambda^{n}}{n!}, \text{ by a change of variable}\\
&= e^{-\lambda} \lambda e^{\lambda} \\
&= \lambda.
      \end{split}
    \end{equation*}
Similarly, using that $\V N = \E{N^2} - (\E N)^2$, 
    \begin{equation*}
      \begin{split}
      \E{N^2}
&= \sum_{n=0}^\infty n^2 e^{-\lambda}\frac{\lambda^n}{n!}  \\
&= e^{-\lambda} \sum_{n=1}^\infty n \frac{\lambda^n}{(n-1)!}  \\
&= e^{-\lambda} \sum_{n=0}^\infty (n+1) \frac{\lambda^{n+1}}{n!}  \\
&= e^{-\lambda} \lambda \sum_{n=0}^\infty n \frac{\lambda^{n}}{n!}  +e^{-\lambda}\lambda \sum_{n=0}^\infty\frac{\lambda^{n}}{n!}  \\
&= \lambda^2  + \lambda.
\end{split}
\end{equation*}

The Poisson \emph{process} $\{N(t)\}$ is a much more complicated
object than the Poisson distributed random variable $N(t)$. The
process contains it is an \emph{ uncountable set} of random variables,
whereas $N(t)$ is just \emph{one} random variable. However, for a
fixed $t$ the element $N(t)$ is a random variable that is Poisson
distributed with parameter $\lambda t$. Using the above, the answer of
the question follows immediately.


With generating functions we get the result without too much
effort. Suppose $N$ is a Poisson distributed random variable. Then 
\begin{equation*}
  \begin{split}
\phi(z)
&=  \E{z^N} \\
&= \sum_{k=0}^\infty z^k \P{N=k} \\
&= \sum_{k=0}^\infty z^k \frac{\lambda^k}{k!} e^{-\lambda}  \\
&= e^{-\lambda} \sum_{k=0}^\infty  \frac{(z\lambda^k}{k!}  \\
&= e^{-\lambda} e^{z \lambda} \\
&= e^{(z-1)\lambda}.
  \end{split}
\end{equation*}
Then $\E N = \d \phi(z)/\d z$ at $z=1$. Now, $\phi'(1) = \lambda$. Also $\E{N(N-1)}= \phi''(1) = \lambda^2$. From this we see that $\E{N^2} = \phi''(1) + \phi(1) = \lambda^2 + \lambda$. 

The SCV is defined as $\V{N(t)}/(\E{N(t)})^2$. From the above it
follows that this is $(\lambda t)/(\lambda t)^2 = 1/\lambda t$.
Clearly, as $t$
increases, $1/\lambda t$ decreases.



Finally, the SCV of an exponentially distributed random variable is
1. The SCV of the related Poisson process is \emph{not} identically~1
for all $t$.

\end{solution}
\end{question}



%%% Local Variables:
%%% mode: latex
%%% TeX-master: "book"
%%% End:
