\section{Service Interruptions}
\label{sec:serv-interr}


\subsection*{Theory and Exercises}

\Opensolutionfile{hint}
\Opensolutionfile{ans}


See the related section 1.11 of the book of Zijm.

\begin{exercise}
  Determine the expectation and variance of $S_N = \sum_{i=1}^N X_i$
  where $N$ is a random variable (with finite second moment) and
  $\{X_i\}$ are i.i.d. as the generic random variable $X$.
\begin{solution}
We first consider the
expectation.  If it is given that $N=n$, then
\begin{equation*}
  \E{S_N \given N=n} =  \E{\sum_{i=1}^n X_i}= \E{X_1} + \E{X_2} + \cdots \E{X_n} = n \E X,
\end{equation*}
where the last equation follows from the fact that the $X_i$ have the same distribution. With this, 
\begin{equation*}
  \E{\sum_{i=1}^N X_i} =   \E{\E{\sum_{i=1}^N X_i \given N}} =  \E{N\E{X}} = \E X \E N.
\end{equation*}
This result in known as Wald's equation. 

To compute the variance we use that $\V{S_N} =\E{S_N^2} - (\E{S_N})^2$. Thus, we first compute
\begin{equation*}
  \begin{split}
  \E{ S_N^2 \given  N=n} 
&=  \E{\left(\sum_{i=1}^n X_i\right)^2}  
=  \E{\sum_{i=1}^n X_i^2 + \sum_{i\neq j}^n X_i X_j}  \\
&= \sum_{i=1}^n \E{X^2_i} + \sum_{i \neq j} \E{X_i} \E{X_j},
  \end{split}
\end{equation*}
since $\E{X_iX_j} = \E{X_i} \E{X_j}$ by independence. Therefore, using
that the $\{X_i\}$ are i.i.d. as $X$,
\begin{equation*}
  \E{ S_N^2 \given N=n} = n \E{X^2} + n(n-1) (\E X)^2. 
\end{equation*}
Thus, interpreting $\E{S_N^2\given N}$ as a function of $N$, we get
\begin{equation*}
  \E{ S_N^2 \given N} = N \E{X^2} + N(N-1) (\E X)^2,
\end{equation*}
from which follows that

\begin{equation*}
  \begin{split}
\E{ S_N^2}  &= \E{\E{ S_N^2 \given N}}  \\
&= \E{N \E{X^2} + N(N-1)(\E X)^2 } \\
& = \E N \E{ X^2} + \E{N^2} (\E X)^2 - \E N (\E X)^2 \\
& = \E N (\E{ X^2}  - (\E X)^2) + \E{N^2} (\E X)^2 \\
& = \E N \V X + \E{N^2} (\E X)^2.
  \end{split}
\end{equation*}
And finally,
\begin{equation}
  \begin{split}
  \V{S_N} 
&= \E{S_N^2}  - (\E{S_N})^2 \\
&= \E N \V X + \E{N^2} (\E X)^2   - (\E N)^2 (\E X)^2 \\
&= \E N \V X + \V N (\E X)^2.
  \end{split}
\end{equation}

\end{solution}
\end{exercise}


\begin{exercise} 

  A small company receives orders at a rate $\lambda$.  Interarrival
  times between orders are exponentially distributed. Once $B$ orders
  have been assembled, the company starts produces this batch of $B$
  jobs.  The average service time of an order is $1/\mu$. Between two
  batches the company has to re-adjust some of the machines before it
  can start processing another batch. These setup times form an
  i.i.d. sequence of times with average $\E{S}$ time units.

\begin{enumerate}
\item Should the batch size be limited from above or below to ensure that the system remains stable? 
\item How many orders arrive on average during a setup?
\item Use the above two points to determine a criterion on the minimal
  batch size $B$.
\end{enumerate}
\begin{hint}
1: If the batch sizes $B$ are small, relatively much time is
  spent on setups. 2: If the setup time $S$ is constant, then the
  number of orders that arrive is Poisson distributed. 3: Given a
  batch size $B$, find the time to produce such a batch, then include
  a setup time $S$ to compute the length of one production cycle. The
  expected number of jobs served during such a cycle must not exceed the number of jobs served during the cycle.
\end{hint}
\begin{solution}
\begin{enumerate}
\item Only from below. If batches are very large, the fraction of time spent on setups becomes very small.
\item The expected number of arrivals is 
\begin{equation*}
    \lambda \E S.
  \end{equation*}
To see this, note that  the number of arrivals $N$ during a constant time is Poisson. Thus,
\begin{equation*}
  \E{N} = \E{ \E{N\given S}} = \E{ \E{\lambda S\given S}} = \lambda \E{\E{S \given S}} = \lambda \E{S}.
\end{equation*}

\item  The number of jobs that arrive, on average, during one
  production cycle must be smaller than the total amount of jobs that
  can be served, on average, during one cycle.   The number of arrivals during the setup is $\lambda \E{S}$. The number of arrivals during serving the batch is $\lambda B/\mu$. Thus, 
  \begin{equation*}
    \lambda ( B/\mu + \E{S}) \leq B.
  \end{equation*}
  When equality holds here, the system is critically loaded. As we
  discussed before, that it not a good idea.
\end{enumerate}
\end{solution}
\end{exercise}

\begin{exercise}
Zijm.Ex.1.11.1
 \begin{solution}
Yes. The availability is $93\%$. Since $\E{S_e} = \E{S_0}/A$, and if $\E{S_0} $ is known, $\E{S_e}$ follows.
\end{solution}
\end{exercise}

\begin{exercise}
Zijm.Ex.1.11.2
 \begin{solution}
   No, from Zijm.Eq.1.51, the average repair time has to be known. The
   repair time here is of course the time it takes for a mechanic to
   show up at work again.
\end{solution}
\end{exercise}
\begin{exercise}
Zijm.Ex.1.11.3
 \begin{solution}
   It might, but perhaps a normal distribution would be better. It
   makes sense to make a histogram of the recover times to see whether
   some clear pattern is present. 

   Besides this, I don't know how sensitive the result for $C_e^2$ is
   on the distribution of the repair times. Perhaps it is not that
   sensitive, so in that case it would be ok to simply use the
   exponential distribution. 

The sensitivity study would be an interesting topic for simulation. 
\end{solution}
\end{exercise}
\begin{exercise}
Zijm.Ex.1.11.4 and 1.11.5
 \begin{solution}
   We need to make an assumption about the distribution of the repair
   times. Inferring from the text, the repair time is always $2$
   days. Lets also assume that all jobs accumulate in front of the
   broken machine, in other words, the broken machine is part of the
   job routing of each job. Then
   \begin{equation*}
     \P{N(2) > 20} = 
\sum_{n=21}^\infty e^{-5\cdot 2} \frac{10^n}{n!}.
1  - \sum_{n=0}^{20} e^{-5\cdot 2} \frac{10^n}{n!} = 1-0.9984,
   \end{equation*}
i.e., very small.  This is the code I used: 

<<term=True>>=
from math import exp, factorial
exp(-10) * sum((10)**n / factorial(n) for n in range(21))
@
\end{solution}
\end{exercise}

\begin{exercise}
Zijm.Ex.1.11.4

 \begin{solution}
If the shop already contains 10 jobs, somewhere upstream of the broken machine, then 
   \begin{equation*}
     \P{N(2) > 10} = 1- 0.58.
   \end{equation*}
where  I used
<<term=True>>=
exp(-10) * sum((10)**n / factorial(n) for n in range(11))
@
\end{solution}
\end{exercise}

\begin{exercise}
Zijm.Ex.1.11.6
 \begin{solution}
   Cleaning times will be pretty constant. Changing dies, or other
   machine parts, is also typically quite predictable, although it can
   take a lot of time, in particular in case a crane or other heavy
   machinery is needed to replace parts. If the machine require
   temporary adjustments, then the variation in setup times may be
   quite a bit higher.
\end{solution}
\end{exercise}

\begin{exercise}
Zijm.Ex.1.11.7
 \begin{solution}
   Then the effective service times, and in particular, $C_e^2$ will
   be quite a bit bigger. It is preferable to avoid such a situation. 

   Mathematically, it is only given that $N_s$ is a random
   variable. As, however, this does not state anything about its
   distribution, we cannot make any general claim. The intent of the
   problem is to have you check the relevant formulas and notice that
   the variance of $N_s$ appears in the formulas.
\end{solution}
\end{exercise}

\begin{comment}
  
\begin{exercise}
  Suppose a single machine workstation has to process $n$ different
  product families. Jobs of family $i$ arrive as a Poisson process
  with rate $\lambda_i$.  Service times are exponentially distributed
  with average $\mu_i^{-1}$ for family $i$. A setup of time $S$ is
  required if the machine switches from one family to another. If the
  machine works according to a cyclic schedule, i.e., produce family
  1, then 2, then 3, and so on until family $n$, Can you find batch
  sizes $B_i$ for family $i$ such that an average waiting time can be
  guaranteed?
  \begin{solution}
    \TBD. 

    This problem can be generalized and has been widely studied as one
    of the core problems in stochastic machine scheduling. For
    instance, what to do if the setup times are family-dependent, or
    even more general, dependent on the sequence in which the families
    are planned? This is a very hard problem. 
  \end{solution}
\end{exercise}
\end{comment}





\Closesolutionfile{hint}
\Closesolutionfile{ans}
\subsection*{Hints}
\input{hint}
\subsection*{Solutions}
\input{ans}


%%% Local Variables:
%%% mode: latex
%%% TeX-master: "notes_all"
%%% End:
